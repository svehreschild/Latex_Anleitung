\begin{table}[H]
    \centering
    \renewcommand{\arraystretch}{1.3}
    \begin{tabular}{lll}
        \toprule
        \textbf{Formatierung}                & \textbf{LaTeX-Code}                                 & \textbf{Beispiel}        \\
        \midrule
        Fette Symbole (kann mehr als mathbf) & \texttt{\textbackslash bm\{\textbackslash varphi\}} & \( \bm{\varphi} \)       \\
        Fette Buchstaben                     & \texttt{\textbackslash mathbf\{x\}}                 & \( \mathbf{x} \)         \\
        Serifenlose Schrift                  & \texttt{\textbackslash mathsf\{x\}}                 & \( \mathsf{x} \)         \\
        Monospace (Typewriter)               & \texttt{\textbackslash mathtt\{x\}}                 & \( \mathtt{x} \)         \\
        Kursive Schrift                      & \texttt{\textbackslash mathit\{x\}}                 & \( \mathit{x} \)         \\
        Normale Schrift (Roman)              & \texttt{\textbackslash mathrm\{sin\}}               & \( \mathrm{sin} x \)     \\
        \midrule
        Durchgestrichen                      & \texttt{\textbackslash cancel\{x\}}                 & \( \cancel{x} \)         \\
        Unterstrichen                        & \texttt{\textbackslash underline\{x\}}              & \( \underline{x} \)      \\
        Überstrichen                         & \texttt{\textbackslash overline\{x\}}               & \( \overline{x} \)       \\
        \midrule
        Text in Mathe-Umgebungen             & \texttt{\textbackslash text\{Text\}}                & \( \text{Text} \)        \\
        \midrule
        Farbige Schrift (rot)                & \texttt{\textbackslash textcolor\{red\}\{x\}}       & \( \textcolor{red}{x} \) \\
        Farbige Gleichung (blau)             & \texttt{\textbackslash color\{blue\} x = 5}         & \( \color{blue} x = 5 \) \\
        \midrule
        Leerzeichen                          & \texttt{a b}                                        & \( a  b \)               \\
        Kleiner Abstand                      & \texttt{a\textbackslash ,b}                         & \( a\,b \)               \\
        Mittlerer Abstand                    & \texttt{a\textbackslash :b}                         & \( a\: b \)              \\
        Großer Abstand                       & \texttt{a\textbackslash ;b}                         & \( a\;b \)               \\
        Sehr großer Abstand                  & \texttt{a\textbackslash quad b}                     & \( a \quad b \)          \\
        Extra großer Abstand                 & \texttt{a\textbackslash qquad b}                    & \( a \qquad b \)         \\
        Negativer Abstand                    & \texttt{a\textbackslash!b}                          & \( a\!b \)               \\
        \bottomrule
    \end{tabular}
    \caption{Formatierungsoptionen in Mathematikumgebungen}
    \label{tab:math_formatierung}
\end{table}
