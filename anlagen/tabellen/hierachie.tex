\begin{table}[H]
    \centering
    \begin{tabular}{lp{7cm}l}
        \toprule
        \textbf{Befehl}                           & \textbf{Ebenen-Tiefe}                       & \textbf{Verfügbar in} \\
        \midrule
        \texttt{\textbackslash part\{\}}          & Höchste Ebene                               & \texttt{report, book} \\
        \texttt{\textbackslash chapter\{\}}       & Kapitel                                     & \texttt{report, book} \\
        \texttt{\textbackslash section\{\}}       & Hauptabschnitt                              & Alle Dokumentklassen  \\
        \texttt{\textbackslash subsection\{\}}    & Unterabschnitt                              & Alle Dokumentklassen  \\
        \texttt{\textbackslash subsubsection\{\}} & Noch kleinere Ebene                         & Alle Dokumentklassen  \\
        \texttt{\textbackslash paragraph\{\}}     & Absatz mit Überschrift (keine Nummerierung) & Alle Dokumentklassen  \\
        \texttt{\textbackslash subparagraph\{\}}  & Noch kleiner, meist inline                  & Alle Dokumentklassen  \\
        \bottomrule
    \end{tabular}
    \caption{Dokumentenhierarchie}
    \label{tab:dokumenten_hierarchie}
\end{table}
