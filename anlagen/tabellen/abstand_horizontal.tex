\begin{table}[H]
    \centering
    \begin{tabular}{lp{10cm}}
        \toprule
        \textbf{Befehl}                        & \textbf{Wirkung}                                                  \\
        \midrule
        \texttt{\textbackslash hspace\{1cm\}}  & Fügt \textbf{1 cm} horizontalen Abstand ein                       \\
        \texttt{\textbackslash hspace*\{1cm\}} & Fügt \textbf{1 cm} horizontalen Abstand ein, auch am Zeilenanfang \\
        \texttt{\textbackslash kern Xpt}       & Fügt \textbf{X pt Abstand} zwischen Zeichen ein                   \\
        \texttt{\textbackslash quad}           & Fügt \textbf{einfachen Tabulator-Abstand} (1 em) ein              \\
        \texttt{\textbackslash qquad}          & Fügt \textbf{doppelten Tabulator-Abstand} (2 em) ein              \\
        \texttt{\textbackslash enspace}        & Fügt \textbf{halben Leerzeichen-Abstand} (0.5 em) ein             \\
        \texttt{\textbackslash hfill}          & Füllt den \textbf{gesamten verfügbaren horizontalen Platz}        \\
        \texttt{\textbackslash hfil}           & Fügt flexiblen Abstand ein                                        \\
        \midrule
        \texttt{\textbackslash !}              & Entfernt ein \textbf{kleines Leerzeichen} (negativer Abstand)     \\
        \texttt{\textbackslash ignorespaces}   & Entfernt \textbf{alle nachfolgenden Leerzeichen}                  \\
        \bottomrule
    \end{tabular}
    \caption{horizontale Abstandsoptionen}
    \label{tab:horizontale_abstaende}
\end{table}