\begin{table}[H]
    \centering
    \renewcommand{\arraystretch}{1.3}
    \begin{longtable}{>{\raggedright}p{4cm} >{\raggedright}p{6cm} >{\raggedright}p{5cm}}
        \caption{Wichtige Rechenoperatoren und mathematische Symbole}
        \toprule
        \textbf{Operator}                                    & \textbf{LaTeX-Code}                                                                     & \textbf{Beispiel}                    \\
        \hline
        \endfirsthead
        \toprule
        \textbf{Operator}                                    & \textbf{LaTeX-Code}                                                                     & \textbf{Beispiel}                    \\
        \midrule
        \endhead

        \bottomrule
        \hline
        \endfoot
        \textbf{Gleichungen und Relationen}                  &                                                                                         &                                      \\
        Gleichheit                                           & \texttt{=}                                                                              & $ a = b $                            \\
        Ungleichheit                                         & \texttt{\textbackslash neq}                                                             & $ a \neq b $                         \\
        Größer als                                           & \texttt{>}                                                                              & $ a > b $                            \\
        Kleiner als                                          & \texttt{<}                                                                              & $ a < b $                            \\
        Größer-gleich                                        & \texttt{\textbackslash geq}                                                             & $ a \geq b $                         \\
        Kleiner-gleich                                       & \texttt{\textbackslash leq}                                                             & $ a \leq b $                         \\

        \textbf{Grundrechenarten}                            &                                                                                         &                                      \\
        Plus                                                 & \texttt{+}                                                                              & $ a + b $                            \\
        Minus                                                & \texttt{-}                                                                              & $ a - b $                            \\
        Mal (×)                                              & \texttt{\textbackslash times}                                                           & $ a \times b $                       \\
        Mal (Punkt)                                          & \texttt{\textbackslash cdot}                                                            & $ a \cdot b $                        \\
        Geteilt (Bruch)                                      & \texttt{\textbackslash frac\{a\}\{b\}}                                                  & $ \frac{a}{b} $                      \\
        Geteilt (÷)                                          & \texttt{\textbackslash div}                                                             & $ a \div b $                         \\
        Prozent                                              & \texttt{\%}                                                                             & $ 50\% $                             \\
        Potenz                                               & \texttt{a\^{}b}                                                                         & $ a^b $                              \\
        Index                                                & \texttt{a\_\{b\}}                                                                       & $ a_b $                              \\
        Modulo                                               & \texttt{a \textbackslash bmod b}                                                        & $ a \bmod b $                        \\

        \textbf{Summen, Produkte, Integrale und Ableitungen} &                                                                                         &                                      \\
        Summenzeichen                                        & \texttt{\textbackslash sum\_\{i=1\}\^{}\{n\} ...}                                       & $ \sum_{i=1}^{n} ... $               \\
        Produktzeichen                                       & \texttt{\textbackslash prod\_\{i=1\}\^{}\{n\} ...}                                      & $ \prod_{i=1}^{n} ... $              \\
        Integral                                             & \texttt{\textbackslash int\_\{a\}\^{}\{b\} f(x) \textbackslash, dx}                     & $ \int_{a}^{b} f(x) \,dx $           \\
        Ableitung                                            & \texttt{\textbackslash frac\{d\}\{dx\} f(x)}                                            & $ \frac{d}{dx} f(x) $                \\
        Ableitung (partiell)                                 & \texttt{\textbackslash frac\{\textbackslash partial\}\{\textbackslash partial x\} f(x)} & $ \frac{\partial}{\partial x} f(x) $ \\
        Ableitung (nach der Zeit)                            & \texttt{\textbackslash dot\{x\}}                                                        & $ \dot{x} $                          \\

        \textbf{Funktionen}                                  &                                                                                         &                                      \\
        Betrag                                               & \texttt{\textbackslash left| x \textbackslash right|}                                   & $ \left| x \right| $                 \\
        Wurzel                                               & \texttt{\textbackslash sqrt\{x\}}                                                       & $ \sqrt{x} $                         \\
        n-te Wurzel                                          & \texttt{\textbackslash sqrt[n]\{x\}}                                                    & $ \sqrt[n]{x} $                      \\
        Logarithmus                                          & \texttt{\textbackslash log{(x)}}                                                        & $ \log{(x)} $                        \\
        Natürlicher Logarithmus                              & \texttt{\textbackslash ln{(x)}}                                                         & $ \ln{(x)} $                         \\
        Exponentialfunktion                                  & \texttt{e\^{}x}, \texttt{\textbackslash exp{(x)}}                                       & $ e^x, \exp{(x)} $                   \\

        \textbf{Trigonometrie}                               &                                                                                         &                                      \\
        Sinus                                                & \texttt{\textbackslash sin{(x)}}                                                        & $ \sin{(x)} $                        \\
        Kosinus                                              & \texttt{\textbackslash cos{(x)}}                                                        & $ \cos{(x)} $                        \\
        Tangens                                              & \texttt{\textbackslash tan{(x)}}                                                        & $ \tan{(x)} $                        \\
        Cotangens                                            & \texttt{\textbackslash cot{(x)}}                                                        & $ \cot{(x)} $                        \\
        Arkussinus                                           & \texttt{\textbackslash arcsin{(x)}}                                                     & $ \arcsin{(x)} $                     \\
        Arkuskosinus                                         & \texttt{\textbackslash arccos{(x)}}                                                     & $ \arccos{(x)} $                     \\
        Arkustangens                                         & \texttt{\textbackslash arctan{(x)}}                                                     & $ \arctan{(x)} $                     \\

        \textbf{Logik- und Mengenlehre}                      &                                                                                         &                                      \\
        Konjunktion (UND)                                    & \texttt{\textbackslash wedge}                                                           & $ A \wedge B $                       \\
        Disjunktion (ODER)                                   & \texttt{\textbackslash vee}                                                             & $ A \vee B $                         \\
        Negation                                             & \texttt{\textbackslash neg}                                                             & $ \neg A $                           \\
        Implikation                                          & \texttt{\textbackslash Rightarrow}                                                      & $ A \Rightarrow B $                  \\
        Äquivalenz                                           & \texttt{\textbackslash Leftrightarrow}                                                  & $ A \Leftrightarrow B $              \\

        \textbf{Zahlenmengen}                                &                                                                                         &                                      \\
        Natürliche Zahlen                                    & \texttt{\textbackslash mathbb\{N\}}                                                     & \( \mathbb{N} \)                     \\
        Ganze Zahlen                                         & \texttt{\textbackslash mathbb\{Z\}}                                                     & \( \mathbb{Z} \)                     \\
        Rationale Zahlen                                     & \texttt{\textbackslash mathbb\{Q\}}                                                     & \( \mathbb{Q} \)                     \\
        Reelle Zahlen                                        & \texttt{\textbackslash mathbb\{R\}}                                                     & \( \mathbb{R} \)                     \\
        Komplexe Zahlen                                      & \texttt{\textbackslash mathbb\{C\}}                                                     & \( \mathbb{C} \)                     \\

        \textbf{Operator}                                    &                                                                                         &                                      \\
        Laplace-Transformation                               & \texttt{\textbackslash mathcal\{L\}\{f(t)\}}                                            & \( \mathcal{L}\{f(t)\} \)            \\
        Fourier-Transformation                               & \texttt{\textbackslash mathcal\{F\}\{f(t)\}}                                            & \( \mathcal{F}\{f(t)\} \)            \\
        Nabla-Operator                                       & \texttt{\textbackslash nabla}                                                           & \( \nabla \)                         \\
        Gradient                                             & \texttt{\textbackslash nabla f}                                                         & \( \nabla f \)                       \\
        Divergenz                                            & \texttt{\textbackslash nabla \textbackslash cdot \textbackslash vec{F}}                 & \( \nabla \cdot \vec{F} \)           \\
        Rotation                                             & \texttt{\textbackslash nabla \textbackslash times \textbackslash vec{F}}                & \( \nabla \times \vec{F} \)          \\
        Laplace-Operator                                     & \texttt{\textbackslash Delta}                                                           & \( \Delta \)                         \\
    \end{longtable}
    \label{tab:operatoren}
\end{table}