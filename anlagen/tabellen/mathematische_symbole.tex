\begin{table}[H]
    \centering
    \renewcommand{\arraystretch}{1.3}
    \begin{tabular}{lll}
        \toprule
        \textbf{Operator}       & \textbf{LaTeX-Code}                                                 & \textbf{Beispiel}          \\
        \midrule
        Gleichheit              & \texttt{=}                                                          & $ a = b $                  \\
        Ungleichheit            & \texttt{\textbackslash neq}                                         & $ a \neq b $               \\
        Größer als              & \texttt{>}                                                          & $ a > b $                  \\
        Kleiner als             & \texttt{<}                                                          & $ a < b $                  \\
        Größer-gleich           & \texttt{\textbackslash geq}                                         & $ a \geq b $               \\
        Kleiner-gleich          & \texttt{\textbackslash leq}                                         & $ a \leq b $               \\
        Plus                    & \texttt{+}                                                          & $ a + b $                  \\
        Minus                   & \texttt{-}                                                          & $ a - b $                  \\
        Mal (×)                 & \texttt{\textbackslash times}                                       & $ a \times b $             \\
        Mal (Punkt)             & \texttt{\textbackslash cdot}                                        & $ a \cdot b $              \\
        Geteilt (Bruch)         & \texttt{\textbackslash frac\{a\}\{b\}}                              & $ \frac{a}{b} $            \\
        Geteilt (÷)             & \texttt{\textbackslash div}                                         & $ a \div b $               \\
        Prozent                 & \texttt{\%}                                                         & $ 50\% $                   \\
        Summenzeichen           & \texttt{\textbackslash sum\_\{i=1\}\^{}\{n\} ...}                   & $ \sum_{i=1}^{n} ... $     \\
        Produktzeichen          & \texttt{\textbackslash prod\_\{i=1\}\^{}\{n\} ...}                  & $ \prod_{i=1}^{n} ... $    \\
        Integral                & \texttt{\textbackslash int\_\{a\}\^{}\{b\} f(x) \textbackslash, dx} & $ \int_{a}^{b} f(x) \,dx $ \\
        Wurzel                  & \texttt{\textbackslash sqrt\{x\}}                                   & $ \sqrt{x} $               \\
        n-te Wurzel             & \texttt{\textbackslash sqrt[n]\{x\}}                                & $ \sqrt[n]{x} $            \\
        Logarithmus             & \texttt{\textbackslash log{(x)}}                                    & $ \log{(x)} $              \\
        Natürlicher Logarithmus & \texttt{\textbackslash ln{(x)}}                                     & $ \ln{(x)} $               \\
        Exponentialfunktion     & \texttt{e\^{}x}, \texttt{\textbackslash exp{(x)}}                   & $ e^x, \exp{(x)} $         \\
        Sinus                   & \texttt{\textbackslash sin{(x)}}                                    & $ \sin{(x)} $              \\
        Kosinus                 & \texttt{\textbackslash cos{(x)}}                                    & $ \cos{(x)} $              \\
        Tangens                 & \texttt{\textbackslash tan{(x)}}                                    & $ \tan{(x)} $              \\
        Cotangens               & \texttt{\textbackslash cot{(x)}}                                    & $ \cot{(x)} $              \\
        Arkussinus              & \texttt{\textbackslash arcsin{(x)}}                                 & $ \arcsin{(x)} $           \\
        Arkuskosinus            & \texttt{\textbackslash arccos{(x)}}                                 & $ \arccos{(x)} $           \\
        Arkustangens            & \texttt{\textbackslash arctan{(x)}}                                 & $ \arctan{(x)} $           \\
        Modulo                  & \texttt{a \textbackslash bmod b}                                    & $ a \bmod b $              \\
        Konjunktion (UND)       & \texttt{\textbackslash wedge}                                       & $ A \wedge B $             \\
        Disjunktion (ODER)      & \texttt{\textbackslash vee}                                         & $ A \vee B $               \\
        Negation                & \texttt{\textbackslash neg}                                         & $ \neg A $                 \\
        Implikation             & \texttt{\textbackslash Rightarrow}                                  & $ A \Rightarrow B $        \\
        Äquivalenz              & \texttt{\textbackslash Leftrightarrow}                              & $ A \Leftrightarrow B $    \\
        Natürliche Zahlen       & \texttt{\textbackslash mathbb\{N\}}                                 & \( \mathbb{N} \)           \\
        Ganze Zahlen            & \texttt{\textbackslash mathbb\{Z\}}                                 & \( \mathbb{Z} \)           \\
        Rationale Zahlen        & \texttt{\textbackslash mathbb\{Q\}}                                 & \( \mathbb{Q} \)           \\
        Reelle Zahlen           & \texttt{\textbackslash mathbb\{R\}}                                 & \( \mathbb{R} \)           \\
        Komplexe Zahlen         & \texttt{\textbackslash mathbb\{C\}}                                 & \( \mathbb{C} \)           \\
        \bottomrule
    \end{tabular}
    \caption{Wichtige Rechenoperatoren und mathematische Symbole}
    \label{tab:operatoren}
\end{table}