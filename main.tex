\documentclass[titlepage,table]{article}

% Sprach-Pakete:
\usepackage[ngerman]{babel}     % Deutsche Silbentrennung, Begriffe & Typografie
\usepackage[T1]{fontenc}        % Korrekte Darstellung & Trennung von Umlauten

% Layout-Einstellungen
\frenchspacing                  % Einheitlicher Abstand nach Satzzeichen (keine extra Leerzeichen)
\tolerance=9000                 % Erlaubt größere Lücken zwischen Wörtern, um unschöne Umbrüche zu vermeiden
\usepackage{parskip}            % Entfernt Absatzeinrückung und fügt Abstand zwischen Absätzen hinzu
\usepackage{lmodern}            % Schriftart Latin Modern
\usepackage{titlesec}           % Benutzerdefinierte Überschriften
\usepackage{geometry}           % Ermöglicht benutzerdefinierte Seitenränder
\geometry{a4paper, left=25mm, right=25mm, top=3.5cm, bottom=2.5cm} % Ränder setzen
\usepackage{hyperref}           % Aktiviert klickbare Links (URLs, Querverweise, Inhaltsverzeichnis)
\usepackage{fancyhdr}           % Für individuelle Kopf- und Fußzeilen
\usepackage{color}              % Ermöglicht farbige Darstellung von Text und Elementen

% Bilder
\usepackage{graphicx}           % Fügt Bilder (JPG, PNG, PDF) in LaTeX-Dokumente ein    -> \includegraphics[width=...]{bild.png}
\usepackage[inkscapeformat=png]{svg} % Ermöglicht SVG-Grafiken, konvertiert zu PNG
\usepackage{subcaption}         % Ermöglicht mehrere Teilabbildungen mit Untertiteln

% Tabellen
\usepackage{hhline}             % Erlaubt doppelte und selektive Linien in Tabellen

% Quellcode
\usepackage{listings, pmboxdraw} % Quellcode einbinden und optisch ansprechend rahmen

% Mathematik
\usepackage{amsmath}            % Mathematische Symbole, align-Umgebung (&=)
\usepackage{amssymb}            % Extra Mathe-Symbole (z. B. Mengen, Relationen)
\usepackage[makeroom]{cancel}   % Streicht mathematische Ausdrücke durch (mit zusätzlichem Raum)



%%%%%%%%%%%%%%%%%%%%%%%%%%%%%%%%%%%%%%%%%%%%%%%%%%%%%%%%%%%%%%%%%%%%%%%%%%%%%%%%%%%%%%%%%%%%%%%%%%%%%%%%%%%%%%%%%
\usepackage[backend=bibtex, sorting=nyt, urldate=long]{biblatex}
\usepackage[toc,page]{appendix}
\usepackage{acronym}
\usepackage{comment}
\usepackage{siunitx}
\usepackage{pdfpages}        % for including pdfs
\usepackage{multirow}       % for multirow in tables
\usepackage[table]{xcolor}     % for coloring cells in tables
%\geometry{a4paper}

\usepackage{booktabs}

%---------------------------------------------------------------------------------------

% Kopf- und Fußzeile
\fancyhead{} % clear all header fields
\fancyhead[RO,LE]{\leftmark}

\begin{document}
% \includepdf{chapters/Deckblatt.pdf}
% \input{chapters/titlepage}
\tableofcontents
\newpage
\pagestyle{fancy}
\renewcommand{\labelenumi}{\alph{enumi}}
\pagenumbering{arabic} %%ggf \fancyfoot[C]{Seite \thepage~von~\pageref{LastPage}}   \usepackage{lastpage}

\renewcommand{\familydefault}{\sfdefault}


% Text

\section{Erste Schritte}
\subsection{Einstellungen und Struktur}

Zunächst werden die Grundeinstellungen für das Layout des Dokuments festgelegt. Dazu wird eine Latex-Datei \textbf{main.tex} erstellt.

Mit dem Befehl \textbf{\texttt{\textbackslash documentclass[Option1, Option2, ...]\{Dokumentenklasse\}}} wird die Dokumentenklasse festgelegt außerdem können weitere Optionen angegeben werden. 

Mögliche Dokumentenklassen sind:
\input{Anlagen/Tabellen/Dokumentklassen.tex}

Mögliche Optionen sind:
\input{Anlagen/Tabellen/Einstellungen_Dokumentklassen.tex}


\section{Fehlt noch:}

%kommentare
fett, kursiv
reffernzen
%\ und {} in anweisungen
%\text, texttt, textbf, textit, underline , emph 
%\input
%Tabellen, Booktabs, Multicolumn, Multirow, Landscape

\end{document}