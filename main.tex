\documentclass[titlepage,table]{article}

% Sprach-Pakete:
\usepackage[ngerman]{babel}     % Deutsche Silbentrennung, Begriffe & Typografie
\usepackage[T1]{fontenc}        % Korrekte Darstellung & Trennung von Umlauten

% Layout-Einstellungen
\frenchspacing                  % Einheitlicher Abstand nach Satzzeichen (keine extra Leerzeichen)
\tolerance=9000                 % Erlaubt größere Lücken zwischen Wörtern, um unschöne Umbrüche zu vermeiden
\usepackage{parskip}            % Entfernt Absatzeinrückung und fügt Abstand zwischen Absätzen hinzu
\usepackage{lmodern}            % Schriftart Latin Modern
\usepackage{titlesec}           % Benutzerdefinierte Überschriften
\usepackage{lastpage}           % Gesamtseitenzahl abrufen
\usepackage{fancyhdr}           % Für individuelle Kopf- und Fußzeilen -> s.u.
\usepackage{geometry}           % Ermöglicht benutzerdefinierte Seitenränder
    \geometry{a4paper, left=25mm, right=25mm, top=3.5cm, bottom=2.5cm} % Ränder setzen

% praktische Pakete
\usepackage{hyperref}           % Aktiviert klickbare Links (URLs, Querverweise, Inhaltsverzeichnis)
\usepackage{color}              % Ermöglicht farbige Darstellung von Text und Elementen
\usepackage{comment}            % große Textblöcke auskommentieren -> \begin{comment} ... \end{comment}
\usepackage{siunitx}            % Einheiten -> \SI{wert}{\einheit}
\usepackage{float}              % Positionierung von Gleitobjekten (Bilder, Tabellen) -> \begin{figure}[H]

% Bilder / PDFs
\usepackage{graphicx}           % Fügt Bilder (JPG, PNG, PDF) in LaTeX-Dokumente ein    -> \includegraphics[width=...]{bild.png}
\usepackage[inkscapeformat=png]{svg} % Ermöglicht SVG-Grafiken, konvertiert zu PNG
\usepackage{subcaption}         % Ermöglicht mehrere Teilabbildungen mit Untertiteln
\usepackage{pdfpages}           % Einfügen von PDF-Dokumenten -> \includepdf[pages={...}]{dokument.pdf}

% Listen
\usepackage{enumitem}           % Ermöglicht Anpassungen von Listen -> \begin{itemize}[label=...]
\usepackage{multicol}           % Ermöglicht mehrspaltige Listen -> \begin{multicols}{2} ... \end{multicols}

% Tabellen
\usepackage{hhline}             % Erlaubt doppelte und selektive Linien in Tabellen
\usepackage{multirow}           % Erlaubt das Zusammenfassen mehrerer Zeilen in Tabellen
\usepackage{xcolor}             % Ermöglicht das Färben von Tabellenzellen -> \cellcolor{color} und viele weitere Farboptionen
\usepackage{booktabs}           % Schönere Tabellenlinien -> \toprule, \midrule, \bottomrule
\usepackage{longtable}          % Tabellen über mehrere Seiten -> \begin{longtable} ... \end{longtable}

% Quellcode
\usepackage{listings, pmboxdraw} % Quellcode einbinden und optisch ansprechend rahmen
\lstset{
    %language={[LaTeX]TeX},              % Programmiersprache
    frame=single,                       % Rahmen um den Code
    numbers=left,                       % Zeilennummern links
    stepnumber=1,                       % Jede Zeile nummerieren
    numberstyle=\footnotesize,          % Kleine Zeilennummern
    basicstyle=\ttfamily,               % Monospace-Schriftart
    keywordstyle=\color{blue},          % Keywords blau
    commentstyle=\color{gray},          % Kommentare grau
    stringstyle=\color{green},          % Strings grün
    backgroundcolor=\color{white},      % Hintergrundfarbe
    showstringspaces=false,             % Keine Leerzeichen in Strings
    captionpos=b,                        % Position unter dem Code
    breaklines=true,                    % Zeilenumbruch
}


% Mathematik
\usepackage{amsmath}            % Mathematische Symbole, align-Umgebung (&=)
\usepackage{amssymb}            % Extra Mathe-Symbole (z. B. Mengen, Relationen)
\usepackage[makeroom]{cancel}   % Streicht mathematische Ausdrücke durch (mit zusätzlichem Raum)
\usepackage{bm}                 % Fette Symbole in Matheumgebungen -> \bm{...}

% Literaturverzeichnis
\usepackage[backend=bibtex, sorting=nyt, urldate=long]{biblatex} % BibTeX-Backend, sortiert nach Name, Jahr, Titel; langes URL-Abrufdatum

% Abkürzungsliste
\usepackage{acronym} % -> \ac{...} 

% Anhänge
\usepackage[toc,page]{appendix} % Ermöglicht Anhänge mit eigener Überschrift


%-----------------------------------------------------------------------------------------------------------------------------------------------------------------------------------

\begin{document}

% Deckblatt
% \includepdf{kapitel/deckblatt.pdf}                 % komplettes Deckblatt als PDF einfügen
\begin{titlepage}  % Deckblatt als eigene Seite
    \begin{center}

        \vspace*{0.5cm}  % Abstand nach oben

        % Logo
        \includegraphics[width=4cm]{anlagen/bilder/logo.png}
        \vspace{1cm}  % Abstand nach dem Logo

        % Titel und Untertitel
        {\Huge \textbf{\LaTeX{} Anleitung}} \\[0.5cm]
        {\Large \textbf{(kurz für Lamport TeX)}} \\[1.5cm]

        % Horizontale Linie für Trennung 
        \rule{12cm}{0.5pt} \\[1.5cm]  % 12cm lange, 0.5pt dicke Linie

        % Zusätzliche Infos (Autor, Datum, etc.)
        \textbf{\Large Autor:} {\Large Simon Vehreschild} \\[0.5cm]
        % \textbf{\Large Betreuer:} \Large Dr. Beispiel \\[0.5cm]
        \textbf{\Large Datum:} {\Large \today} \\[3cm]

        % --- Universität / Fakultät
        {\large \LaTeX{} ist ein Textsatzsystem, das speziell für wissenschaftliche Arbeiten entwickelt wurde.
        Auf den folgenden Seiten werden einige nützliche Tipps und Tricks vorgestellt, die die Arbeit mit \LaTeX{} vereinfachen sollen}

        \vfill

        {\large Die Inhalte sind ausschließlich für den privaten Gebrauch bestimmt und dürfen nicht weiterverbreitet oder vervielfältigt werden.} \\[1cm]

        \vfill  % Füllt den restlichen Platz, um den Text in der Mitte zu halten
    \end{center}
\end{titlepage}                            % Deckblatt als LaTeX-Datei einfügen (selber erstellt)
% %-------------------------------------------------------------
\begin{titlepage}
    %-------------------------------------------------------------
    \begin{center}
        {\Large\bf Inbetriebnahme Messtechnik}\\[3cm]
        
        {\bf Inbetriebnahme}\\
        für \\
        Gassensorik\\[1.5cm]
        
        an der\\
        Hochschule Niederrhein\\
        Fachbereich Elektrotechnik und Informatik\\
        Studiengang {\em Elektrotechnik}\\[3cm]
        
        vorgelegt von\\
        Tom Arlt und Simon Vehreschild\\
        1335730 und 1543175\\
        Gruppe 12-6\\[3cm]
        Datum: \today\\[3cm]
        
    \end{center}
\end{titlepage}

\pagestyle{empty} % ohne Seitennummerierung


% Kopf- und Fußzeilen
\pagestyle{fancy}                                    % Aktiviert individuelle Kopf- und Fußzeilen
\fancyfoot{}                                         % Löscht alle Standard-Fußzeilen
\fancyfoot[C]{Seite \thepage~von~\pageref{LastPage}} % Fußzeile mit "Seite X von Y"     % Praktisch auch:[RO,LE]
\fancyfoot[R]{Simon Vehreschild}                     % Rechte Seite: Autorname
\fancyfoot[L]{\LaTeX{} Anleitung}                    % Linke Seite: Dokumentname#
\renewcommand{\footrulewidth}{0.4pt}                 % Linie über der Fußzeile
\fancyhead{}                                         % Löscht alle Standard-Kopfzeilen
\renewcommand{\headrulewidth}{0.4pt}                 % Linie unter der Kopfzeile
\fancyhead[L]{\includegraphics[width=1.2cm]{anlagen/bilder/logo.png}} % Logo links in der Kopfzeile
\fancyhead[R]{\nouppercase{\leftmark}}                             % Kopfzeile: Kapitelname

% Inhaltsverzeichnis
\setcounter{tocdepth}{3}                            % Inhaltsverzeichnis bis zur Ebene 3 anzeigen
\setcounter{secnumdepth}{3}                         % Nummerierung bis zur Ebene 3
\tableofcontents
\newpage
\listoftables
\listoffigures
\newpage

% Kapitel in separaten Dateien
\section{Erste Schritte}
\subsection{Einstellungen und Struktur}

Zunächst werden die Grundeinstellungen für das Layout des Dokuments festgelegt. Dazu wird eine \LaTeX{}-Datei \textbf{main.tex} erstellt, in welcher die Dokumentenklasse, die verwendeten Pakete und die Layout-Einstellungen festgelegt werden. Außerdem sind dort das Deckblatt, die Kopf- und Fußzeilen und das Inhaltsverzeichnis zu definieren, sowie die einzelnen Kapitel einzubinden.

\subsubsection{Festlegen der Dokumentenklasse}
Mit dem Befehl \textbf{\texttt{\textbackslash documentclass[Option1, Option2, ...]\{Dokumentenklasse\}}} wird die Dokumentenklasse festgelegt außerdem können weitere Optionen angegeben werden.

Mögliche Dokumentenklassen sind:
\begin{table}[H]
    \centering
    \begin{tabular}{ll}
        \toprule
        \textbf{Klasse}  & \textbf{Beschreibung}                                                         \\
        \midrule
        \texttt{article} & Kurze Texte, wissenschaftliche Artikel, Berichte,                             \\
                         & Unterstützt \textbackslash section\{\}  aber keine \textbackslash chapter\{\} \\
        \texttt{report}  & Längere Dokumente mit Kapiteln (z. B. Abschlussarbeiten)                      \\
                         & Unterstützt \textbackslash chapter\{\}                                        \\
        \texttt{book}    & Bücher mit Kapiteln, Abschnitten und Teilen                                   \\
                         & Unterstützt \textbackslash part\{\}                                           \\
        \texttt{letter}  & Briefe                                                                        \\

        \texttt{beamer}  & Präsentationen (wie PowerPoint)                                               \\
        \bottomrule
    \end{tabular}
    \caption{Standard-Dokumentenklassen für \texttt{\textbackslash documentclass}}
    \label{tab:dokumentklassen}
\end{table}

Mögliche Optionen sind:
\begin{table}[h]
    \centering
    \begin{tabular}{ll}
        \toprule
        \textbf{Option}                        & \textbf{Beschreibung}                                              \\
        \midrule
        \texttt{Xpt}                           & Schriftgröße (Standard: 10pt)                                      \\
        \texttt{a4paper, b5paper, letterpaper} & Papierformat (A4, B5, Letter, usw.)                                \\
        \texttt{onecolumn, twocolumn}          & Einspaltig, zweispaltig (Standard: onecolumn)                      \\
        \texttt{titlepage, notitlepage}        & Eigene Titelseite oder nicht    (Standard: notitlepage)            \\
        \texttt{twoside, oneside}              & Doppelseitiges oder einseitiges Layout (Bücher vs. Artikel)        \\
        \texttt{landscape}                     & Querformat                                                         \\
        \texttt{draft, final}                  & Entwurfsmodus (keine Bilder) oder finale Version (Standard: final) \\
        \texttt{fleqn}                         & Mathe-Formeln linksbündig (Standard: zentriert)                    \\
        \texttt{leqno, reqno}                  & Gleichungsnummern links oder rechts (`amsmath`) (Standard: reqno)  \\
        \bottomrule
    \end{tabular}
    \caption{Wichtige Optionen für \texttt{\textbackslash documentclass}}
    \label{tab:documentclass-options}
\end{table}


\subsubsection{Inkludieren von Paketen}
Der Befehl \textbf{\texttt{\textbackslash usepackage[Optionen]\{Paketname\}}} ermöglicht das Einbinden von Paketen, die zusätzliche Funktionen und Einstellungen bereitstellen.
Eine Übersicht an Paketen und deren Funktionen ist in der bereitgestellten main.tex Datei zu finden.

\subsection{Öffnen und Schließen eines Dokuments}
Das eigentliche Dokument wird mit dem Befehl \textbf{\texttt{\textbackslash begin\{document\}}} eröffnet und mit dem Befehl \textbf{\texttt{\textbackslash end\{document\}}} beendet.
Alle Anweisungen davor dienen lediglich der Vorbereitung des Dokuments. Alle Anweisungen danach werden ignoriert. Alle Inhalte der folgenden Unterpunkte sind innerhalb dieser beiden Befehle zu platzieren.

\subsection{Einbinden / Erstellen des Deckblatts}
In der Regel wird ein Deckblatt benötigt, welches Informationen wie den Titel, den Autor, das Datum und ein Logo enthält. Dieses kann entweder als PDF-Datei eingebunden, falls ein bereits erstelltes Deckblatt verwendet werden soll oder als neue \LaTeX{}-Datei erstellt werden.
Das einbinden einer PDF-Datei erfolgt mit dem Befehl \textbf{\texttt{\textbackslash includepdf\{Pfad/zur/Datei.pdf\}}}. Für diesen Befehl wird das Paket \textbf{pdfpages} benötigt (siehe auch \ref{sec:einbinden_von_pdf_dateien}).
Alternativ kann das Deckblatt auch als \LaTeX{}-Datei eingebunden werden, indem der Inhalt des Deckblatts in eine neue Datei geschrieben und mit dem Befehl \textbf{\texttt{\textbackslash input\{Pfad/zur/Datei.tex\}}} eingebunden wird. Hilfreich ist die Funktion \textbf{\texttt{\textbackslash begin\{titlepage\}}} und \textbf{\texttt{\textbackslash end\{titlepage\}}} um zwischen diesen Befehlen das Deckblatt zu erstellen. In dem Ordner \textbf{kapitel} sind zwei Beispieldateien zu finden.

\subsection{Anpassen der Kopf- und Fußzeilen}
Die Kopf- und Fußzeilen können mit dem Paket \textbf{fancyhdr} besonders elegant angepasst werden. Zunächst wird mit dem Befehl \textbf{\texttt{\textbackslash pagestyle\{fancy\}}} die Verwendung von individuellen Kopf- und Fußzeilen aktiviert. Anschließend werden mit den Befehlen \textbf{\texttt{\textbackslash fancyfoot\{\}}} und \textbf{\texttt{\textbackslash fancyhead\{\}}} die Standardeinstellungen für Fuß- und Kopfzeile gelöscht. Die Befehle \textbf{\texttt{\textbackslash fancyfoot[Position]\{Text\}}} und \textbf{\texttt{\textbackslash fancyhead[Position]\{Text\}}} ermöglichen das Hinzufügen von Text oder Bildern in die Fuß- oder Kopfzeile.

Als Position sind die folgenden Angaben möglich:
\begin{itemize}
    \item \textbf{C} - zentriert
    \item \textbf{L} - linksbündig
    \item \textbf{R} - rechtsbündig
    \item \textbf{C} / \textbf{L} / \textbf{R} in Kombination mit \textbf{E} oder \textbf{O} für gerade oder ungerade Seitenzahlen möglich
\end{itemize}

Ist eine Seitennummerierung in der Form Seite X von Y gewünscht, kann dies mit dem Befehl \textbf{\texttt{\textbackslash fancyfoot[C]\{Seite \textbackslash thepage$\thicksim$von$\thicksim$\textbackslash pageref\{LastPage\}\}}} ermöglicht werden. Hierbei wird das Paket \textbf{lastpage} benötigt.

Soll die Kopfzeile den aktuellen \textbackslash section\{\} Namen anzeigen, kann dies mit dem Befehl \\ \textbf{\texttt{\textbackslash fancyhead[R]\{\{\textbackslash leftmark\}\}}} erreicht werden.

Eine Trennlinie zwischen Fußzeile und Dokument, sowie zwischen Kopfzeile und Dokument kann mit den Befehlen \textbf{\texttt{\textbackslash renewcommand\{\textbackslash footrulewidth\}\{...pt\}}} und \textbf{\texttt{\textbackslash renewcommand\{\textbackslash headrulewidth\}\{...pt\}}} eingefügt oder mit 0pt gelöscht werden.

Manchmal ist es gewünscht auf einer Seite keine Kopf- und/oder Fußzeile zu haben. Dies ist möglich mit:
\begin{itemize}
    \item \textbf{\texttt{\textbackslash thispagestyle\{empty\}}} - Keine Kopf- oder Fußzeile
    \item \textbf{\texttt{\textbackslash thispagestyle\{plain\}}} - Normale Fußzeile ohne Kopfzeile
\end{itemize}

\subsection{Erstellen des Inhaltsverzeichnisses (und weiterer Verzeichnisse)}
\label{sec:inhaltsverzeichnis}
Das Inhaltsverzeichnis wird mit dem Befehl \textbf{\texttt{\textbackslash tableofcontents}} erstellt. Die Tiefe des Inhaltsverzeichnis kann mit dem Befehl \textbf{\texttt{\textbackslash setcounter\{tocdepth\}\{X\}}} festgelegt werden. Dabei steht X für die \underbar{maximale Tiefe} des Inhaltsverzeichnisses und kann folgende Werte annehmen:

\begin{itemize}
    \item \textbf{0} - Nur \texttt{\textbackslash part\{\}} (nur in \texttt{book} oder \texttt{report})
    \item \textbf{1} - \texttt{\textbackslash part\{\}} und \texttt{\textbackslash chapter\{\}} (nur in \texttt{book} oder \texttt{report})
    \item \textbf{2} - \texttt{\textbackslash chapter\{\}} und \texttt{\textbackslash section\{\}}
    \item \textbf{3} - \texttt{\textbackslash section\{\}} und \texttt{\textbackslash subsection\{\}}
    \item \textbf{4} - \texttt{\textbackslash subsection\{\}} und \texttt{\textbackslash subsubsection\{\}}
    \item \textbf{5} - \texttt{\textbackslash subsubsection\{\}} und \texttt{\textbackslash paragraph\{\}}
    \item \textbf{6} - \texttt{\textbackslash paragraph\{\}} und \texttt{\textbackslash subparagraph\{\}}
\end{itemize}

Für die Anwendung dieses Befehls ist darauf zu achten, dass der Befehl \textbf{\texttt{\textbackslash setcounter\{tocdepth\}\{X\}}} vor dem Befehl \textbf{\texttt{\textbackslash tableofcontents}} platziert werden muss.

Sollen weitere Inhaltsverzeichnisse für Abbildungen, Tabellen, etc. erstellt werden, so kann dies mit den Befehlen \textbf{\texttt{\textbackslash listoffigures}}, \textbf{\texttt{\textbackslash listoftables}}, etc. erreicht werden.


Darüber hinaus sind weitere Einstellungen mit den Paketen \textbf{tocloft}, \textbf{titletoc} und \textbf{minitoc} möglich.

\subsection{Einbinden von Kapiteln}
Besonders bei langen und umfangreichen Dokumenten ist es sinnvoll, die einzelnen Kapitel in separaten Dateien zu erstellen und diese dann in das Hauptdokument einzubinden. Dies geschieht mit dem Befehl \textbf{\texttt{\textbackslash input\{Pfad/zur/Datei.tex\}}}. Dabei ist darauf zu achten, dass der Pfad relativ zum Hauptdokument angegeben wird.
Ich persönlich strukturiere meine \LaTeX{}-Dateien in einem eigenen Ordner (kapitel) und binde sie in einer dafür vorgesehenen Datei (uebersicht.tex) ein, welche in der main.tex Datei eingefügt wird. Somit muss die main.tex Datei nach einmaliger Einrichtung nicht mehr verändert werden.

Mehr zu dem Inkludieren von \LaTeX{}-Dateien ist in Kapitel \ref{sec:einbinden_von_latex_dateien} zu finden.


\newpage
\section{Wichtige Befehle}
\subsection{Kommentare}
Kommentare werden in \LaTeX{} mit einem Prozentzeichen eingeleitet. Alles, was nach dem Prozentzeichen in der selben Zeile steht, wird von \LaTeX{} ignoriert. Mehrzeilige Kommentare sind mit der \texttt{\textbf{comment}}-Umgebung möglich. Dafür wird das Paket \textbf{comment} benötigt. Funktional identisch ist der Befehl \textbf{\texttt{\textbackslash iffalse ... \textbackslash fi}}.

\begin{lstlisting}[language={[LaTeX]TeX}]
% Dies ist ein einzeiliger Kommentar
Text ohne Kommentar.
    
\begin{comment}
Dies ist ein
langer Kommentarblock
\end{comment}

\iffalse
Dies ist ein
langer Kommentarblock
\fi
\end{lstlisting}

\subsection{Textformatierung}
\subsubsection{Schriftgröße}
Die Schriftgröße kann mit den folgenden Befehlen geändert werden:

\begin{table}[h]
    \centering
    \begin{tabular}{lll}
        \toprule
        \textbf{Befehl}             & \textbf{Größe}           & \textbf{Beispieltext} \\
        \midrule
        \textbackslash tiny         & Sehr klein               & {\tiny Text}          \\
        \textbackslash scriptsize   & Kleiner als normal       & {\scriptsize Text}    \\
        \textbackslash footnotesize & Kleine Schrift           & {\footnotesize Text}  \\
        \textbackslash small        & Etwas kleiner als normal & {\small Text}         \\
        \textbackslash normalsize   & Standardgröße            & {\normalsize Text}    \\
        \textbackslash large        & Etwas größer als normal  & {\large Text}         \\
        \textbackslash Large        & Noch größer              & {\Large Text}         \\
        \textbackslash LARGE        & Sehr groß                & {\LARGE Text}         \\
        \textbackslash huge         & Sehr große Schrift       & {\huge Text}          \\
        \textbackslash Huge         & Extrem große Schrift     & {\Huge Text}          \\
        \bottomrule
    \end{tabular}
    \caption{Schriftgrößen}
    \label{tab:schriftgroessen}
\end{table}

Durch Verwendung von geschweiften Klammern kann der Befehl auf einen eingeschränkten Textbereich angewendet werden:
Beispiel: \textbf{\texttt{\{\textbackslash large großer Text\}}}

\newpage

\subsubsection{Schriftstil}
Der Schriftstil kann mit den folgenden Befehlen geändert werden:

\begin{table}[h]
    \centering
    \begin{tabular}{lll}
        \toprule
        \textbf{Befehl}               & \textbf{Wirkung}                 & \textbf{Beispieltext} \\
        \midrule
        \textbackslash textbf\{\}     & Fett                             & \textbf{Text}         \\
        \textbackslash textit\{\}     & Kursiv                           & \textit{Text}         \\
        \textbackslash texttt\{\}     & Monospace (Typewriter)           & \texttt{Text}         \\
        \textbackslash underline\{\}  & Unterstrichen                    & \underline{Text}      \\
        \textbackslash emph\{\}       & Hervorgehoben (Kursiv o. Fett)   & \emph{Text}           \\
        \textbackslash textsc\{\}     & Kapitälchen (Small Caps)         & \textsc{Text}         \\
        \textbackslash textnormal\{\} & Normale Schrift                  & \textnormal{Text}     \\
        \textbackslash textsf\{\}     & Serifenlose Schrift (Sans-Serif) & \textsf{Text}         \\
        \textbackslash textsl\{\}     & Schräggestellt (Slanted)         & \textsl{Text}         \\
        \textbackslash textmd\{\}     & Mittlere Schriftstärke (Medium)  & \textmd{Text}         \\
        \textbackslash textup\{\}     & Aufrechte Schrift (Upright)      & \textup{Text}         \\
        \bottomrule
    \end{tabular}
    \caption{Schriftstil-Optionen}
    \label{tab:schriftstile}
\end{table}

Die Schriftstile können mit den Schriftgrößen kombiniert und mit anderen Schriftstilen überlagert werden.
Es ist zu beachten, dass \textbf{\texttt{\textbackslash lowercase\{\}}} und \textbf{\texttt{\textbackslash uppercase\{\}}} zu Problemen mit Umlauten führen können, daher ist die Verwendung von \textbf{\texttt{\textbackslash MakeUppercase\{\}}} und \textbf{\texttt{\textbackslash MakeLowercase\{\}}} zu empfehlen.

\subsubsection{Schriftfarbe}
\label{sec:schriftfarbe}
Die Schriftfarbe kann auf folgende Arten geändert werden:

\begin{itemize}
    \item \textbf{\texttt{\textbackslash textcolor\{red!80\}\{Roter Text\}}}
          \hspace{2.26cm} --- \hspace{1cm} \textcolor{red!80}{Roter Text}

    \item \textbf{\texttt{\textbackslash textcolor[rgb]\{0,0.5,1\}\{RGB-Farbe\}}}
          \hspace{1.35cm} --- \hspace{1cm} \textcolor[rgb]{0,0.5,1}{RGB-Farbe}

    \item \textbf{\texttt{\textbackslash textcolor[HTML]\{00FF00\}\{Hex-Farbcode\}}}
          \hspace{0.8cm} --- \hspace{1cm} \textcolor[HTML]{00FF00}{Hex-Farbcode}
\end{itemize}

Für diese Varianten wird das Paket \textbf{xcolor} benötigt. Darüber hinaus ist zu beachten, dass \textbf{\texttt{\textbackslash textcolor[rgb]\{...\}}} einen RGB-Wert im Intervall $[0,1]$ (statt $[0,255]$) erwartet.

Nach dem \textbf{\texttt{!}} kann ein Wert zwischen 0 und 100 angegeben werden, um die Deckkraft (Farbintensität) zu steuern.


\subsubsection{Rahmen und Boxen}
\label{sec:rahmen_und_boxen}
Textpassagen Bilder, Tabellen und andere ähnliche Objekte können mit folgenden Befehlen in Rahmen bzw. Boxen gesetzt werden:
\begin{table}[H]
    \centering
    \begin{tabular}{lll}
        \toprule
        \textbf{Befehl}                                            & \textbf{Wirkung}                          & \textbf{Darstellung}          \\
        \midrule
        \texttt{\textbackslash fbox\{Text\}}                       & Rahmen um den Text                        & \fbox{Text}                   \\
        \texttt{\textbackslash framebox(45,10)\{Text\}}            & Box mit fester Breite und Höhe            & \framebox(45,10){Text}        \\
        \texttt{\textbackslash colorbox\{cyan\}\{Text\}}           & Farbige Box mit Hintergrundfarbe (xcolor) & \colorbox{cyan}{Text}         \\
        \texttt{\textbackslash fcolorbox\{red\}\{yellow\}\{Text\}} & Farbige Box mit Rahmen (xcolor)           & \fcolorbox{red}{yellow}{Text} \\
        \texttt{\textbackslash shadowbox\{Text\}}                  & Box mit Schatten (fancybox)               & \shadowbox{Text}              \\
        \texttt{\textbackslash doublebox\{Text\}}                  & Box mit doppeltem Rahmen (fancybox)       & \doublebox{Text}              \\
        \texttt{\textbackslash ovalbox\{Text\}}                    & Ovale Box (fancybox)                      & \ovalbox{Text}                \\
        \bottomrule
    \end{tabular}
    \caption{Rahmen und Boxen}
    \label{tab:boxen}
\end{table}


\subsubsection{Spezialfälle}

\begin{table}[H]
    \centering
    \begin{tabular}{lll}
        \toprule
        \textbf{Funktion}       & \textbf{Befehl}                     & \textbf{Darstellung}             \\
        \midrule
        Text hochstellen        & \texttt{x\textasciicircum\{4\}}     & $x^{4}$                          \\
        Text tiefstellen        & \texttt{H\_\{2\}O}                  & $H_{2}O$                         \\
        Geschütztes Leerzeichen & \texttt{Hallo\textasciitilde Welt!} & Hallo~Welt! (Verhindert Umbruch) \\
        \bottomrule
    \end{tabular}
    \caption{Spezial Befehle zur Textformatierung}
    \label{tab:textformatierung}
\end{table}

Einige Zeichen, wie \textbf{\%}, \textbf{\#}, \textbf{\&}, \textbf{\_} und \textbf{\{} müssen in \LaTeX{} mit einem Backslash maskiert werden, um sie als Textzeichen zu verwenden (\textbf{\textbackslash\%}, \textbf{\textbackslash\#}, \textbf{\textbackslash\&}, \textbf{\textbackslash\_}, \textbf{\textbackslash\{}).

Für einige Zeichen gibt es auch spezielle Befehle, die eine Darstellung unabhängig von der Funktion des Zeichens ermöglichen. Beispiele sind \textbf{\textbackslash textbackslash} für den Backslash \textbf{\textbackslash}, \textbf{\textbackslash textasciitilde} für die Tilde \textbf{\textasciitilde} und \textbf{\textbackslash textasciicircum} für das Zirkumflex \textbf{\textasciicircum}.


\subsection{Abstandsformatierung}
\subsubsection{Vertikaler Abstand}
Der vertikale Abstand kann mit den folgenden Befehlen geändert werden:

\begin{table}[h]
    \centering
    \begin{tabular}{lp{10cm}}
        \toprule
        \textbf{Befehl}                           & \textbf{Wirkung}                                                      \\
        \midrule
        \texttt{\textbackslash vspace\{1cm\}}     & Fügt \textbf{1 cm} vertikalen Abstand ein (am Seitenanfang ignoriert) \\
        \texttt{\textbackslash vspace*\{1cm\}}    & Fügt \textbf{1 cm} vertikalen Abstand ein, auch am Seitenanfang       \\
        \texttt{\textbackslash addvspace\{1cm\}}  & Fügt \textbf{1 cm Abstand nur hinzu, wenn vorher keiner war}          \\
        \texttt{\textbackslash smallskip}         & Fügt \textbf{kleinen flexiblen Abstand} ein                           \\
        \texttt{\textbackslash medskip}           & Fügt \textbf{mittleren flexiblen Abstand} ein                         \\
        \texttt{\textbackslash bigskip}           & Fügt \textbf{großen flexiblen Abstand} ein                            \\
        \texttt{\textbackslash vfill}             & Füllt den \textbf{gesamten verbleibenden Platz} mit Abstand           \\
        \midrule
        \texttt{\textbackslash newline}           & Erzwingt einen \textbf{Zeilenumbruch} und beginnt eine neue Zeile     \\
        \texttt{\textbackslash newpage}           & Erzwingt einen \textbf{Seitenumbruch}                                 \\
        \texttt{\textbackslash clearpage}         & Erzwingt einen \textbf{Seitenumbruch} nach allen restlichen Gleitobj. \\
        \texttt{\textbackslash linebreak[prio]}   & Erzwingt einen \textbf{Zeilenumbruch} mit Priorität (1-4)             \\
        \texttt{\textbackslash nolinebreak[prio]} & Verhindert einen \textbf{Zeilenumbruch} mit Priorität (1-4)           \\
        \texttt{\textbackslash pagebreak[prio]}   & Erzwingt einen \textbf{Seitenumbruch} mit Priorität (1-4)             \\
        \texttt{\textbackslash nopagebreak[prio]} & Verhindert einen \textbf{Seitenumbruch} mit Priorität (1-4)           \\
        \bottomrule
    \end{tabular}
    \caption{vertikale Abstandsoptionen}
    \label{tab:vertikale_abstaende}
\end{table}

Die Priorität \textbf{[prio]} gibt an, wie wichtig der Umbruch ist (4 ist der höchste Wert).

Eine weitere Möglichkeit, vertikalen Abstand zu erzeugen, ist die Verwendung von Leerzeilen. Diese erzeugen (genau) einen neuen Absatz und somit vertikalen Abstand.

Darüber hinaus kann \textbf{\texttt{\textbackslash stretch\{X\}}} als Argument an \textbf{\texttt{\textbackslash vspace}} übergeben werden, um flexiblen vertikalen Abstand zu erzeugen. Der Wert von \texttt{X} gibt dabei an, wie viel Teile (Verhältnis) des verfügbaren Platzes der Abstand einnehmen soll.

\subsubsection{Horizontaler Abstand}
Der horizontale Abstand kann mit den folgenden Befehlen geändert werden:

\begin{table}[h]
    \centering
    \begin{tabular}{lp{10cm}}
        \toprule
        \textbf{Befehl}                        & \textbf{Wirkung}                                                  \\
        \midrule
        \texttt{\textbackslash hspace\{1cm\}}  & Fügt \textbf{1 cm} horizontalen Abstand ein                       \\
        \texttt{\textbackslash hspace*\{1cm\}} & Fügt \textbf{1 cm} horizontalen Abstand ein, auch am Zeilenanfang \\
        \texttt{\textbackslash kern Xpt}       & Fügt \textbf{X pt Abstand} zwischen Zeichen ein                   \\
        \texttt{\textbackslash quad}           & Fügt \textbf{einfachen Tabulator-Abstand} (1 em) ein              \\
        \texttt{\textbackslash qquad}          & Fügt \textbf{doppelten Tabulator-Abstand} (2 em) ein              \\
        \texttt{\textbackslash enspace}        & Fügt \textbf{halben Leerzeichen-Abstand} (0.5 em) ein             \\
        \texttt{\textbackslash hfill}          & Füllt den \textbf{gesamten verfügbaren horizontalen Platz}        \\
        \midrule
        \texttt{\textbackslash !}              & Entfernt ein \textbf{kleines Leerzeichen} (negativer Abstand)     \\
        \texttt{\textbackslash ignorespaces}   & Entfernt \textbf{alle nachfolgenden Leerzeichen}                  \\
        \bottomrule
    \end{tabular}
    \caption{horizontale Abstandsoptionen}
    \label{tab:horizontale_abstaende}
\end{table}

Eine weitere Möglichkeit, horizontalen Abstand zu erzeugen, ist die Verwendung von Leerzeichen. Diese erzeugen (genau) ein Leerzeichen und somit horizontalen Abstand.

Darüber hinaus kann \textbf{\texttt{\textbackslash stretch\{X\}}} als Argument an \texttt{\textbackslash hspace} übergeben werden, um flexiblen horizontalen Abstand zu erzeugen. Der Wert von \texttt{X} gibt dabei an, wie viel Teile (Verhältnis) des verfügbaren Platzes der Abstand einnehmen soll.

Ein em ist die Breite des Großbuchstaben M in der aktuellen Schriftart und ein pt entspricht etwa 0.35 mm.

\subsubsection{Zentrierung}
In \LaTeX{} können Texte und Objekte zentriert werden. Dafür gibt es die folgenden Befehle:
\begin{table}[H]
    \centering
    \renewcommand{\arraystretch}{1.2}
    \begin{tabular}{lp{6cm}c}
        \toprule
        \textbf{Befehl} & \textbf{Funktion}                  \\
        \midrule
        \texttt{\textbackslash begin\{center\} ... \textbackslash end\{center\}}
                        & Zentriert alles im Block           \\

        \texttt{\textbackslash centering}
                        & Zentriert innerhalb einer Umgebung \\
        \bottomrule
    \end{tabular}
    \caption{Möglichkeiten zur Zentrierung}
    \label{tab:center_vs_centering}
\end{table}

\subsection{Einbinden von \LaTeX{}-Dateien}
\label{sec:einbinden_von_latex_dateien}
Häufig ist es notwendig und sinnvoll, den \LaTeX{}-Code in mehrere Dateien aufzuteilen. Dies kann zur Strukturierung des Dokuments, zur Wiederverwendung von Codeblöcken oder zur Auslagerung von z.B. Tabellen sinnvoll sein.

Für das Einbinden gibt es folgende Befehle:
\begin{table}[H]
    \centering
    \renewcommand{\arraystretch}{1.1}
    \begin{tabular}{lccc}
        \toprule
        \textbf{Befehl}                                       & \textbf{Seitenumbruch} \\
        \midrule
        \texttt{\textbackslash input\{Pfad/zur/Datei.tex\}}   & Nein                   \\
        \texttt{\textbackslash include\{Pfad/zur/Datei.tex\}} & Ja                     \\

        \bottomrule
    \end{tabular}
    \caption{Einbindung von \LaTeX{}-Dateien}
    \label{tab:latex_einbindung}
\end{table}

Es ist zu beachten, dass der Befehl \texttt{\textbackslash include} in bereits inkludierten Dateien fehleranfällig sein kann und daher vermieden werden sollte.

\subsection{Einbinden von PDF-Dateien}
\label{sec:einbinden_von_pdf_dateien}
PDF-Dateien können in \LaTeX{}-Dokumente eingebunden werden. Dafür wird das Paket \textbf{pdfpages} benötigt. Die Einbindung erfolgt mit dem Befehl \textbf{\texttt{\textbackslash includepdf\{Pfad/zur/Datei.pdf\}}}. Es können auch nur bestimmte Seiten eingebunden werden mit der Option \textbf{\texttt{\textbackslash includepdf[pages=\{...\}]\{Pfad/zur/Datei.pdf\}}}.


\subsection{Dokumentenhierarchie}
Jedes \LaTeX{}-Dokument hat eine Hierarchie, die durch die folgende Befehle definiert wird:

\begin{table}[H]
    \centering
    \begin{tabular}{lll}
        \toprule
        \textbf{Befehl}                           & \textbf{Ebenen-Tiefe}       & \textbf{Verfügbar in} \\
        \midrule
        \texttt{\textbackslash part\{\}}          & Höchste Ebene               & \texttt{report, book} \\
        \texttt{\textbackslash chapter\{\}}       & Kapitel                     & \texttt{report, book} \\
        \texttt{\textbackslash section\{\}}       & Hauptabschnitt              & Alle Dokumentklassen  \\
        \texttt{\textbackslash subsection\{\}}    & Unterabschnitt              & Alle Dokumentklassen  \\
        \texttt{\textbackslash subsubsection\{\}} & geringer als Unterabschnitt & Alle Dokumentklassen  \\
        \texttt{\textbackslash paragraph\{\}}     & Absatz mit Überschrift      & Alle Dokumentklassen  \\
        \texttt{\textbackslash subparagraph\{\}}  & Noch kleiner (meist inline) & Alle Dokumentklassen  \\
        \bottomrule
    \end{tabular}
    \caption{Dokumentenhierarchie}
    \label{tab:dokumenten_hierarchie}
\end{table}

Die Nummerierung der Ebenen erfolgt automatisch bis zur Ebene \textbf{subsubsection} (X=3).
Mit dem Befehl \textbf{\texttt{\textbackslash setcounter\{secnumdepth\}\{X\}}} kann die Nummerierungstiefe angepasst werden. Das X entspricht der Nummerierungstiefe (siehe \autoref{tab:secnumdepth}).


Die Hierarchie-Ebenen sind auch im Inhaltsverzeichnis sichtbar und können dort mit \textbf{\texttt{\textbackslash setcounter\{tocdepth\}\{X\}}} angepasst werden (siehe auch \ref{sec:inhaltsverzeichnis}).

In manchen Fällen ist es erwünscht, eine Ebene nicht zu nummerieren und auch nicht im Inhaltsverzeichnis aufzuführen. Dies kann mit dem Befehl \textbf{\texttt{\textbackslash section*\{Überschrift\}}} erreicht werden.

Soll die Ebene zwar im Inhaltsverzeichnis erscheinen, aber nicht nummeriert werden, kann der Befehl \textbf{\texttt{\textbackslash addcontentsline\{toc\}\{section\}\{Überschrift\}}} nach der jeweiligen section mit * verwendet werden.

Falls im Inhaltsverzeichnis eine andere Überschrift als im Text erscheinen soll, kann der Befehl \textbf{\texttt{\textbackslash section[Kurztitel]\{ausführlicher Titel\}}} verwendet werden. Der Text in den eckigen Klammern wird im Inhaltsverzeichnis angezeigt, der Text in den geschweiften Klammern im Text als Überschrift.

In Berichten werden in der Regel nur die Ebenen \textbf{section} bis \textbf{subsubsection} verwendet. Für diese Hierarchieebenen folgen nun Beispiele:
\section*{1 \hspace{0.15cm} section}

\subsection*{1.1 \hspace{0.15cm} subsection}

\subsubsection*{1.1.1 \hspace{0.15cm} subsubsection}

\newpage

\subsection{Querverweise (Label und Referenzen)}
\label{sec:querverweise}
\LaTeX{} bietet ein automatisches Querverweis-System, welches es ermöglicht, auf Kapitel, Abschnitte, Bilder, Tabellen, Formeln und andere Elemente im Dokument zu verweisen.
Dafür werden Labels an den gewünschten Stellen im Dokument gesetzt und an anderer Stelle mit Referenzen darauf verwiesen.

\textbf{Labels} können mit dem Befehl \textbf{\texttt{\textbackslash label\{name\}}} gesetzt werden. Der Name kann frei gewählt werden, sollte aber eindeutig sein. Hilfreich ist die konventionelle Benennung, \textbf{fig:} für Bilder, \textbf{tab:} für Tabellen, \textbf{sec:} für Abschnitte und \textbf{eq:} für Formeln.

\textbf{Referenzen} können mit dem Befehl \textbf{\texttt{\textbackslash ref\{name\}}} gesetzt werden. Der Name entspricht dem des Label, auf das verwiesen werden soll.

Mit dem Befehl \textbf{\texttt{\textbackslash pageref\{name\}}} kann auf die Seitenzahl des Labels verwiesen werden.

Der Befehl \textbf{\texttt{\textbackslash autoref\{name\}}} kann den \textbf{Typ des Labels} (Abschnitt, Tabelle, Abbildung, Gleichung) ausgeben. Allerdings ist das Paket \textbf{hyperref} erforderlich.

Soll der \textbf{Name des referenzierten Objektes} ausgegeben werden, kann dies mit \textbf{\texttt{\textbackslash nameref\{name\}}} erzielt werden.

Mit dem Befehl \textbf{\texttt{\textbackslash hyperref[name]\{eigener Kommentar\}}} kann ein Link auf das Label gesetzt werden. Dafür wird das Paket \textbf{hyperref} benötigt.

Mit \textbf{\texttt{\textbackslash href\{URL\}\{Text\}}} kann ein Link zu einer URL gesetzt werden.

In \autoref{tab:querverweise} wird dies anhand eines Beispiels demonstriert:
\begin{table}[H]
    \centering
    \begin{tabular}{ll}
        \toprule
        \textbf{Befehl}                                                              & \textbf{Ergebnis}                                   \\
        \midrule
        \texttt{\textbackslash ref\{sec:querverweise\}}                              & \ref{sec:querverweise}                              \\
        \texttt{\textbackslash eqref\{sec:querverweise\}}                            & \eqref{sec:querverweise}                            \\
        \texttt{\textbackslash pageref\{sec:querverweise\}}                          & \pageref{sec:querverweise}                          \\
        \texttt{\textbackslash autoref\{sec:querverweise\}}                          & \autoref{sec:querverweise}                          \\
        \texttt{\textbackslash nameref\{sec:querverweise\}}                          & \nameref{sec:querverweise}                          \\
        \texttt{\textbackslash hyperref[sec:querverweise]\{siehe bei Querverweise\}} & \hyperref[sec:querverweise]{siehe bei Querverweise} \\
        \bottomrule
    \end{tabular}
    \caption{Übersicht über verschiedene Referenzbefehle}
    \label{tab:querverweise}
\end{table}

Bei den Befehlen (Ausnahme \textbackslash eqref und \textbackslash href) kann ein \textbf{*} verwendet werden um den Hyperlink zu entfernen. Beispiel: \texttt{\textbackslash ref*\{sec:querverweise\}}


\subsection{Zitieren}
\label{sec:zitieren}
In \LaTeX{} können Zitate und Literaturverweise mit dem Befehl \textbf{\texttt{\textbackslash cite\{Schlüssel\}}} gesetzt werden. Der Schlüssel entspricht dem Namen des Eintrags in der Literaturdatenbank.
Für das Zitieren wird das Paket \textbf{cite} benötigt.

Es gibt verschiedene Zitierstile, die festgelegt werden können (siehe auch: \nameref{sec:literaturverzeichnis_erklärung}). Im folgenden wird der Stil \textbf{plain} verwendet.

\subsubsection{Einfache Zitate}

\begin{minipage}[c]{0.48\textwidth}
    \begin{lstlisting}[language={[LaTeX]TeX}]
Mustermann beschreibt dies in seiner Arbeit \cite{buch1}.
    \end{lstlisting}
\end{minipage}
\hfill
\begin{minipage}[c]{0.48\textwidth}
    Mustermann beschreibt dies in seiner Arbeit \cite{buch1}.
\end{minipage}


\subsubsection{Mehrere Quellen zitieren}

\begin{minipage}[c]{0.48\textwidth}
    \begin{lstlisting}[language={[LaTeX]TeX}]
Laut Mustermann und Doe \cite{buch1, artikel1} ist ...
    \end{lstlisting}
\end{minipage}
\hfill
\begin{minipage}[c]{0.48\textwidth}
    Laut Mustermann und Doe \cite{buch1, artikel1} ist ...
\end{minipage}


\subsubsection{Zitate mit Seitenangabe}

\begin{minipage}[c]{0.48\textwidth}
    \begin{lstlisting}[language={[LaTeX]TeX}]
Laut Mustermann \cite[S. 5]{buch1} ist ...
    \end{lstlisting}
\end{minipage}
\hfill
\begin{minipage}[c]{0.48\textwidth}
    Laut Mustermann \cite[S. 5]{buch1} ist ...
\end{minipage}


\subsubsection{Quellen nur im Literaturverzeichnis anzeigen}
Mit dem Befehl \textbf{\texttt{\textbackslash nocite\{Schlüssel\}}} können Einträge aus der Literaturdatenbank (literatur.bib) im Literaturverzeichnis angezeigt werden ohne im Text zitiert zu werden.
Die Verwendung von \textbf{\texttt{\textbackslash nocite\{*\}}} zeigt alle Einträge der Literaturdatenbank im Literaturverzeichnis an.


\subsection{Einheiten mit \texttt{siunitx}}

Das Paket \textbf{siunitx} ermöglicht die einfache und konsistente Darstellung von Einheiten in Texten und Formeln.
Werte mit Einheiten können entweder mit dem Befehl \textbf{\texttt{\textbackslash SI\{Wert\}\{\textbackslash Einheit\}}} oder direkt mit \textbf{\texttt{Wert\textbackslash si\{\textbackslash Einheit\}}} gesetzt werden.

\begin{minipage}[c]{0.6\textwidth}
    \begin{lstlisting}[language={[LaTeX]TeX}, lineskip=2pt]
8,9 \si{\micro\meter}               
45 \si{\degreeCelsius}              
5 \si{\milli\meter\per\second}      
5 \si{\milli\meter\div\second}      
5 \si{\frac{\milli\meter}{\second}} 
\SI{44444,2}{\giga\meter\per\minute}    
    \end{lstlisting}
\end{minipage}
\hfill
\begin{minipage}[c]{0.3\textwidth}
    \begin{itemize}[itemsep=2pt, label=$\rightarrow$]
        \item 8,9 \si{\micro\meter}
        \item 45 \si{\degreeCelsius}
        \item 5 \si{\milli\meter\per\second}
        \item 5 \si{\milli\meter\div\second}
        \item 5 \si{\frac{\milli\meter}{\second}}
        \item \SI{44444,2}{\giga\meter\per\minute}
    \end{itemize}
\end{minipage}

In der Präambel des Dokuments kann mit \texttt{\textbf{\textbackslash sisetup\{output-decimal-marker = \{,\}\}}} das Dezimaltrennzeichen global auf ein Komma und ggf. mit \textbf{\texttt{\textbackslash sisetup\{group-separator = \{.\}\}}} das Gruppentrennzeichen auf einen Punkt gesetzt werden.


\subsection{Eigene Befehle (Makros)}
In latex können eigene Befehle definiert werden, um häufig verwendete Befehle oder Texte zu vereinfachen. Dies kann mit dem Befehl \textbf{\texttt{\textbackslash newcommand\{\textbackslash befehl\}[Anzahl-Argumente]\{Definition\}}} erreicht werden.

Darüber hinaus kann mit \textbf{\texttt{\textbackslash renewcommand\{\textbackslash befehl\}[Anzahl-Argumente]\{Definition\}}} ein bereits existierender Befehl überschrieben werden und mit \newline \textbf{\texttt{\textbackslash providecommand\{\textbackslash befehl\}[Anzahl-Argumente]\{Definition\}}} wird der Befehl nur definiert, wenn er noch nicht existiert.

\subsubsection{Makros ohne Argumente}

\begin{minipage}[c]{0.65\textwidth}
    \begin{lstlisting}[language={[LaTeX]TeX}]
\newcommand{\meinText}{Das ist ein Beispiel.}
\meinText
        \end{lstlisting}
\end{minipage}
\hfill
\begin{minipage}[c]{0.3\textwidth}
    \newcommand{\meinText}{Das ist ein Beispiel.}
    \meinText
\end{minipage}

\subsubsection{Makros mit Argumenten}

\begin{minipage}[c]{0.65\textwidth}
    \begin{lstlisting}[language={[LaTeX]TeX}]
\newcommand{\quersumme}[3]{Die Summe aus #1, #2 und #3 ist:\newline\[ #1 + #2 + #3 \]}
\quersumme{5}{6}{9}
        \end{lstlisting}
\end{minipage}
\hfill
\begin{minipage}[c]{0.3\textwidth}
    \newcommand{\quersumme}[3]{Die Summe aus #1, #2 und #3 ist:\newline\[ #1 + #2 + #3 \]}
    \quersumme{5}{6}{9}
\end{minipage}


\end{document}
