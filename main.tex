\documentclass[titlepage,table]{article}

% Sprach-Pakete:
\usepackage[ngerman]{babel}     % Deutsche Silbentrennung, Begriffe & Typografie
\usepackage[T1]{fontenc}        % Korrekte Darstellung & Trennung von Umlauten

% Layout-Einstellungen
\frenchspacing                  % Einheitlicher Abstand nach Satzzeichen (keine extra Leerzeichen)
\tolerance=9000                 % Erlaubt größere Lücken zwischen Wörtern, um unschöne Umbrüche zu vermeiden
\usepackage{parskip}            % Entfernt Absatzeinrückung und fügt Abstand zwischen Absätzen hinzu
\usepackage{lmodern}            % Schriftart Latin Modern
\usepackage{titlesec}           % Benutzerdefinierte Überschriften
\usepackage{lastpage}           % Gesamtseitenzahl abrufen
\usepackage{fancyhdr}           % Für individuelle Kopf- und Fußzeilen -> s.u.
\usepackage{geometry}           % Ermöglicht benutzerdefinierte Seitenränder
    \geometry{a4paper, left=25mm, right=25mm, top=3.5cm, bottom=2.5cm} % Ränder setzen

% praktische Pakete
\usepackage{hyperref}           % Aktiviert klickbare Links (URLs, Querverweise, Inhaltsverzeichnis)
\usepackage{color}              % Ermöglicht farbige Darstellung von Text und Elementen
\usepackage{comment}            % große Textblöcke auskommentieren -> \begin{comment} ... \end{comment}
\usepackage{siunitx}            % Einheiten -> \SI{wert}{\einheit}
\usepackage{float}              % Positionierung von Gleitobjekten (Bilder, Tabellen) -> \begin{figure}[H]

% Bilder / PDFs
\usepackage{graphicx}           % Fügt Bilder (JPG, PNG, PDF) in LaTeX-Dokumente ein    -> \includegraphics[width=...]{bild.png}
\usepackage[inkscapeformat=png]{svg} % Ermöglicht SVG-Grafiken, konvertiert zu PNG
\usepackage{subcaption}         % Ermöglicht mehrere Teilabbildungen mit Untertiteln
\usepackage{pdfpages}           % Einfügen von PDF-Dokumenten -> \includepdf[pages={...}]{dokument.pdf}
\usepackage{fancybox}           % Schöne Boxen für Bilder -> \shadowbox{...}

% Listen
\usepackage{enumitem}           % Ermöglicht Anpassungen von Listen -> \begin{itemize}[label=...]
\usepackage{multicol}           % Ermöglicht mehrspaltige Listen -> \begin{multicols}{2} ... \end{multicols}

% Tabellen
\usepackage{hhline}             % Erlaubt doppelte und selektive Linien in Tabellen
\usepackage{multirow}           % Erlaubt das Zusammenfassen mehrerer Zeilen in Tabellen
\usepackage{xcolor}             % Ermöglicht das Färben von Tabellenzellen -> \cellcolor{color} und viele weitere Farboptionen
\usepackage{booktabs}           % Schönere Tabellenlinien -> \toprule, \midrule, \bottomrule
\usepackage{longtable}          % Tabellen über mehrere Seiten -> \begin{longtable} ... \end{longtable}
\usepackage{array}              % Ermöglicht das Anpassen von Tabellenspalten -> \begin{tabular}{>{\centering\arraybackslash}p{2cm}}

% Quellcode
\usepackage{listings, pmboxdraw} % Quellcode einbinden und optisch ansprechend rahmen
\lstset{
    %language={[LaTeX]TeX},             % Programmiersprache
    frame=single,                       % Rahmen um den Code
    numbers=left,                       % Zeilennummern links
    stepnumber=1,                       % Jede Zeile nummerieren
    lineskip=0pt,                       % Zeilenabstand anpassen
    numberstyle=\footnotesize,          % Kleine Zeilennummern
    basicstyle=\ttfamily,               % Monospace-Schriftart
    keywordstyle=\color{blue},          % Keywords blau
    commentstyle=\color{gray},          % Kommentare grau
    stringstyle=\color{green},          % Strings grün
    backgroundcolor=\color{white},      % Hintergrundfarbe
    showstringspaces=false,             % Keine Leerzeichen in Strings
    captionpos=b,                       % Position unter dem Code
    breaklines=true,                    % Zeilenumbruch
}

% Mathematik
\usepackage{amsmath}            % Mathematische Symbole, align-Umgebung (&=)
\usepackage{amssymb}            % Extra Mathe-Symbole (z. B. Mengen, Relationen)
\usepackage[makeroom]{cancel}   % Streicht mathematische Ausdrücke durch (mit zusätzlichem Raum)
\usepackage{bm}                 % Fette Symbole in Matheumgebungen -> \bm{...}

% Literaturverzeichnis
\usepackage{cite}               % Verbessert Zitierstil für BibTeX

% Abkürzungsliste
\usepackage{acronym} % -> \ac{...} 

% Anhänge
\usepackage[toc,page]{appendix} % Ermöglicht Anhänge mit eigener Überschrift

%-----------------------------------------------------------------------------------------------------------------------------------------------------------------------------------

\begin{document}

% Deckblatt
% \includepdf{kapitel/deckblatt.pdf}                 % komplettes Deckblatt als PDF einfügen
\begin{titlepage}  % Deckblatt als eigene Seite
    \begin{center}

        \vspace*{0.5cm}  % Abstand nach oben

        % Logo
        \includegraphics[width=4cm]{anlagen/bilder/logo.png}
        \vspace{1cm}  % Abstand nach dem Logo

        % Titel und Untertitel
        {\Huge \textbf{\LaTeX{} Anleitung}} \\[1.5cm]

        % Horizontale Linie für Trennung 
        \rule{12cm}{0.5pt} \\[1.5cm]  % 12cm lange, 0.5pt dicke Linie

        % Zusätzliche Infos (Autor, Datum, etc.)
        \textbf{\Large Autor:} {\Large Simon Vehreschild} \\[0.5cm]
        % \textbf{\Large Betreuer:} \Large Dr. Beispiel \\[0.5cm]
        \textbf{\Large Datum:} {\Large \today} \\[3cm]

        \vfill

        {\large \LaTeX{} ist ein Textsatzsystem, das speziell für wissenschaftliche Arbeiten entwickelt wurde.
            Auf den folgenden Seiten finden Sie einige Informationen, die Ihnen die Arbeit mit \LaTeX{} erleichtern sollen.} \\[3cm]

        \vfill

        {\large Ich danke Tom Arlt, der mir die Grundlagen von \LaTeX{} beigebracht hat.}

        \vfill

        {\large Die Inhalte sind ausschließlich für den privaten Gebrauch bestimmt und dürfen nicht weiterverbreitet oder vervielfältigt werden.}

        \vfill  % Füllt den restlichen Platz, um den Text in der Mitte zu halten
    \end{center}
\end{titlepage}                            % Deckblatt als LaTeX-Datei einfügen (selber erstellt)
% %-------------------------------------------------------------
\begin{titlepage}
    %-------------------------------------------------------------
    \begin{center}
        {\Large\bf Inbetriebnahme Messtechnik}\\[3cm]

        {\bf Inbetriebnahme}\\
        für \\
        Gassensorik\\[1.5cm]

        an der\\
        Hochschule Niederrhein\\
        Fachbereich Elektrotechnik und Informatik\\
        Studiengang {\em Elektrotechnik}\\[3cm]

        vorgelegt von\\
        Tom Arlt und Simon Vehreschild\\
        1335730 und 1543175\\
        Gruppe 12-6\\[3cm]
        Datum: \today\\[3cm]

    \end{center}
\end{titlepage}

\pagestyle{empty}
\newpage

% Kopf- und Fußzeilen
\pagestyle{fancy}                                    % Aktiviert individuelle Kopf- und Fußzeilen
\fancyfoot{}                                         % Löscht alle Standard-Fußzeilen
\fancyfoot[C]{Seite \thepage~von~\pageref{LastPage}} % Fußzeile mit "Seite X von Y"     % Praktisch auch:[RO,LE]
\fancyfoot[R]{Simon Vehreschild}                     % Rechte Seite: Autorname
\fancyfoot[L]{\LaTeX{} Anleitung}                    % Linke Seite: Dokumentname#
\renewcommand{\footrulewidth}{0.4pt}                 % Linie über der Fußzeile
\fancyhead{}                                         % Löscht alle Standard-Kopfzeilen
\renewcommand{\headrulewidth}{0.4pt}                 % Linie unter der Kopfzeile
\fancyhead[L]{\includegraphics[width=1.2cm]{anlagen/bilder/logo.png}} % Logo links in der Kopfzeile
\fancyhead[R]{\nouppercase{\leftmark}}                             % Kopfzeile: Kapitelname

% Inhaltsverzeichnis
\setcounter{tocdepth}{3}                            % Inhaltsverzeichnis bis zur Ebene 3 anzeigen
\setcounter{secnumdepth}{3}                         % Nummerierung bis zur Ebene 3
\tableofcontents
\newpage
\listoftables
\listoffigures
\newpage

% Kapitel in separaten Dateien
\section{Erste Schritte}
\subsection{Einstellungen und Struktur}

Zunächst werden die Grundeinstellungen für das Layout des Dokuments festgelegt. Dazu wird eine \LaTeX{}-Datei \textbf{main.tex} erstellt, in welcher die Dokumentenklasse, die verwendeten Pakete und die Layout-Einstellungen festgelegt werden. Außerdem sind dort das Deckblatt, die Kopf- und Fußzeilen und das Inhaltsverzeichnis zu definieren, sowie die einzelnen Kapitel einzubinden.

\subsubsection{Festlegen der Dokumentenklasse}
Mit dem Befehl \textbf{\texttt{\textbackslash documentclass[Option1, Option2, ...]\{Dokumentenklasse\}}} wird die Dokumentenklasse festgelegt außerdem können weitere Optionen angegeben werden.

Mögliche Dokumentenklassen sind:
\begin{table}[h]
    \centering
    \begin{tabular}{ll}
        \toprule
        \textbf{Klasse}  & \textbf{Beschreibung}                                                               \\
        \midrule
        \texttt{article} & Kurze Texte, wissenschaftliche Artikel, Berichte,                                   \\
                         & Unterstützt \textbackslash section\{...\}  aber keine \textbackslash chapter\{...\} \\
        \texttt{report}  & Längere Dokumente mit Kapiteln (z. B. Abschlussarbeiten)                            \\
                         & Unterstützt \textbackslash chapter\{...\}                                           \\
        \texttt{book}    & Bücher mit Kapiteln, Abschnitten und Teilen                                         \\
                         & Unterstützt \textbackslash part\{...\}                                              \\
        \texttt{letter}  & Briefe                                                                              \\

        \texttt{beamer}  & Präsentationen (wie PowerPoint)                                                     \\
        \bottomrule
    \end{tabular}
    \caption{Standard-Dokumentenklassen für \texttt{\textbackslash documentclass}}
    \label{tab:dokumentklassen}
\end{table}

Mögliche Optionen sind:
\begin{table}[h]
    \centering
    \begin{tabular}{ll}
        \toprule
        \textbf{Option}                        & \textbf{Beschreibung}                                              \\
        \midrule
        \texttt{...pt}                         & Schriftgröße (Standard: 10pt)                                      \\
        \texttt{a4paper, b5paper, letterpaper} & Papierformat (A4, B5, Letter, usw.)                                \\
        \texttt{onecolumn, twocolumn}          & Einspaltig, zweispaltig (Standard: onecolumn)                      \\
        \texttt{titlepage, notitlepage}        & Eigene Titelseite oder nicht    (Standard: notitlepage)            \\
        \texttt{twoside, oneside}              & Doppelseitiges oder einseitiges Layout (Bücher vs. Artikel)        \\
        \texttt{landscape}                     & Querformat                                                         \\
        \texttt{draft, final}                  & Entwurfsmodus (keine Bilder) oder finale Version (Standard: final) \\
        \texttt{fleqn}                         & Mathe-Formeln linksbündig (Standard: zentriert)                    \\
        \texttt{leqno, reqno}                  & Gleichungsnummern links oder rechts (`amsmath`) (Standard: reqno)  \\
        \bottomrule
    \end{tabular}
    \caption{Wichtige Optionen für \texttt{\textbackslash documentclass}}
    \label{tab:documentclass-options}
\end{table}


\subsubsection{Inkludieren von Paketen}
Der Befehl \textbf{\texttt{\textbackslash usepackage[Optionen]\{Paketname\}}} ermöglicht das Einbinden von Paketen, die zusätzliche Funktionen und Einstellungen bereitstellen.
Eine Übersicht an Paketen und deren Funktionen ist in der bereitgestellten main.tex Datei zu finden.

\subsection{Öffnen und Schließen eines Dokuments}
Das eigentliche Dokument wird mit dem Befehl \textbf{\texttt{\textbackslash begin\{document\}}} eröffnet und mit dem Befehl \textbf{\texttt{\textbackslash end\{document\}}} beendet.
Alle Anweisungen davor dienen lediglich der Vorbereitung des Dokuments. Alle Anweisungen danach werden ignoriert. Alle Inhalte der folgenden Unterpunkte sind innerhalb dieser beiden Befehle zu platzieren.

\subsection{Einbinden / Erstellen des Deckblatts}
In der Regel wird ein Deckblatt benötigt, welches Informationen wie den Titel, den Autor, das Datum und ein Logo enthält. Dieses kann entweder als PDF-Datei eingebunden, falls ein bereits erstelltes Deckblatt verwendet werden soll oder als neue \LaTeX{}-Datei erstellt werden.
Das Einbinden einer PDF-Datei erfolgt mit dem Befehl \textbf{\texttt{\textbackslash includepdf\{Pfad/zur/Datei.pdf\}}} (siehe auch: \ref{sec:einbinden_von_pdf_dateien}). Für diesen Befehl wird das Paket \textbf{pdfpages} benötigt.

Alternativ kann das Deckblatt auch als \LaTeX{}-Datei eingebunden werden, indem der Inhalt des Deckblatts in eine neue Datei geschrieben und mit dem Befehl \textbf{\texttt{\textbackslash input\{Pfad/zur/Datei.tex\}}} eingebunden wird (siehe auch: \ref{sec:einbinden_von_latex_dateien}). Hilfreich ist die Funktion \textbf{\texttt{\textbackslash begin\{titlepage\}}} und \textbf{\texttt{\textbackslash end\{titlepage\}}} um zwischen diesen Befehlen das Deckblatt zu erstellen. In dem Ordner \textbf{kapitel} sind zwei Beispieldateien zu finden.

\subsection{Anpassen der Kopf- und Fußzeilen}
Die Kopf- und Fußzeilen können mit dem Paket \textbf{fancyhdr} besonders elegant angepasst werden. Zunächst wird mit dem Befehl \textbf{\texttt{\textbackslash pagestyle\{fancy\}}} die Verwendung von individuellen Kopf- und Fußzeilen aktiviert. Anschließend werden mit den Befehlen \textbf{\texttt{\textbackslash fancyfoot\{\}}} und \textbf{\texttt{\textbackslash fancyhead\{\}}} die Standardeinstellungen für Fuß- und Kopfzeile gelöscht. Die Befehle \textbf{\texttt{\textbackslash fancyfoot[Position]\{Text\}}} und \textbf{\texttt{\textbackslash fancyhead[Position]\{Text\}}} ermöglichen das Hinzufügen von Text oder Bildern in die Fuß- oder Kopfzeile.

Als Position sind die folgenden Angaben möglich:
\begin{itemize}
  \item \textbf{C} - zentriert
  \item \textbf{L} - linksbündig
  \item \textbf{R} - rechtsbündig
  \item \textbf{C} / \textbf{L} / \textbf{R} in Kombination mit \textbf{E} oder \textbf{O} für gerade oder ungerade Seitenzahlen möglich
\end{itemize}

Ist eine Seitennummerierung in der Form Seite X von Y gewünscht, kann dies mit dem Befehl \textbf{\texttt{\textbackslash fancyfoot[C]\{Seite \textbackslash thepage$\thicksim$von$\thicksim$\textbackslash pageref\{LastPage\}\}}} ermöglicht werden. Hierbei wird das Paket \textbf{lastpage} benötigt.

Soll die Kopfzeile den aktuellen \textbackslash section\{\} Namen anzeigen, kann dies mit dem Befehl \\ \textbf{\texttt{\textbackslash fancyhead[R]\{\{\textbackslash leftmark\}\}}} erreicht werden.

Eine Trennlinie zwischen Fußzeile und Dokument, sowie zwischen Kopfzeile und Dokument kann mit den Befehlen \textbf{\texttt{\textbackslash renewcommand\{\textbackslash footrulewidth\}\{...pt\}}} und \textbf{\texttt{\textbackslash renewcommand\{\textbackslash headrulewidth\}\{...pt\}}} eingefügt oder mit 0pt gelöscht werden.

Manchmal ist es gewünscht auf einer Seite keine Kopf- und/oder Fußzeile zu haben. Dies ist möglich mit:
\begin{itemize}
  \item \textbf{\texttt{\textbackslash thispagestyle\{empty\}}} - Keine Kopf- oder Fußzeile
  \item \textbf{\texttt{\textbackslash thispagestyle\{plain\}}} - Normale Fußzeile ohne Kopfzeile
\end{itemize}

\newpage

\subsection{Erstellen des Inhaltsverzeichnisses (und weiterer Verzeichnisse)}
\label{sec:inhaltsverzeichnis}
Das Inhaltsverzeichnis wird mit dem Befehl \textbf{\texttt{\textbackslash tableofcontents}} erstellt. Die Tiefe des Inhaltsverzeichnis kann mit dem Befehl \textbf{\texttt{\textbackslash setcounter\{tocdepth\}\{X\}}} festgelegt werden. Dabei steht X für die \underbar{maximale Tiefe} des Inhaltsverzeichnisses und kann folgende Werte annehmen:

\begin{table}[H]
  \centering
  \begin{tabular}{cll}
    \toprule
    \textbf{Wert (X)} & \textbf{Maximal nummerierte Ebene}                        \\
    \midrule
    -1                & Keine Nummerierung                                        \\
    0                 & Nur \texttt{\textbackslash part\{\}}                      \\
    1                 & Bis \texttt{\textbackslash chapter\{\}}                   \\
    2                 & Bis \texttt{\textbackslash section\{\}}                   \\
    3                 & Bis \texttt{\textbackslash subsection\{\}} (Standardwert) \\
    4                 & Bis \texttt{\textbackslash subsubsection\{\}}             \\
    5                 & Bis \texttt{\textbackslash paragraph\{\}}                 \\
    6                 & Bis \texttt{\textbackslash subparagraph\{\}}              \\
    \bottomrule
  \end{tabular}
  \caption{Nummerierungstiefe mit \texttt{secnumdepth}}
  \label{tab:secnumdepth}
\end{table}

Für die Anwendung dieses Befehls ist darauf zu achten, dass \textbf{\texttt{\textbackslash setcounter\{tocdepth\}\{X\}}} vor \textbf{\texttt{\textbackslash tableofcontents}} platziert werden muss.

Sollen weitere Verzeichnisse für Abbildungen, Tabellen, etc. erstellt werden, so kann dies mit den Befehlen \textbf{\texttt{\textbackslash listoffigures}}, \textbf{\texttt{\textbackslash listoftables}}, etc. erreicht werden.

Darüber hinaus sind weitere Einstellungen mit den Paketen \textbf{tocloft}, \textbf{titletoc} und \textbf{minitoc} möglich.

\subsection{Einbinden von Kapiteln}
Bei langen und umfangreichen Dokumenten ist es besonders sinnvoll, die einzelnen Kapitel in separaten Dateien zu erstellen und diese dann in das Hauptdokument einzubinden. Dies geschieht mit dem Befehl \textbf{\texttt{\textbackslash input\{Pfad/zur/Datei.tex\}}} (siehe auch: \ref{sec:einbinden_von_latex_dateien}). Dabei ist darauf zu achten, dass der Pfad relativ zum Hauptdokument angegeben wird.

Ich persönlich strukturiere meine \LaTeX{}-Dateien in einem eigenen Ordner (kapitel) und binde sie in einer dafür vorgesehenen Datei (uebersicht.tex) ein, welche in der main.tex Datei eingefügt wird. So muss die Datei main.tex nach einmaliger Einrichtung nicht mehr verändert werden.


\subsection{Abkürzungsverzeichnis}
\label{sec:abkuerzungsverzeichnis_erklärung}
In \LaTeX{} kann ein Abkürzungsverzeichnis mit Hilfe des Pakets \textbf{acronym} erstellt werden.

\subsubsection{Erstellen der Abkürzungen}
Abkürzungen werden in der \texttt{\textbf{acronym}}-Umgebung definiert. Dabei wird der Befehl \newline \textbf{\texttt{\textbackslash acro\{Abkürzung\}\{Definition\}}} verwendet.
Damit die Übersichtlichkeit gewahrt bleibt, werden die Abkürzungen (inklusive \texttt{acronym}-Umgebung) in einer separaten Datei (abkuerzungen.tex) erstellt und in das Hauptdokument (main.tex) eingebunden.

Ein Beispiel für ein Abkürzungsverzeichnis folgt:

\begin{lstlisting}[language={[LaTeX]TeX}]
\section*{Abkurzungsverzeichnis}
\label{sec:abkuerzungsverzeichnis}
\begin{acronym}
  \acro{URL}{Uniform Resource Locator}
  \acro{USB}{Universal Serial Bus}
  \acro{GPS}{Global Positioning System}
  \acro{z.B.}{zum Beispiel}
\end{acronym}
\end{lstlisting}


\subsubsection{Verwenden der Abkürzungen}
Um eine Abkürzung im Text zu verwenden gibt es folgende Befehle:

\begin{table}[H]
  \centering
  \begin{tabular}{>{\bfseries}cll}
    \toprule
    \textbf{Befehl}                 & \textbf{Beschreibung}               & \textbf{Beispiel} \\
    \midrule
    \textbackslash ac\{Abkürzung\}  & Abkürzung im Text (erster Aufruf)   & \ac{URL}          \\
    \textbackslash ac\{Abkürzung\}  & Abkürzung im Text (weitere Aufrufe) & \ac{URL}          \\
    \textbackslash acl\{Abkürzung\} & Vollständige Form ohne Abkürzung    & \acl{URL}         \\
    \textbackslash acf\{Abkürzung\} & Ausgabe wie erster Aufruf           & \acf{URL}         \\
    \textbackslash acs\{Abkürzung\} & Nur Abkürzung (auch erster Aufruf)  & \acs{USB}         \\
    \textbackslash acp\{Abkürzung\} & Abkürzung im Plural                 & \acp{URL}         \\
    \bottomrule
  \end{tabular}
  \caption{Befehle zum Verwenden von Abkürzungen im Text}
  \label{tab:acronyms_usage}
\end{table}

Alle Abkürzungen enthalten einen Link zum Abkürzungsverzeichnis, damit der Leser dort die Definition nachschlagen kann.

\subsubsection{Einbinden des Abkürzungsverzeichnisses}
Das Abkürzungsverzeichnis wird automatisch an der Stelle eingefügt, an der die \texttt{\textbf{acronym}}-Umgebung definiert wurde. Im Abkürzungsverzeichnis werden die Abkürzungen mit einem Link zur ersten Verwendung im Text versehen (falls eine vorhanden ist).


\subsection{Literaturverzeichnis}
\label{sec:literaturverzeichnis_erklärung}
In \LaTeX{} kann ein \hyperref[sec:literaturverzeichnis]{Literaturverzeichnis} mit Hilfe von BibTeX erstellt werden.

\subsubsection{Erstellen einer \textbf{\texttt{.bib}}-Datei}
Zunächst werden die Literaturangaben in einer \textbf{\texttt{.bib}}-Datei (literatur.bib) gespeichert. Vorzugsweise im gleichen Ordner wie die Hauptdatei.

Es folgt ein Beispiel für eine \textbf{\texttt{literatur.bib}}-Datei:

\begin{lstlisting}[language={[LaTeX]TeX}]
@book{buch1,
  author    = {Anna Beispiel and Bernd Mustermann and Carla Demo},
  title     = {Fortgeschrittene LaTeX-Techniken},
  publisher = {Beispiel-Verlag},
  year      = {2018}
}

@article{artikel1,
  author  = {John Doe},
  title   = {Eine neue Theorie zur Typografie},
  journal = {Journal fur Wissenschaftliche Typografie},
  volume  = {12},
  number  = {3},
  pages   = {45-67},
  year    = {2015}
}

@inproceedings{konferenz1,
  author    = {Jane Smith and Paul Example},
  title     = {Typografische Verbesserungen mit LaTeX},
  booktitle = {Proceedings der Konferenz fur Dokumentenverarbeitung},
  pages     = {123-130},
  year      = {2022}
}

@misc{web1,
  author    = {Overleaf},
  title     = {LaTeX-Wiki},
  year      = {2025},
  note      = {URL: \url{https://www.overleaf.com/learn/latex} (abgerufen am 10. Marz 2025)}
}

@phdthesis{thesis1,
  author  = {Alex Student},
  title   = {Untersuchungen zur mathematischen Satztechnik},
  school  = {Universitat Beispielstadt},
  year    = {2023}
}
\end{lstlisting}

\subsubsection{Einbinden des Literaturverzeichnisses}
Zunächst wird der Stil des Literaturverzeichnisses mit dem Befehl \textbf{\texttt{\textbackslash bibliographystyle\{Stil\}}} festgelegt.
Es stehen folgende Stile zur Auswahl:
\begin{itemize}
  \item \textbf{plain} – Alphabetische Sortierung nach Autoren (Standardstil).
  \item \textbf{unsrt} – Reihenfolge der Zitierung im Text (nicht alphabetisch).
  \item \textbf{alpha} – Alphabetische Sortierung nach Autoren, aber mit Kurzreferenzen wie `[Ein05]`.
  \item \textbf{abbrv} – Wie `plain`, aber mit abgekürzten Vornamen und Titeln.
  \item \textbf{apalike} – Ähnlich wie APA-Zitierstil (Autor-Jahr statt Nummerierung).
\end{itemize}

Anschließend wird die \textbf{\texttt{.bib}}-Datei mit dem Befehl \textbf{\texttt{\textbackslash bibliography\{Pfad/zur/Datei\}}} eingebunden.

Nun können Zitate im Text eingefügt werden (siehe auch: \nameref{sec:zitieren}).

Damit alle Einträge aus der literatur.bib-Datei im Literaturverzeichnis erscheinen, kann der Befehl \textbf{\texttt{\textbackslash nocite\{*\}}} verwendet werden.
\section{Wichtige Befehle}
\subsection{Kommentare}
Kommentare werden in Latex mit einem Prozentzeichen eingeleitet. Alles, was nach dem Prozentzeichen steht in der selben Zeile steht, wird von Latex ignoriert. Mehrzeilige Kommentare sind mit den Befehlen \textbf{\texttt{\textbackslash begin\{comment\}}} und \textbf{\texttt{\textbackslash end\{comment\}}} möglich. Es wird das Paket \textbf{comment} benötigt.

\begin{lstlisting}
    % Dies ist ein einzeiliger Kommentar
    Text ohne Kommentar.
    
    \begin{comment}
    Dies ist ein
    langer Kommentarblock
    \end{comment}
\end{lstlisting}

\subsection{Textformatierung}
\subsubsection{Schriftgröße}
Die Schriftgröße kann mit den folgenden Befehlen geändert werden:

\begin{table}[h]
    \centering
    \begin{tabular}{lll}
        \toprule
        \textbf{Befehl}             & \textbf{Größe}           & \textbf{Beispieltext} \\
        \midrule
        \textbackslash tiny         & Sehr klein               & {\tiny Text}          \\
        \textbackslash scriptsize   & Kleiner als normal       & {\scriptsize Text}    \\
        \textbackslash footnotesize & Kleine Schrift           & {\footnotesize Text}  \\
        \textbackslash small        & Etwas kleiner als normal & {\small Text}         \\
        \textbackslash normalsize   & Standardgröße            & {\normalsize Text}    \\
        \textbackslash large        & Etwas größer als normal  & {\large Text}         \\
        \textbackslash Large        & Noch größer              & {\Large Text}         \\
        \textbackslash LARGE        & Sehr groß                & {\LARGE Text}         \\
        \textbackslash huge         & Sehr große Schrift       & {\huge Text}          \\
        \textbackslash Huge         & Extrem große Schrift     & {\Huge Text}          \\
        \bottomrule
    \end{tabular}
    \caption{Schriftgrößen}
    \label{tab:schriftgroessen}
\end{table}

Durch Verwendung von geschweiften Klammern kann der Befehl auf einen Textbereich angewendet werden:
\textbf{\texttt{\{\textbackslash large großer Text\}}}

\subsubsection{Schriftstil}
Der Schriftstil kann mit den folgenden Befehlen geändert werden:

\begin{table}[h]
    \centering
    \begin{tabular}{lll}
        \toprule
        \textbf{Befehl}               & \textbf{Wirkung}                 & \textbf{Beispieltext} \\
        \midrule
        \textbackslash textbf\{\}     & Fett                             & \textbf{Text}         \\
        \textbackslash textit\{\}     & Kursiv                           & \textit{Text}         \\
        \textbackslash texttt\{\}     & Monospace (Typewriter)           & \texttt{Text}         \\
        \textbackslash underline\{\}  & Unterstrichen                    & \underline{Text}      \\
        \textbackslash emph\{\}       & Hervorgehoben (Kursiv o. Fett)   & \emph{Text}           \\
        \textbackslash textsc\{\}     & Kapitälchen (Small Caps)         & \textsc{Text}         \\
        \textbackslash textnormal\{\} & Normale Schrift                  & \textnormal{Text}     \\
        \textbackslash textsf\{\}     & Serifenlose Schrift (Sans-Serif) & \textsf{Text}         \\
        \textbackslash textsl\{\}     & Schräggestellt (Slanted)         & \textsl{Text}         \\
        \textbackslash textmd\{\}     & Mittlere Schriftstärke (Medium)  & \textmd{Text}         \\
        \textbackslash textup\{\}     & Aufrechte Schrift (Upright)      & \textup{Text}         \\
        \bottomrule
    \end{tabular}
    \caption{Schriftstil-Optionen}
    \label{tab:schriftstile}
\end{table}

Die Schriftstile können mit den Schriftgrößen kombiniert und mit anderen Schriftstilen überlagert werden.


\subsubsection{Schriftfarbe}
Die Schriftfarbe kann auf folgende drei Arten geändert werden:

\begin{itemize}
    \item \textbf{\texttt{\textbackslash textcolor\{red\}\{Roter Text\}}}
          \hspace{2.83cm} --- \hspace{1cm} \textcolor{red}{Roter Text}

    \item \textbf{\texttt{\textbackslash textcolor[rgb]\{0,0.5,1\}\{RGB-Farbe\}}}
          \hspace{1.35cm} --- \hspace{1cm} \textcolor[rgb]{0,0.5,1}{RGB-Farbe}

    \item \textbf{\texttt{\textbackslash textcolor[HTML]\{00FF00\}\{Hex-Farbcode\}}}
          \hspace{0.8cm} --- \hspace{1cm} \textcolor[HTML]{00FF00}{Hex-Farbcode}
\end{itemize}

Für diese Varianten wird das Paket \textbf{xcolor} benötigt. Darüberhinaus ist zu beachten, dass textcolor einen auf 1 normierten RGB-Wert erwartet.


\subsubsection{Rahmen und Boxen}
Textpassagen können mit folgenden Befehlen in Rahmen oder Boxen gesetzt werden:

\begin{table}[h]
    \centering
    \begin{tabular}{lll}
        \toprule
        \textbf{Befehl}                                  & \textbf{Wirkung}                 & \textbf{Darstellung}   \\
        \midrule
        \texttt{\textbackslash fbox\{Text\}}             & Rahmen um den Text               & \fbox{Text}            \\
        \texttt{\textbackslash framebox(45,10)\{Text\}}  & Box mit fester Breite und Höhe   & \framebox(45,10){Text} \\
        \texttt{\textbackslash colorbox\{cyan\}\{Text\}} & Farbige Box mit Hintergrundfarbe & \colorbox{cyan}{Text}  \\
        \bottomrule
    \end{tabular}
    \caption{Rahmen und Boxen}
    \label{tab:boxen}
\end{table}


\subsubsection{Spezialfälle}

\begin{table}[h]
    \centering
    \begin{tabular}{lll}
        \toprule
        \textbf{Funktion}       & \textbf{Anweisung}                  & \textbf{Darstellung}             \\
        \midrule
        Text hochstellen        & \texttt{x\textasciicircum\{4\}}     & $x^{4}$                          \\
        Text tiefstellen        & \texttt{H\_\{2\}O}                  & $H_{2}O$                         \\
        Geschütztes Leerzeichen & \texttt{Hallo\textasciitilde Welt!} & Hallo~Welt! (Verhindert Umbruch) \\
        \bottomrule
    \end{tabular}
    \caption{Spezial Befehle zur Textformatierung}
    \label{tab:textformatierung}
\end{table}

\newpage

\subsection{Abstandsformatierung}




% Literaturverzeichnis
\newpage
\bibliographystyle{plain}  % Stil des Literaturverzeichnisses
\bibliography{literatur}   % Verknüpfung mit literatur.bib
\nocite{*}                 % Alle Einträge aus literatur.bib anzeigen
\label{sec:literaturverzeichnis}

\end{document}
