\documentclass[titlepage,table]{article}
%\usepackage{titlesec}

% language stuff
%\usepackage{ngerman}           % deutsche Überschriften etc.
\usepackage[utf8]{inputenc} % direkte Einbgabe von Umlauten
\usepackage[ngerman]{babel} % this is needed for umlauts
\usepackage[T1]{fontenc}    % this is needed for correct output of umlauts in pdf

% Layout-Einstellungen
\frenchspacing                % no extra space after periods
\usepackage{parskip}          % paragraph gaps instead of indentation
\usepackage{times}            % default font Times
\tolerance=9000               % avoid words across right border

% miscellaneous
\usepackage{graphicx}         % graphics
\usepackage{subcaption}       % subfigures
\usepackage{hhline}           % double lines in tables
\usepackage{amsfonts}         % real numbers etc.
\usepackage{amsmath}          % Gleichungen untereinander &=
\newcommand*\diff{\mathop{}\!\mathrm{d}} % dx -> Integrale
\newcommand*\Diff[1]{\mathop{}\!\mathrm{d^#1}}
\usepackage{amssymb}          % more math symbols
\usepackage[makeroom]{cancel} % cross out math formula
\usepackage[rightcaption]{sidecap} % figure captions on the right (optional)
\usepackage{hyperref}         % for URLs
\usepackage{listings, pmboxdraw}         % for code samples
\usepackage{fancyhdr}         % for header line
%\usepackage{color}

% Hier bei Bedarf die Seitenränder einstellen
\usepackage{geometry}
\usepackage{lastpage}
\usepackage[inkscapeformat=png]{svg}

\usepackage[backend=bibtex, sorting=nyt, urldate=long]{biblatex}
\usepackage[toc,page]{appendix}
\usepackage{acronym}
\usepackage{comment}
\usepackage{siunitx}
\usepackage{pdfpages}        % for including pdfs
\usepackage{multirow}       % for multirow in tables
\usepackage[table]{xcolor}     % for coloring cells in tables
%\geometry{a4paper}

\usepackage{booktabs}
\geometry{a4paper,left=25mm,right=25mm, top=3.5cm, bottom=2.5cm} 


% Kopf- und Fußzeile
\fancyhead{} % clear all header fields
\fancyhead[RO,LE]{\leftmark}

\begin{document}
% \includepdf{chapters/Deckblatt.pdf}
% \input{chapters/titlepage}
\tableofcontents
\newpage
\pagestyle{fancy}
\renewcommand{\labelenumi}{\alph{enumi}}
\pagenumbering{arabic}
\renewcommand{\familydefault}{\sfdefault}


% Text

\section{Erste Schritte}
\subsection{Einstellungen und Struktur}

Zunächst werden die Grundeinstellungen für das Layout des Dokuments festgelegt. Dazu wird eine Latex-Datei \textbf{main.tex} erstellt.

Mit dem Befehl \textbf{\texttt{\textbackslash documentclass[Option1, Option2, ...]\{Dokumentenklasse\}}} wird die Dokumentenklasse festgelegt außerdem können weitere Optionen angegeben werden. 

Mögliche Dokumentenklassen sind:
\input{Anlagen/Tabellen/Dokumentklassen.tex}

Mögliche Optionen sind:
\input{Anlagen/Tabellen/Einstellungen_Dokumentklassen.tex}


\section{Fehlt noch:}

%kommentare
fett, kursiv
reffernzen
%\ und {} in anweisungen
%\text, texttt, textbf, textit, underline , emph 
%\input
%Tabellen, Booktabs, Multicolumn, Multirow, Landscape

\end{document}