\section{Erste Schritte}
\subsection{Einstellungen und Struktur}

Zunächst werden die Grundeinstellungen für das Layout des Dokuments festgelegt. Dazu wird eine Latex-Datei \textbf{main.tex} erstellt, in welcher die Dokumentenklasse, die verwendeten Pakete und die Layout-Einstellungen festgelegt werden.

\subsubsection{Festlegen der Dokumentenklasse}
Mit dem Befehl \textbf{\texttt{\textbackslash documentclass[Option1, Option2, ...]\{Dokumentenklasse\}}} wird die Dokumentenklasse festgelegt außerdem können weitere Optionen angegeben werden.

Mögliche Dokumentenklassen sind:
\begin{table}[h]
    \centering
    \begin{tabular}{ll}
        \toprule
        \textbf{Klasse}  & \textbf{Beschreibung}                                                               \\
        \midrule
        \texttt{article} & Kurze Texte, wissenschaftliche Artikel, Berichte,                                   \\
                         & Unterstützt \textbackslash section\{...\}  aber keine \textbackslash chapter\{...\} \\
        \texttt{report}  & Längere Dokumente mit Kapiteln (z. B. Abschlussarbeiten)                            \\
                         & Unterstützt \textbackslash chapter\{...\}                                           \\
        \texttt{book}    & Bücher mit Kapiteln, Abschnitten und Teilen                                         \\
                         & Unterstützt \textbackslash part\{...\}                                              \\
        \texttt{letter}  & Briefe                                                                              \\

        \texttt{beamer}  & Präsentationen (wie PowerPoint)                                                     \\
        \bottomrule
    \end{tabular}
    \caption{Standard-Dokumentenklassen für \texttt{\textbackslash documentclass}}
    \label{tab:dokumentklassen}
\end{table}

Mögliche Optionen sind:
\begin{table}[h]
    \centering
    \begin{tabular}{ll}
        \toprule
        \textbf{Option}                        & \textbf{Beschreibung}                                              \\
        \midrule
        \texttt{...pt}                         & Schriftgröße (Standard: 10pt)                                      \\
        \texttt{a4paper, b5paper, letterpaper} & Papierformat (A4, B5, Letter, usw.)                                \\
        \texttt{onecolumn, twocolumn}          & Einspaltig, zweispaltig (Standard: onecolumn)                      \\
        \texttt{titlepage, notitlepage}        & Eigene Titelseite oder nicht    (Standard: notitlepage)            \\
        \texttt{twoside, oneside}              & Doppelseitiges oder einseitiges Layout (Bücher vs. Artikel)        \\
        \texttt{landscape}                     & Querformat                                                         \\
        \texttt{draft, final}                  & Entwurfsmodus (keine Bilder) oder finale Version (Standard: final) \\
        \texttt{fleqn}                         & Mathe-Formeln linksbündig (Standard: zentriert)                    \\
        \texttt{leqno, reqno}                  & Gleichungsnummern links oder rechts (`amsmath`) (Standard: reqno)  \\
        \bottomrule
    \end{tabular}
    \caption{Wichtige Optionen für \texttt{\textbackslash documentclass}}
    \label{tab:documentclass-options}
\end{table}


\subsubsection{Inkludieren von Paketen}
Der Befehl \textbf{\texttt{\textbackslash usepackage[Optionen]\{Paketname\}}} ermöglicht das Einbinden von Paketen, die zusätzliche Funktionen und Einstellungen bereitstellen.





\section{Fehlt noch:}


%kommentare
fett, kursiv
reffernzen
%\ und {} in anweisungen
% \text, texttt, textbf, textit, underline , emph , farbig mit \textcolor{color}{text}
%\input
%Tabellen, Booktabs, Multicolumn, Multirow, Landscape
%\newcommand{cmd}{def}