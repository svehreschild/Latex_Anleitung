\section{Erste Schritte}
\subsection{Einstellungen und Struktur}

Zunächst werden die Grundeinstellungen für das Layout des Dokuments festgelegt. Dazu wird eine Latex-Datei \textbf{main.tex} erstellt.

Mit dem Befehl \textbf{\texttt{\textbackslash documentclass[Option1, Option2, ...]\{Dokumentenklasse\}}} wird die Dokumentenklasse festgelegt außerdem können weitere Optionen angegeben werden. 

Mögliche Dokumentenklassen sind:
\input{Anlagen/Tabellen/Dokumentklassen.tex}

Mögliche Optionen sind:
\input{Anlagen/Tabellen/Einstellungen_Dokumentklassen.tex}


\section{Fehlt noch:}

%kommentare
fett, kursiv
reffernzen
%\ und {} in anweisungen
%\text, texttt, textbf, textit, underline , emph 
%\input
%Tabellen, Booktabs, Multicolumn, Multirow, Landscape