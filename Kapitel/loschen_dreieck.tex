\subsection{Dreieckspannung}

Im letzten Versuchsteil werden zwei verschiedene Dreieckspannungen mit unterschiedlichen Offsets $u_0$ erzeugt. Die Einstellungen des Signalgenerators sind in \autoref{fig:dreieck_signale} zu sehen. Die Messwerte der Spannungen, die Messbereiche und die Genauigkeit der jeweiligen Messbereiche sind in \autoref{tab:messerwerte_dreieck} aufgeführt. Da das Voltcraft 77 IV Multimeter kein True-RMS Multimeter ist, wird der Effektivwert ohne Gleichanteil ($AC$ bzw. $U_\sim$) nicht korrekt ermittelt, deshalb wurde in \autoref{tab:messerwerte_dreieck} ein theoretischer Anzeigewert berechnet.

\begin{figure}[h]
    \centering
    \begin{subfigure}{0.4\textwidth}
        \centering
        \includegraphics[width=\textwidth]{assets/photos/gen_invert-4.png}
        \caption{Dreiecksignal Typ 1}
        \label{fig:dreieck_signal_t2}
    \end{subfigure}
    \hspace{0.5cm}
    \begin{subfigure}{0.4\textwidth}
        \centering
        \includegraphics[width=\textwidth]{assets/photos/gen_invert-5.png}
        \caption{Dreiecksignal Typ 2}
        \label{fig:dreieck_signal_t1}
    \end{subfigure}
    \caption{Einstellungen des Signalgenerators für die Dreieckspannungen}
    \label{fig:dreieck_signale}
\end{figure}

\input{assets/tables/dreieck_werte.tex}
c
Analog zur Rechteckspannung wird für die Berechnung der Formfaktoren und Crest-Faktoren die \autoref{eq:form_faktor} und \autoref{eq:crest_faktor} verwendet. Der Gleichrichtwert konnte bei Typ 2 mittels des $AC$ Wertes des Fluke 77 IV unter Verwendung der Formel $\overline{|u|} = \frac{U_{\sim Anzeige}}{F_{Sin}}$ berechnet werden. Bei Typ 1 konnte dafür der $DC$ Wert des Voltcraft Multimeters verwendet werden.

Für Typ $1$ und $2$ kann der gemessene Effektivwert ohne Gleichanteil $U_\sim$ des Fluke 77 IV mit \autoref{eq:umrechnung} korrigiert werden. In der $AC$-Einstellung wird der Gleichanteil $U\_$ unterdrückt, somit sind die Messwerte für Typ $1$ und $2$ in dieser Einstellung ähnlich:

\begin{equation}
    \begin{aligned}
        U_{\sim}         & = U_{\sim Anzeige} \cdot \frac{F_{Signal}}{F_{Sin}}                             \\
        U_{\sim \ Typ 1} & = 1,093 \si{\volt} \cdot \frac{1,1589}{\frac{\pi}{\sqrt{8}}} = 1,14 \si{\volt}  \\
        U_{\sim \ Typ 2} & = 1,089 \si{\volt} \cdot \frac{1,1436}{\frac{\pi}{\sqrt{8}}} = 1,121 \si{\volt}
    \end{aligned}
\end{equation}

