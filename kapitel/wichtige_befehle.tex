\section{Wichtige Befehle}
\subsection{Kommentare}
Kommentare werden in Latex mit einem Prozentzeichen eingeleitet. Alles, was nach dem Prozentzeichen in der selben Zeile steht, wird von Latex ignoriert. Mehrzeilige Kommentare sind mit den Befehlen \textbf{\texttt{\textbackslash begin\{comment\}}} und \textbf{\texttt{\textbackslash end\{comment\}}} möglich. Dafür wird das Paket \textbf{comment} benötigt.

\begin{lstlisting}
    % Dies ist ein einzeiliger Kommentar
    Text ohne Kommentar.
    
    \begin{comment}
    Dies ist ein
    langer Kommentarblock
    \end{comment}
\end{lstlisting}

\subsection{Textformatierung}
\subsubsection{Schriftgröße}
Die Schriftgröße kann mit den folgenden Befehlen geändert werden:

\begin{table}[h]
    \centering
    \begin{tabular}{lll}
        \toprule
        \textbf{Befehl}             & \textbf{Größe}           & \textbf{Beispieltext} \\
        \midrule
        \textbackslash tiny         & Sehr klein               & {\tiny Text}          \\
        \textbackslash scriptsize   & Kleiner als normal       & {\scriptsize Text}    \\
        \textbackslash footnotesize & Kleine Schrift           & {\footnotesize Text}  \\
        \textbackslash small        & Etwas kleiner als normal & {\small Text}         \\
        \textbackslash normalsize   & Standardgröße            & {\normalsize Text}    \\
        \textbackslash large        & Etwas größer als normal  & {\large Text}         \\
        \textbackslash Large        & Noch größer              & {\Large Text}         \\
        \textbackslash LARGE        & Sehr groß                & {\LARGE Text}         \\
        \textbackslash huge         & Sehr große Schrift       & {\huge Text}          \\
        \textbackslash Huge         & Extrem große Schrift     & {\Huge Text}          \\
        \bottomrule
    \end{tabular}
    \caption{Schriftgrößen}
    \label{tab:schriftgroessen}
\end{table}

Durch Verwendung von geschweiften Klammern kann der Befehl auf einen Textbereich angewendet werden:
Beispiel: \textbf{\texttt{\{\textbackslash large großer Text\}}}

\subsubsection{Schriftstil}
Der Schriftstil kann mit den folgenden Befehlen geändert werden:

\begin{table}[h]
    \centering
    \begin{tabular}{lll}
        \toprule
        \textbf{Befehl}               & \textbf{Wirkung}                 & \textbf{Beispieltext} \\
        \midrule
        \textbackslash textbf\{\}     & Fett                             & \textbf{Text}         \\
        \textbackslash textit\{\}     & Kursiv                           & \textit{Text}         \\
        \textbackslash texttt\{\}     & Monospace (Typewriter)           & \texttt{Text}         \\
        \textbackslash underline\{\}  & Unterstrichen                    & \underline{Text}      \\
        \textbackslash emph\{\}       & Hervorgehoben (Kursiv o. Fett)   & \emph{Text}           \\
        \textbackslash textsc\{\}     & Kapitälchen (Small Caps)         & \textsc{Text}         \\
        \textbackslash textnormal\{\} & Normale Schrift                  & \textnormal{Text}     \\
        \textbackslash textsf\{\}     & Serifenlose Schrift (Sans-Serif) & \textsf{Text}         \\
        \textbackslash textsl\{\}     & Schräggestellt (Slanted)         & \textsl{Text}         \\
        \textbackslash textmd\{\}     & Mittlere Schriftstärke (Medium)  & \textmd{Text}         \\
        \textbackslash textup\{\}     & Aufrechte Schrift (Upright)      & \textup{Text}         \\
        \bottomrule
    \end{tabular}
    \caption{Schriftstil-Optionen}
    \label{tab:schriftstile}
\end{table}

Die Schriftstile können mit den Schriftgrößen kombiniert und mit anderen Schriftstilen überlagert werden.


\subsubsection{Schriftfarbe}
Die Schriftfarbe kann auf folgende drei Arten geändert werden:

\begin{itemize}
    \item \textbf{\texttt{\textbackslash textcolor\{red\}\{Roter Text\}}}
          \hspace{2.83cm} --- \hspace{1cm} \textcolor{red}{Roter Text}

    \item \textbf{\texttt{\textbackslash textcolor[rgb]\{0,0.5,1\}\{RGB-Farbe\}}}
          \hspace{1.35cm} --- \hspace{1cm} \textcolor[rgb]{0,0.5,1}{RGB-Farbe}

    \item \textbf{\texttt{\textbackslash textcolor[HTML]\{00FF00\}\{Hex-Farbcode\}}}
          \hspace{0.8cm} --- \hspace{1cm} \textcolor[HTML]{00FF00}{Hex-Farbcode}
\end{itemize}

Für diese Varianten wird das Paket \textbf{xcolor} benötigt. Darüberhinaus ist zu beachten, dass textcolor einen auf 1 normierten RGB-Wert erwartet.


\subsubsection{Rahmen und Boxen}
Textpassagen können mit folgenden Befehlen in Rahmen oder Boxen gesetzt werden:

\begin{table}[h]
    \centering
    \begin{tabular}{lll}
        \toprule
        \textbf{Befehl}                                  & \textbf{Wirkung}                 & \textbf{Darstellung}   \\
        \midrule
        \texttt{\textbackslash fbox\{Text\}}             & Rahmen um den Text               & \fbox{Text}            \\
        \texttt{\textbackslash framebox(45,10)\{Text\}}  & Box mit fester Breite und Höhe   & \framebox(45,10){Text} \\
        \texttt{\textbackslash colorbox\{cyan\}\{Text\}} & Farbige Box mit Hintergrundfarbe & \colorbox{cyan}{Text}  \\
        \bottomrule
    \end{tabular}
    \caption{Rahmen und Boxen}
    \label{tab:boxen}
\end{table}


\subsubsection{Spezialfälle}

\begin{table}[h]
    \centering
    \begin{tabular}{lll}
        \toprule
        \textbf{Funktion}       & \textbf{Befehl}                     & \textbf{Darstellung}             \\
        \midrule
        Text hochstellen        & \texttt{x\textasciicircum\{4\}}     & $x^{4}$                          \\
        Text tiefstellen        & \texttt{H\_\{2\}O}                  & $H_{2}O$                         \\
        Geschütztes Leerzeichen & \texttt{Hallo\textasciitilde Welt!} & Hallo~Welt! (Verhindert Umbruch) \\
        \bottomrule
    \end{tabular}
    \caption{Spezial Befehle zur Textformatierung}
    \label{tab:textformatierung}
\end{table}

\subsection{Abstandsformatierung}
\subsubsection{Vertikaler Abstand}
Der vertikale Abstand kann mit den folgenden Befehlen geändert werden:

\begin{table}[h]
    \centering
    \begin{tabular}{lp{10cm}}
        \toprule
        \textbf{Befehl}                           & \textbf{Wirkung}                                                      \\
        \midrule
        \texttt{\textbackslash vspace\{1cm\}}     & Fügt \textbf{1 cm} vertikalen Abstand ein (am Seitenanfang ignoriert) \\
        \texttt{\textbackslash vspace*\{1cm\}}    & Fügt \textbf{1 cm} vertikalen Abstand ein, auch am Seitenanfang       \\
        \texttt{\textbackslash addvspace\{1cm\}}  & Fügt \textbf{1 cm Abstand nur hinzu, wenn vorher keiner war}          \\
        \texttt{\textbackslash smallskip}         & Fügt \textbf{kleinen flexiblen Abstand} ein                           \\
        \texttt{\textbackslash medskip}           & Fügt \textbf{mittleren flexiblen Abstand} ein                         \\
        \texttt{\textbackslash bigskip}           & Fügt \textbf{großen flexiblen Abstand} ein                            \\
        \texttt{\textbackslash vfill}             & Füllt den \textbf{gesamten verbleibenden Platz} mit Abstand           \\
        \midrule
        \texttt{\textbackslash newline}           & Erzwingt einen \textbf{Zeilenumbruch} und beginnt eine neue Zeile     \\
        \texttt{\textbackslash newpage}           & Erzwingt einen \textbf{Seitenumbruch}                                 \\
        \texttt{\textbackslash clearpage}         & Erzwingt einen \textbf{Seitenumbruch} nach allen restlichen Gleitobj. \\
        \texttt{\textbackslash linebreak[prio]}   & Erzwingt einen \textbf{Zeilenumbruch} mit Priorität (1-4)             \\
        \texttt{\textbackslash nolinebreak[prio]} & Verhindert einen \textbf{Zeilenumbruch} mit Priorität (1-4)           \\
        \texttt{\textbackslash pagebreak[prio]}   & Erzwingt einen \textbf{Seitenumbruch} mit Priorität (1-4)             \\
        \texttt{\textbackslash nopagebreak[prio]} & Verhindert einen \textbf{Seitenumbruch} mit Priorität (1-4)           \\
        \bottomrule
    \end{tabular}
    \caption{vertikale Abstandsoptionen}
    \label{tab:vertikale_abstaende}
\end{table}

Eine weitere Möglichkeit, vertikalen Abstand zu erzeugen, ist die Verwendung von Leerzeilen. Diese erzeugen (genau) einen neuen Absatz und somit vertikalen Abstand.
Darüber hinaus kann \texttt{\textbackslash stretch\{X\}} als Argument an \texttt{\textbackslash vspace} übergeben werden, um flexiblen vertikalen Abstand zu erzeugen. Der Wert von \texttt{X} gibt dabei an, wie viel Teile (Verhältnis) des verfügbaren Platzes der Abstand einnehmen soll.

\subsubsection{Horizontaler Abstand}
Der horizontale Abstand kann mit den folgenden Befehlen geändert werden:

\begin{table}[h]
    \centering
    \begin{tabular}{lp{10cm}}
        \toprule
        \textbf{Befehl}                        & \textbf{Wirkung}                                                  \\
        \midrule
        \texttt{\textbackslash hspace\{1cm\}}  & Fügt \textbf{1 cm} horizontalen Abstand ein                       \\
        \texttt{\textbackslash hspace*\{1cm\}} & Fügt \textbf{1 cm} horizontalen Abstand ein, auch am Zeilenanfang \\
        \texttt{\textbackslash kern Xpt}       & Fügt \textbf{X pt Abstand} zwischen Zeichen ein                   \\
        \texttt{\textbackslash quad}           & Fügt \textbf{einfachen Tabulator-Abstand} (1 em) ein              \\
        \texttt{\textbackslash qquad}          & Fügt \textbf{doppelten Tabulator-Abstand} (2 em) ein              \\
        \texttt{\textbackslash enspace}        & Fügt \textbf{halben Leerzeichen-Abstand} (0.5 em) ein             \\
        \texttt{\textbackslash hfill}          & Füllt den \textbf{gesamten verfügbaren horizontalen Platz}        \\
        \midrule
        \texttt{\textbackslash !}              & Entfernt ein \textbf{kleines Leerzeichen} (negativer Abstand)     \\
        \texttt{\textbackslash ignorespaces}   & Entfernt \textbf{alle nachfolgenden Leerzeichen}                  \\
        \bottomrule
    \end{tabular}
    \caption{horizontale Abstandsoptionen}
    \label{tab:horizontale_abstaende}
\end{table}

Eine weitere Möglichkeit, horizontalen Abstand zu erzeugen, ist die Verwendung von Leerzeichen. Diese erzeugen (genau) ein Leerzeichen und somit horizontalen Abstand.
Darüber hinaus kann \texttt{\textbackslash stretch\{X\}} als Argument an \texttt{\textbackslash hspace} übergeben werden, um flexiblen horizontalen Abstand zu erzeugen. Der Wert von \texttt{X} gibt dabei an, wie viel Teile (Verhältnis) des verfügbaren Platzes der Abstand einnehmen soll.

Ein em ist die Breite des Großbuchstaben M in der aktuellen Schriftart und ein pt entspricht etwa 0.35 mm.

\subsubsection{Zentrierung}
In LaTeX können Texte und Objekte zentriert werden. Dafür gibt es die folgenden Befehle:
\begin{table}[h]
    \centering
    \begin{tabular}{lp{6cm}c}
        \toprule
        \textbf{Befehl} & \textbf{Funktion}                                              \\
        \midrule
        \texttt{\textbackslash begin\{center\} ... \textbackslash end\{center\}}
                        & Zentriert alles im Block (Text, Bilder, Tabellen, Formeln)     \\

        \texttt{\textbackslash centering}
                        & Zentriert innerhalb einer Umgebung (Tabellen, Bilder, Formeln) \\
        \bottomrule
    \end{tabular}
    \caption{Möglichkeiten zur Zentrierung}
    \label{tab:center_vs_centering}
\end{table}
