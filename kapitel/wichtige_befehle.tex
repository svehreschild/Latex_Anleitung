\section{Wichtige Befehle}
\subsection{Kommentare}
Kommentare werden in \LaTeX{} mit einem Prozentzeichen eingeleitet. Alles, was nach dem Prozentzeichen in der selben Zeile steht, wird von \LaTeX{} ignoriert. Mehrzeilige Kommentare sind mit den Befehlen \textbf{\texttt{\textbackslash begin\{comment\}}} und \textbf{\texttt{\textbackslash end\{comment\}}} möglich. Dafür wird das Paket \textbf{comment} benötigt.

\begin{lstlisting}
    % Dies ist ein einzeiliger Kommentar
    Text ohne Kommentar.
    
    \begin{comment}
    Dies ist ein
    langer Kommentarblock
    \end{comment}
\end{lstlisting}

\subsection{Textformatierung}
\subsubsection{Schriftgröße}
Die Schriftgröße kann mit den folgenden Befehlen geändert werden:

\begin{table}[h]
    \centering
    \begin{tabular}{lll}
        \toprule
        \textbf{Befehl}             & \textbf{Größe}           & \textbf{Beispieltext} \\
        \midrule
        \textbackslash tiny         & Sehr klein               & {\tiny Text}          \\
        \textbackslash scriptsize   & Kleiner als normal       & {\scriptsize Text}    \\
        \textbackslash footnotesize & Kleine Schrift           & {\footnotesize Text}  \\
        \textbackslash small        & Etwas kleiner als normal & {\small Text}         \\
        \textbackslash normalsize   & Standardgröße            & {\normalsize Text}    \\
        \textbackslash large        & Etwas größer als normal  & {\large Text}         \\
        \textbackslash Large        & Noch größer              & {\Large Text}         \\
        \textbackslash LARGE        & Sehr groß                & {\LARGE Text}         \\
        \textbackslash huge         & Sehr große Schrift       & {\huge Text}          \\
        \textbackslash Huge         & Extrem große Schrift     & {\Huge Text}          \\
        \bottomrule
    \end{tabular}
    \caption{Schriftgrößen}
    \label{tab:schriftgroessen}
\end{table}

Durch Verwendung von geschweiften Klammern kann der Befehl auf einen Textbereich angewendet werden:
Beispiel: \textbf{\texttt{\{\textbackslash large großer Text\}}}

\subsubsection{Schriftstil}
Der Schriftstil kann mit den folgenden Befehlen geändert werden:

\begin{table}[h]
    \centering
    \begin{tabular}{lll}
        \toprule
        \textbf{Befehl}               & \textbf{Wirkung}                 & \textbf{Beispieltext} \\
        \midrule
        \textbackslash textbf\{\}     & Fett                             & \textbf{Text}         \\
        \textbackslash textit\{\}     & Kursiv                           & \textit{Text}         \\
        \textbackslash texttt\{\}     & Monospace (Typewriter)           & \texttt{Text}         \\
        \textbackslash underline\{\}  & Unterstrichen                    & \underline{Text}      \\
        \textbackslash emph\{\}       & Hervorgehoben (Kursiv o. Fett)   & \emph{Text}           \\
        \textbackslash textsc\{\}     & Kapitälchen (Small Caps)         & \textsc{Text}         \\
        \textbackslash textnormal\{\} & Normale Schrift                  & \textnormal{Text}     \\
        \textbackslash textsf\{\}     & Serifenlose Schrift (Sans-Serif) & \textsf{Text}         \\
        \textbackslash textsl\{\}     & Schräggestellt (Slanted)         & \textsl{Text}         \\
        \textbackslash textmd\{\}     & Mittlere Schriftstärke (Medium)  & \textmd{Text}         \\
        \textbackslash textup\{\}     & Aufrechte Schrift (Upright)      & \textup{Text}         \\
        \bottomrule
    \end{tabular}
    \caption{Schriftstil-Optionen}
    \label{tab:schriftstile}
\end{table}

Die Schriftstile können mit den Schriftgrößen kombiniert und mit anderen Schriftstilen überlagert werden.


\subsubsection{Schriftfarbe}
Die Schriftfarbe kann auf folgende drei Arten geändert werden:

\begin{itemize}
    \item \textbf{\texttt{\textbackslash textcolor\{red\}\{Roter Text\}}}
          \hspace{2.83cm} --- \hspace{1cm} \textcolor{red}{Roter Text}

    \item \textbf{\texttt{\textbackslash textcolor[rgb]\{0,0.5,1\}\{RGB-Farbe\}}}
          \hspace{1.35cm} --- \hspace{1cm} \textcolor[rgb]{0,0.5,1}{RGB-Farbe}

    \item \textbf{\texttt{\textbackslash textcolor[HTML]\{00FF00\}\{Hex-Farbcode\}}}
          \hspace{0.8cm} --- \hspace{1cm} \textcolor[HTML]{00FF00}{Hex-Farbcode}
\end{itemize}

Für diese Varianten wird das Paket \textbf{xcolor} benötigt. Darüberhinaus ist zu beachten, dass textcolor einen auf 1 normierten RGB-Wert erwartet.


\subsubsection{Rahmen und Boxen}
Textpassagen können mit folgenden Befehlen in Rahmen oder Boxen gesetzt werden:

\begin{table}[H]
    \centering
    \begin{tabular}{lll}
        \toprule
        \textbf{Befehl}                                  & \textbf{Wirkung}                 & \textbf{Darstellung}   \\
        \midrule
        \texttt{\textbackslash fbox\{Text\}}             & Rahmen um den Text               & \fbox{Text}            \\
        \texttt{\textbackslash framebox(45,10)\{Text\}}  & Box mit fester Breite und Höhe   & \framebox(45,10){Text} \\
        \texttt{\textbackslash colorbox\{cyan\}\{Text\}} & Farbige Box mit Hintergrundfarbe & \colorbox{cyan}{Text}  \\
        \bottomrule
    \end{tabular}
    \caption{Rahmen und Boxen}
    \label{tab:boxen}
\end{table}


\subsubsection{Spezialfälle}

\begin{table}[H]
    \centering
    \begin{tabular}{lll}
        \toprule
        \textbf{Funktion}       & \textbf{Befehl}                     & \textbf{Darstellung}             \\
        \midrule
        Text hochstellen        & \texttt{x\textasciicircum\{4\}}     & $x^{4}$                          \\
        Text tiefstellen        & \texttt{H\_\{2\}O}                  & $H_{2}O$                         \\
        Geschütztes Leerzeichen & \texttt{Hallo\textasciitilde Welt!} & Hallo~Welt! (Verhindert Umbruch) \\
        \bottomrule
    \end{tabular}
    \caption{Spezial Befehle zur Textformatierung}
    \label{tab:textformatierung}
\end{table}

Einige Zeichen, wie \textbf{\%}, \textbf{\#}, \textbf{\&}, \textbf{\_} und \textbf{\{} müssen in \LaTeX{} mit einem Backslash maskiert werden, um sie als Textzeichen zu verwenden (\textbf{\textbackslash\%}, \textbf{\textbackslash\#}, \textbf{\textbackslash\&}, \textbf{\textbackslash\_}, \textbf{\textbackslash\{}).

Für einige Zeichen gibt es auch spezielle Befehle, die eine Darstellung unabhängig von der Funktion des Zeichens ermöglichen. Beispiele sind \textbf{\textbackslash textbackslash} für den Backslash \textbf{\textbackslash}, \textbf{\textbackslash textasciitilde} für die Tilde \textbf{\textasciitilde} und \textbf{\textbackslash textasciicircum} für das Zirkumflex \textbf{\textasciicircum}.

\subsection{Abstandsformatierung}
\subsubsection{Vertikaler Abstand}
Der vertikale Abstand kann mit den folgenden Befehlen geändert werden:

\begin{table}[h]
    \centering
    \begin{tabular}{lp{10cm}}
        \toprule
        \textbf{Befehl}                           & \textbf{Wirkung}                                                      \\
        \midrule
        \texttt{\textbackslash vspace\{1cm\}}     & Fügt \textbf{1 cm} vertikalen Abstand ein (am Seitenanfang ignoriert) \\
        \texttt{\textbackslash vspace*\{1cm\}}    & Fügt \textbf{1 cm} vertikalen Abstand ein, auch am Seitenanfang       \\
        \texttt{\textbackslash addvspace\{1cm\}}  & Fügt \textbf{1 cm Abstand nur hinzu, wenn vorher keiner war}          \\
        \texttt{\textbackslash smallskip}         & Fügt \textbf{kleinen flexiblen Abstand} ein                           \\
        \texttt{\textbackslash medskip}           & Fügt \textbf{mittleren flexiblen Abstand} ein                         \\
        \texttt{\textbackslash bigskip}           & Fügt \textbf{großen flexiblen Abstand} ein                            \\
        \texttt{\textbackslash vfill}             & Füllt den \textbf{gesamten verbleibenden Platz} mit Abstand           \\
        \midrule
        \texttt{\textbackslash newline}           & Erzwingt einen \textbf{Zeilenumbruch} und beginnt eine neue Zeile     \\
        \texttt{\textbackslash newpage}           & Erzwingt einen \textbf{Seitenumbruch}                                 \\
        \texttt{\textbackslash clearpage}         & Erzwingt einen \textbf{Seitenumbruch} nach allen restlichen Gleitobj. \\
        \texttt{\textbackslash linebreak[prio]}   & Erzwingt einen \textbf{Zeilenumbruch} mit Priorität (1-4)             \\
        \texttt{\textbackslash nolinebreak[prio]} & Verhindert einen \textbf{Zeilenumbruch} mit Priorität (1-4)           \\
        \texttt{\textbackslash pagebreak[prio]}   & Erzwingt einen \textbf{Seitenumbruch} mit Priorität (1-4)             \\
        \texttt{\textbackslash nopagebreak[prio]} & Verhindert einen \textbf{Seitenumbruch} mit Priorität (1-4)           \\
        \bottomrule
    \end{tabular}
    \caption{vertikale Abstandsoptionen}
    \label{tab:vertikale_abstaende}
\end{table}

Eine weitere Möglichkeit, vertikalen Abstand zu erzeugen, ist die Verwendung von Leerzeilen. Diese erzeugen (genau) einen neuen Absatz und somit vertikalen Abstand.
Darüber hinaus kann \texttt{\textbackslash stretch\{X\}} als Argument an \texttt{\textbackslash vspace} übergeben werden, um flexiblen vertikalen Abstand zu erzeugen. Der Wert von \texttt{X} gibt dabei an, wie viel Teile (Verhältnis) des verfügbaren Platzes der Abstand einnehmen soll.

\subsubsection{Horizontaler Abstand}
Der horizontale Abstand kann mit den folgenden Befehlen geändert werden:

\begin{table}[h]
    \centering
    \begin{tabular}{lp{10cm}}
        \toprule
        \textbf{Befehl}                        & \textbf{Wirkung}                                                  \\
        \midrule
        \texttt{\textbackslash hspace\{1cm\}}  & Fügt \textbf{1 cm} horizontalen Abstand ein                       \\
        \texttt{\textbackslash hspace*\{1cm\}} & Fügt \textbf{1 cm} horizontalen Abstand ein, auch am Zeilenanfang \\
        \texttt{\textbackslash kern Xpt}       & Fügt \textbf{X pt Abstand} zwischen Zeichen ein                   \\
        \texttt{\textbackslash quad}           & Fügt \textbf{einfachen Tabulator-Abstand} (1 em) ein              \\
        \texttt{\textbackslash qquad}          & Fügt \textbf{doppelten Tabulator-Abstand} (2 em) ein              \\
        \texttt{\textbackslash enspace}        & Fügt \textbf{halben Leerzeichen-Abstand} (0.5 em) ein             \\
        \texttt{\textbackslash hfill}          & Füllt den \textbf{gesamten verfügbaren horizontalen Platz}        \\
        \midrule
        \texttt{\textbackslash !}              & Entfernt ein \textbf{kleines Leerzeichen} (negativer Abstand)     \\
        \texttt{\textbackslash ignorespaces}   & Entfernt \textbf{alle nachfolgenden Leerzeichen}                  \\
        \bottomrule
    \end{tabular}
    \caption{horizontale Abstandsoptionen}
    \label{tab:horizontale_abstaende}
\end{table}

Eine weitere Möglichkeit, horizontalen Abstand zu erzeugen, ist die Verwendung von Leerzeichen. Diese erzeugen (genau) ein Leerzeichen und somit horizontalen Abstand.
Darüber hinaus kann \texttt{\textbackslash stretch\{X\}} als Argument an \texttt{\textbackslash hspace} übergeben werden, um flexiblen horizontalen Abstand zu erzeugen. Der Wert von \texttt{X} gibt dabei an, wie viel Teile (Verhältnis) des verfügbaren Platzes der Abstand einnehmen soll.

Ein em ist die Breite des Großbuchstaben M in der aktuellen Schriftart und ein pt entspricht etwa 0.35 mm.

\subsubsection{Zentrierung}
In LaTeX können Texte und Objekte zentriert werden. Dafür gibt es die folgenden Befehle:
\begin{table}[H]
    \centering
    \begin{tabular}{lp{6cm}c}
        \toprule
        \textbf{Befehl} & \textbf{Funktion}                                              \\
        \midrule
        \texttt{\textbackslash begin\{center\} ... \textbackslash end\{center\}}
                        & Zentriert alles im Block (Text, Bilder, Tabellen, Formeln)     \\

        \texttt{\textbackslash centering}
                        & Zentriert innerhalb einer Umgebung (Tabellen, Bilder, Formeln) \\
        \bottomrule
    \end{tabular}
    \caption{Möglichkeiten zur Zentrierung}
    \label{tab:center_vs_centering}
\end{table}

\subsection{Einbinden von \LaTeX{}-Dateien}
\label{sec:einbinden_von_latex_dateien}
Häufig ist es notwendig und sinnvoll, den \LaTeX{}-Code in mehrere Dateien aufzuteilen. Dies kann z.B. zur Strukturierung des Dokuments, zur Wiederverwendung von Codeblöcken oder zur Auslagerung von z.B. Tabellen sinnvoll sein.

Für das Einbinden gibt es folgende Befehle:
\begin{table}[H]
    \centering
    \begin{tabular}{lccc}
        \toprule
        \textbf{Befehl}                                       & \textbf{Seitenumbruch} & \textbf{Nutzbar in Unterdateien} \\
        \midrule
        \texttt{\textbackslash input\{Pfad/zur/Datei.tex\}}   & Nein                   & Ja                               \\
        \texttt{\textbackslash include\{Pfad/zur/Datei.tex\}} & Ja                     & Nein                             \\

        \bottomrule
    \end{tabular}
    \caption{Einbindung von \LaTeX{}-Dateien}
    \label{tab:latex_einbindung}
\end{table}

Es ist zu beachten, dass der Befehl \texttt{\textbackslash include} nur im Hauptdokument (main.tex) und nicht in bereits eingebundenen Dateien verwendet werden kann.

\subsection{Dokumentenhierarchie}
Jedes \LaTeX{}-Dokument hat eine Hierarchie, die durch die folgende Befehle definiert wird:

\begin{table}[H]
    \centering
    \begin{tabular}{lll}
        \toprule
        \textbf{Befehl}                           & \textbf{Ebenen-Tiefe}       & \textbf{Verfügbar in} \\
        \midrule
        \texttt{\textbackslash part\{\}}          & Höchste Ebene               & \texttt{report, book} \\
        \texttt{\textbackslash chapter\{\}}       & Kapitel                     & \texttt{report, book} \\
        \texttt{\textbackslash section\{\}}       & Hauptabschnitt              & Alle Dokumentklassen  \\
        \texttt{\textbackslash subsection\{\}}    & Unterabschnitt              & Alle Dokumentklassen  \\
        \texttt{\textbackslash subsubsection\{\}} & geringer als Unterabschnitt & Alle Dokumentklassen  \\
        \texttt{\textbackslash paragraph\{\}}     & Absatz mit Überschrift      & Alle Dokumentklassen  \\
        \texttt{\textbackslash subparagraph\{\}}  & Noch kleiner (meist inline) & Alle Dokumentklassen  \\
        \bottomrule
    \end{tabular}
    \caption{Dokumentenhierarchie}
    \label{tab:dokumenten_hierarchie}
\end{table}

Die Nummerierung der Ebenen erfolgt automatisch bis zur Ebene \textbf{subsubsection} (X=3).
Mit dem Befehl \textbf{\texttt{\textbackslash setcounter\{secnumdepth\}\{X\}}} kann die Nummerierungstiefe angepasst werden. Das X entspricht der Nummerierungstiefe:

\begin{table}[h]
    \centering
    \begin{tabular}{cll}
        \toprule
        \textbf{Wert (X)} & \textbf{Maximal nummerierte Ebene}                        \\
        \midrule
        -1                & Keine Nummerierung                                        \\
        0                 & Nur \texttt{\textbackslash part\{\}}                      \\
        1                 & Bis \texttt{\textbackslash chapter\{\}}                   \\
        2                 & Bis \texttt{\textbackslash section\{\}}                   \\
        3                 & Bis \texttt{\textbackslash subsection\{\}} (Standardwert) \\
        4                 & Bis \texttt{\textbackslash subsubsection\{\}}             \\
        5                 & Bis \texttt{\textbackslash paragraph\{\}}                 \\
        6                 & Bis \texttt{\textbackslash subparagraph\{\}}              \\
        \bottomrule
    \end{tabular}
    \caption{Nummerierungstiefe mit \texttt{secnumdepth}}
    \label{tab:secnumdepth}
\end{table}

Die Hierarchie-Ebenen sind auch im Inhaltsverzeichnis sichtbar und kann dort mit \textbf{\texttt{\textbackslash setcounter\{tocdepth\}\{X\}}} angepasst werden (siehe auch \ref{sec:inhaltsverzeichnis}).

In manchen Fällen ist es erwünscht, eine Ebene nicht zu nummerieren und auch nicht im Inhaltsverzeichnis aufzuführen. Dies kann mit dem Befehl \textbf{\texttt{\textbackslash section*\{Überschrift\}}} erreicht werden. Der Stern (*) bewirkt, dass die Ebene nicht nummeriert wird.
Soll die Ebene zwar im Inhaltsverzeichnis erscheinen, aber nicht nummeriert werden, kann der Befehl \textbf{\texttt{\textbackslash addcontentsline\{toc\}\{section\}\{Überschrift\}}} nach der jeweiligen section mit (*) verwendet werden.

Soll im Inhaltsverzeichnis eine andere Überschrift als im Text erscheinen, kann der Befehl \textbf{\texttt{\textbackslash section[Kurztitel]\{ausführlicher Titel\}}} verwendet werden. Der Text in den eckigen Klammern wird im Inhaltsverzeichnis angezeigt, der Text in den geschweiften Klammern im Text.

In Berichten werden in der Regel nur die Ebenen \textbf{section} bis \textbf{subsubsection} verwendet:

\section*{1 \hspace{0.15cm} section}
\subsection*{1.1 \hspace{0.15cm} subsection}
\subsubsection*{1.1.1 \hspace{0.15cm} subsubsection}

