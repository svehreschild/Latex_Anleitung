\section{Wichtige Befehle}
\subsection{Kommentare}
Kommentare werden in \LaTeX{} mit einem Prozentzeichen eingeleitet. Alles, was nach dem Prozentzeichen in der selben Zeile steht, wird von \LaTeX{} ignoriert. Mehrzeilige Kommentare sind mit der \texttt{\textbf{comment}}-Umgebung möglich. Dafür wird das Paket \textbf{comment} benötigt. Funktional identisch ist der Befehl \textbf{\texttt{\textbackslash iffalse ... \textbackslash fi}}.

\begin{lstlisting}[language={[LaTeX]TeX}, basicstyle=\small\ttfamily]
% Dies ist ein einzeiliger Kommentar
    
\begin{comment}
Dies ist ein
langer Kommentarblock
\end{comment}

\iffalse
Dies ist ein
langer Kommentarblock
\fi
\end{lstlisting}

\subsection{Textformatierung}
\subsubsection{Schriftgröße}
Die Schriftgröße kann mit den folgenden Befehlen geändert werden:

\begin{table}[h]
    \centering
    \begin{tabular}{lll}
        \toprule
        \textbf{Befehl}             & \textbf{Größe}           & \textbf{Beispieltext} \\
        \midrule
        \textbackslash tiny         & Sehr klein               & {\tiny Text}          \\
        \textbackslash scriptsize   & Kleiner als normal       & {\scriptsize Text}    \\
        \textbackslash footnotesize & Kleine Schrift           & {\footnotesize Text}  \\
        \textbackslash small        & Etwas kleiner als normal & {\small Text}         \\
        \textbackslash normalsize   & Standardgröße            & {\normalsize Text}    \\
        \textbackslash large        & Etwas größer als normal  & {\large Text}         \\
        \textbackslash Large        & Noch größer              & {\Large Text}         \\
        \textbackslash LARGE        & Sehr groß                & {\LARGE Text}         \\
        \textbackslash huge         & Sehr große Schrift       & {\huge Text}          \\
        \textbackslash Huge         & Extrem große Schrift     & {\Huge Text}          \\
        \bottomrule
    \end{tabular}
    \caption{Schriftgrößen}
    \label{tab:schriftgroessen}
\end{table}

Durch Verwendung von geschweiften Klammern kann der Befehl auf einen eingeschränkten Textbereich angewendet werden:
Beispiel: \textbf{\texttt{\{\textbackslash large großer Text\}}}

\subsubsection{Schriftstil}
Der Schriftstil kann mit den folgenden Befehlen geändert werden:

\begin{table}[h]
    \centering
    \begin{tabular}{lll}
        \toprule
        \textbf{Befehl}               & \textbf{Wirkung}                 & \textbf{Beispieltext} \\
        \midrule
        \textbackslash textbf\{\}     & Fett                             & \textbf{Text}         \\
        \textbackslash textit\{\}     & Kursiv                           & \textit{Text}         \\
        \textbackslash texttt\{\}     & Monospace (Typewriter)           & \texttt{Text}         \\
        \textbackslash underline\{\}  & Unterstrichen                    & \underline{Text}      \\
        \textbackslash emph\{\}       & Hervorgehoben (Kursiv o. Fett)   & \emph{Text}           \\
        \textbackslash textsc\{\}     & Kapitälchen (Small Caps)         & \textsc{Text}         \\
        \textbackslash textnormal\{\} & Normale Schrift                  & \textnormal{Text}     \\
        \textbackslash textsf\{\}     & Serifenlose Schrift (Sans-Serif) & \textsf{Text}         \\
        \textbackslash textsl\{\}     & Schräggestellt (Slanted)         & \textsl{Text}         \\
        \textbackslash textmd\{\}     & Mittlere Schriftstärke (Medium)  & \textmd{Text}         \\
        \textbackslash textup\{\}     & Aufrechte Schrift (Upright)      & \textup{Text}         \\
        \bottomrule
    \end{tabular}
    \caption{Schriftstil-Optionen}
    \label{tab:schriftstile}
\end{table}

Die Schriftstile können mit den Schriftgrößen kombiniert und mit anderen Schriftstilen überlagert werden.
Es ist zu beachten, dass \textbf{\texttt{\textbackslash lowercase\{\}}} und \textbf{\texttt{\textbackslash uppercase\{\}}} zu Problemen mit Umlauten führen können, daher ist die Verwendung von \textbf{\texttt{\textbackslash MakeUppercase\{\}}} und \textbf{\texttt{\textbackslash MakeLowercase\{\}}} zu empfehlen.

\newpage

\subsubsection{Schriftfarbe}
\label{sec:schriftfarbe}
Die Schriftfarbe kann auf folgende Arten geändert werden:

\begin{itemize}
    \item \textbf{\texttt{\textbackslash textcolor\{red!80\}\{Roter Text\}}}
          \hspace{2.26cm} --- \hspace{1cm} \textcolor{red!80}{Roter Text}

    \item \textbf{\texttt{\textbackslash textcolor[rgb]\{0,0.5,1\}\{RGB-Farbe\}}}
          \hspace{1.35cm} --- \hspace{1cm} \textcolor[rgb]{0,0.5,1}{RGB-Farbe}

    \item \textbf{\texttt{\textbackslash textcolor[HTML]\{00FF00\}\{Hex-Farbcode\}}}
          \hspace{0.8cm} --- \hspace{1cm} \textcolor[HTML]{00FF00}{Hex-Farbcode}
\end{itemize}

Für diese Varianten wird das Paket \textbf{xcolor} benötigt. Darüber hinaus ist zu beachten, dass \textbf{\texttt{\textbackslash textcolor[rgb]\{...\}}} einen RGB-Wert im Intervall $[0,1]$ (statt $[0,255]$) erwartet.

Nach dem \textbf{\texttt{!}} kann ein Wert zwischen 0 und 100 angegeben werden, um die Deckkraft (Farbintensität) zu steuern.


\subsubsection{Rahmen und Boxen}
\label{sec:rahmen_und_boxen}
Textpassagen Bilder, Tabellen und andere ähnliche Objekte können mit folgenden Befehlen in Rahmen bzw. Boxen gesetzt werden:
\begin{table}[H]
    \centering
    \begin{tabular}{lll}
        \toprule
        \textbf{Befehl}                                            & \textbf{Wirkung}                          & \textbf{Darstellung}          \\
        \midrule
        \texttt{\textbackslash fbox\{Text\}}                       & Rahmen um den Text                        & \fbox{Text}                   \\
        \texttt{\textbackslash framebox(45,10)\{Text\}}            & Box mit fester Breite und Höhe            & \framebox(45,10){Text}        \\
        \texttt{\textbackslash colorbox\{cyan\}\{Text\}}           & Farbige Box mit Hintergrundfarbe (xcolor) & \colorbox{cyan}{Text}         \\
        \texttt{\textbackslash fcolorbox\{red\}\{yellow\}\{Text\}} & Farbige Box mit Rahmen (xcolor)           & \fcolorbox{red}{yellow}{Text} \\
        \texttt{\textbackslash shadowbox\{Text\}}                  & Box mit Schatten (fancybox)               & \shadowbox{Text}              \\
        \texttt{\textbackslash doublebox\{Text\}}                  & Box mit doppeltem Rahmen (fancybox)       & \doublebox{Text}              \\
        \texttt{\textbackslash ovalbox\{Text\}}                    & Ovale Box (fancybox)                      & \ovalbox{Text}                \\
        \bottomrule
    \end{tabular}
    \caption{Rahmen und Boxen}
    \label{tab:boxen}
\end{table}


\subsubsection{Spezialfälle}

\begin{table}[H]
    \centering
    \begin{tabular}{lll}
        \toprule
        \textbf{Funktion}       & \textbf{Befehl}                     & \textbf{Darstellung}             \\
        \midrule
        Text hochstellen        & \texttt{x\textasciicircum\{4\}}     & $x^{4}$                          \\
        Text tiefstellen        & \texttt{H\_\{2\}O}                  & $H_{2}O$                         \\
        Geschütztes Leerzeichen & \texttt{Hallo\textasciitilde Welt!} & Hallo~Welt! (Verhindert Umbruch) \\
        \bottomrule
    \end{tabular}
    \caption{Spezial Befehle zur Textformatierung}
    \label{tab:textformatierung}
\end{table}

Einige Zeichen, wie \textbf{\%}, \textbf{\#}, \textbf{\&}, \textbf{\_} und \textbf{\{} müssen in \LaTeX{} mit einem Backslash maskiert werden, um sie als Textzeichen zu verwenden (\textbf{\textbackslash\%}, \textbf{\textbackslash\#}, \textbf{\textbackslash\&}, \textbf{\textbackslash\_}, \textbf{\textbackslash\{}).

Für einige Zeichen gibt es auch spezielle Befehle, die eine Darstellung unabhängig von der Funktion des Zeichens ermöglichen. Beispiele sind \textbf{\textbackslash textbackslash} für den Backslash \textbf{\textbackslash}, \textbf{\textbackslash textasciitilde} für die Tilde \textbf{\textasciitilde} und \textbf{\textbackslash textasciicircum} für das Zirkumflex \textbf{\textasciicircum}.

\newpage

\subsection{Abstandsformatierung}
\subsubsection{Vertikaler Abstand}
Der vertikale Abstand kann mit den folgenden Befehlen geändert werden:

\begin{table}[h]
    \centering
    \begin{tabular}{lp{10cm}}
        \toprule
        \textbf{Befehl}                           & \textbf{Wirkung}                                                      \\
        \midrule
        \texttt{\textbackslash vspace\{1cm\}}     & Fügt \textbf{1 cm} vertikalen Abstand ein (am Seitenanfang ignoriert) \\
        \texttt{\textbackslash vspace*\{1cm\}}    & Fügt \textbf{1 cm} vertikalen Abstand ein, auch am Seitenanfang       \\
        \texttt{\textbackslash addvspace\{1cm\}}  & Fügt \textbf{1 cm Abstand nur hinzu, wenn vorher keiner war}          \\
        \texttt{\textbackslash smallskip}         & Fügt \textbf{kleinen flexiblen Abstand} ein                           \\
        \texttt{\textbackslash medskip}           & Fügt \textbf{mittleren flexiblen Abstand} ein                         \\
        \texttt{\textbackslash bigskip}           & Fügt \textbf{großen flexiblen Abstand} ein                            \\
        \texttt{\textbackslash vfill}             & Füllt den \textbf{gesamten verbleibenden Platz} mit Abstand           \\
        \midrule
        \texttt{\textbackslash newline}           & Erzwingt einen \textbf{Zeilenumbruch} und beginnt eine neue Zeile     \\
        \texttt{\textbackslash newpage}           & Erzwingt einen \textbf{Seitenumbruch}                                 \\
        \texttt{\textbackslash clearpage}         & Erzwingt einen \textbf{Seitenumbruch} nach allen restlichen Gleitobj. \\
        \texttt{\textbackslash linebreak[prio]}   & Erzwingt einen \textbf{Zeilenumbruch} mit Priorität (1-4)             \\
        \texttt{\textbackslash nolinebreak[prio]} & Verhindert einen \textbf{Zeilenumbruch} mit Priorität (1-4)           \\
        \texttt{\textbackslash pagebreak[prio]}   & Erzwingt einen \textbf{Seitenumbruch} mit Priorität (1-4)             \\
        \texttt{\textbackslash nopagebreak[prio]} & Verhindert einen \textbf{Seitenumbruch} mit Priorität (1-4)           \\
        \bottomrule
    \end{tabular}
    \caption{vertikale Abstandsoptionen}
    \label{tab:vertikale_abstaende}
\end{table}

Die Priorität \textbf{[prio]} gibt an, wie wichtig der Umbruch ist (4 ist der höchste Wert).

Eine weitere Möglichkeit, vertikalen Abstand zu erzeugen, ist die Verwendung von Leerzeilen. Diese erzeugen (genau) einen neuen Absatz und somit vertikalen Abstand.

Darüber hinaus kann \textbf{\texttt{\textbackslash stretch\{X\}}} als Argument an \textbf{\texttt{\textbackslash vspace}} übergeben werden, um flexiblen vertikalen Abstand zu erzeugen. Der Wert von \texttt{X} gibt dabei an, wie viel Teile (Verhältnis) des verfügbaren Platzes der Abstand einnehmen soll.

\subsubsection{Horizontaler Abstand}
Der horizontale Abstand kann mit den folgenden Befehlen geändert werden:

\begin{table}[h]
    \centering
    \begin{tabular}{lp{10cm}}
        \toprule
        \textbf{Befehl}                        & \textbf{Wirkung}                                                  \\
        \midrule
        \texttt{\textbackslash hspace\{1cm\}}  & Fügt \textbf{1 cm} horizontalen Abstand ein                       \\
        \texttt{\textbackslash hspace*\{1cm\}} & Fügt \textbf{1 cm} horizontalen Abstand ein, auch am Zeilenanfang \\
        \texttt{\textbackslash kern Xpt}       & Fügt \textbf{X pt Abstand} zwischen Zeichen ein                   \\
        \texttt{\textbackslash quad}           & Fügt \textbf{einfachen Tabulator-Abstand} (1 em) ein              \\
        \texttt{\textbackslash qquad}          & Fügt \textbf{doppelten Tabulator-Abstand} (2 em) ein              \\
        \texttt{\textbackslash enspace}        & Fügt \textbf{halben Leerzeichen-Abstand} (0.5 em) ein             \\
        \texttt{\textbackslash hfill}          & Füllt den \textbf{gesamten verfügbaren horizontalen Platz}        \\
        \midrule
        \texttt{\textbackslash !}              & Entfernt ein \textbf{kleines Leerzeichen} (negativer Abstand)     \\
        \texttt{\textbackslash ignorespaces}   & Entfernt \textbf{alle nachfolgenden Leerzeichen}                  \\
        \bottomrule
    \end{tabular}
    \caption{horizontale Abstandsoptionen}
    \label{tab:horizontale_abstaende}
\end{table}

Eine weitere Möglichkeit, horizontalen Abstand zu erzeugen, ist die Verwendung von Leerzeichen. Diese erzeugen (genau) ein Leerzeichen und somit horizontalen Abstand.

Darüber hinaus kann \textbf{\texttt{\textbackslash stretch\{X\}}} als Argument an \texttt{\textbackslash hspace} übergeben werden, um flexiblen horizontalen Abstand zu erzeugen. Der Wert von \texttt{X} gibt dabei an, wie viel Teile (Verhältnis) des verfügbaren Platzes der Abstand einnehmen soll.

Ein em ist die Breite des Großbuchstaben M in der aktuellen Schriftart und ein pt entspricht etwa 0.35 mm.

\subsubsection{Zentrierung}
In \LaTeX{} können Texte und Objekte zentriert werden. Dafür gibt es die folgenden Befehle:
\begin{table}[H]
    \centering
    \renewcommand{\arraystretch}{1.2}
    \begin{tabular}{lp{6cm}c}
        \toprule
        \textbf{Befehl} & \textbf{Funktion}                  \\
        \midrule
        \texttt{\textbackslash begin\{center\} ... \textbackslash end\{center\}}
                        & Zentriert alles im Block           \\

        \texttt{\textbackslash centering}
                        & Zentriert innerhalb einer Umgebung \\
        \bottomrule
    \end{tabular}
    \caption{Möglichkeiten zur Zentrierung}
    \label{tab:center_vs_centering}
\end{table}

\subsection{Einbinden von \LaTeX{}-Dateien}
\label{sec:einbinden_von_latex_dateien}
Häufig ist es notwendig und sinnvoll, den \LaTeX{}-Code in mehrere Dateien aufzuteilen. Dies kann zur Strukturierung des Dokuments, zur Wiederverwendung von Codeblöcken oder zur Auslagerung von z.B. Tabellen sinnvoll sein.

Für das Einbinden gibt es folgende Befehle:
\begin{table}[H]
    \centering
    \renewcommand{\arraystretch}{1.1}
    \begin{tabular}{lccc}
        \toprule
        \textbf{Befehl}                                       & \textbf{Seitenumbruch} \\
        \midrule
        \texttt{\textbackslash input\{Pfad/zur/Datei.tex\}}   & Nein                   \\
        \texttt{\textbackslash include\{Pfad/zur/Datei.tex\}} & Ja                     \\

        \bottomrule
    \end{tabular}
    \caption{Einbindung von \LaTeX{}-Dateien}
    \label{tab:latex_einbindung}
\end{table}

Es ist zu beachten, dass der Befehl \texttt{\textbackslash include} in bereits inkludierten Dateien fehleranfällig sein kann und daher vermieden werden sollte.

\subsection{Einbinden von PDF-Dateien}
\label{sec:einbinden_von_pdf_dateien}
PDF-Dateien können in \LaTeX{}-Dokumente eingebunden werden. Dafür wird das Paket \textbf{pdfpages} benötigt. Die Einbindung erfolgt mit dem Befehl \textbf{\texttt{\textbackslash includepdf\{Pfad/zur/Datei.pdf\}}}. Es können auch nur bestimmte Seiten eingebunden werden mit der Option \textbf{\texttt{\textbackslash includepdf[pages=\{...\}]\{Pfad/zur/Datei.pdf\}}}.

\newpage

\subsection{Dokumentenhierarchie}
Jedes \LaTeX{}-Dokument hat eine Hierarchie, die durch die folgende Befehle definiert wird:

\begin{table}[H]
    \centering
    \begin{tabular}{lll}
        \toprule
        \textbf{Befehl}                           & \textbf{Ebenen-Tiefe}       & \textbf{Verfügbar in} \\
        \midrule
        \texttt{\textbackslash part\{\}}          & Höchste Ebene               & \texttt{report, book} \\
        \texttt{\textbackslash chapter\{\}}       & Kapitel                     & \texttt{report, book} \\
        \texttt{\textbackslash section\{\}}       & Hauptabschnitt              & Alle Dokumentklassen  \\
        \texttt{\textbackslash subsection\{\}}    & Unterabschnitt              & Alle Dokumentklassen  \\
        \texttt{\textbackslash subsubsection\{\}} & geringer als Unterabschnitt & Alle Dokumentklassen  \\
        \texttt{\textbackslash paragraph\{\}}     & Absatz mit Überschrift      & Alle Dokumentklassen  \\
        \texttt{\textbackslash subparagraph\{\}}  & Noch kleiner (meist inline) & Alle Dokumentklassen  \\
        \bottomrule
    \end{tabular}
    \caption{Dokumentenhierarchie}
    \label{tab:dokumenten_hierarchie}
\end{table}

Die Nummerierung der Ebenen erfolgt automatisch bis zur Ebene \textbf{subsubsection} (X=3).
Mit dem Befehl \textbf{\texttt{\textbackslash setcounter\{secnumdepth\}\{X\}}} kann die Nummerierungstiefe angepasst werden. Das X entspricht der Nummerierungstiefe (siehe \autoref{tab:secnumdepth}).


Die Hierarchie-Ebenen sind auch im Inhaltsverzeichnis sichtbar und können dort mit \textbf{\texttt{\textbackslash setcounter\{tocdepth\}\{X\}}} angepasst werden (siehe auch \ref{sec:inhaltsverzeichnis}).

In manchen Fällen ist es erwünscht, eine Ebene nicht zu nummerieren und auch nicht im Inhaltsverzeichnis aufzuführen. Dies kann mit dem Befehl \textbf{\texttt{\textbackslash section*\{Überschrift\}}} erreicht werden.

Soll die Ebene zwar im Inhaltsverzeichnis erscheinen, aber nicht nummeriert werden, kann der Befehl \textbf{\texttt{\textbackslash addcontentsline\{toc\}\{section\}\{Überschrift\}}} nach der jeweiligen section mit * verwendet werden.

Falls im Inhaltsverzeichnis eine andere Überschrift als im Text erscheinen soll, kann der Befehl \textbf{\texttt{\textbackslash section[Kurztitel]\{ausführlicher Titel\}}} verwendet werden. Der Text in den eckigen Klammern wird im Inhaltsverzeichnis angezeigt, der Text in den geschweiften Klammern im Text als Überschrift.

In Berichten werden in der Regel nur die Ebenen \textbf{section} bis \textbf{subsubsection} verwendet. Für diese Hierarchieebenen folgen nun Beispiele:
\section*{1 \hspace{0.15cm} section}

\subsection*{1.1 \hspace{0.15cm} subsection}

\subsubsection*{1.1.1 \hspace{0.15cm} subsubsection}


\subsection{Querverweise (Label und Referenzen)}
\label{sec:querverweise}
\LaTeX{} bietet ein automatisches Querverweis-System, welches es ermöglicht, auf Kapitel, Abschnitte, Bilder, Tabellen, Formeln und andere Elemente im Dokument zu verweisen.
Dafür werden Labels an den gewünschten Stellen im Dokument gesetzt und an anderer Stelle mit Referenzen darauf verwiesen.

\underline{\textbf{Labels}} können mit dem Befehl \textbf{\texttt{\textbackslash label\{name\}}} gesetzt werden. Der Name kann frei gewählt werden, sollte aber eindeutig sein. Hilfreich ist die konventionelle Benennung, \textbf{fig:} für Bilder, \textbf{tab:} für Tabellen, \textbf{sec:} für Abschnitte und \textbf{eq:} für Formeln.

\underline{\textbf{Referenzen}} können mit dem Befehl \textbf{\texttt{\textbackslash ref\{name\}}} gesetzt werden. Der Name entspricht dem des Label, auf das verwiesen werden soll.

Mit dem Befehl \textbf{\texttt{\textbackslash pageref\{name\}}} kann auf die Seitenzahl des Labels verwiesen werden.

Der Befehl \textbf{\texttt{\textbackslash autoref\{name\}}} kann den \textbf{Typ des Labels} (Abschnitt, Tabelle, Abbildung, Gleichung) ausgeben. Allerdings ist das Paket \textbf{hyperref} erforderlich.

Soll der \textbf{Name des referenzierten Objektes} ausgegeben werden, kann dies mit \textbf{\texttt{\textbackslash nameref\{name\}}} erzielt werden.

Mit dem Befehl \textbf{\texttt{\textbackslash hyperref[name]\{eigener Kommentar\}}} kann ein Link auf das Label gesetzt werden. Dafür wird das Paket \textbf{hyperref} benötigt.

Mit \textbf{\texttt{\textbackslash href\{URL\}\{Text\}}} kann ein Link zu einer URL gesetzt werden.

In \autoref{tab:querverweise} wird dies durch ein Beispiel demonstriert:
\begin{table}[H]
    \centering
    \begin{tabular}{ll}
        \toprule
        \textbf{Befehl}                                                              & \textbf{Ergebnis}                                   \\
        \midrule
        \texttt{\textbackslash ref\{sec:querverweise\}}                              & \ref{sec:querverweise}                              \\
        \texttt{\textbackslash eqref\{sec:querverweise\}}                            & \eqref{sec:querverweise}                            \\
        \texttt{\textbackslash pageref\{sec:querverweise\}}                          & \pageref{sec:querverweise}                          \\
        \texttt{\textbackslash autoref\{sec:querverweise\}}                          & \autoref{sec:querverweise}                          \\
        \texttt{\textbackslash nameref\{sec:querverweise\}}                          & \nameref{sec:querverweise}                          \\
        \texttt{\textbackslash hyperref[sec:querverweise]\{siehe bei Querverweise\}} & \hyperref[sec:querverweise]{siehe bei Querverweise} \\
        \texttt{\textbackslash url\{https://www.google.de\}}                         & \url{https://www.google.de}                         \\
        \texttt{\textbackslash href\{https://www.google.de\}\{Google\}}              & \href{https://www.google.de}{Google}                \\
        \midrule
        \texttt{\textbackslash label\{sec:querverweise\}}                            & \multicolumn{1}{c}{-}                               \\
        \bottomrule
    \end{tabular}
    \caption{Übersicht über verschiedene Referenz- und Label-Befehle}
    \label{tab:querverweise}
\end{table}

Bei den Befehlen (Ausnahme \textbackslash eqref und \textbackslash href) kann ein \textbf{*} verwendet werden um den Hyperlink zu entfernen. Beispiel: \texttt{\textbackslash ref*\{sec:querverweise\}}


\subsection{Zitieren}
\label{sec:zitieren}
In \LaTeX{} können Zitate und Literaturverweise mit dem Befehl \textbf{\texttt{\textbackslash cite\{Schlüssel\}}} gesetzt werden. Der Schlüssel entspricht dem Namen des Eintrags in der Literaturdatenbank.
Für das Zitieren wird das Paket \textbf{cite} benötigt.

Es gibt verschiedene Zitierstile, die festgelegt werden können (siehe auch: \nameref{sec:literaturverzeichnis_erklärung}). Im folgenden wird der Stil \textbf{plain} verwendet.

\subsubsection{Einfache Zitate}

\begin{minipage}[c]{0.48\textwidth}
    \begin{lstlisting}[language={[LaTeX]TeX}]
Mustermann beschreibt dies in seiner Arbeit \cite{buch1}.
    \end{lstlisting}
\end{minipage}
\hfill
\begin{minipage}[c]{0.48\textwidth}
    Mustermann beschreibt dies in seiner Arbeit \cite{buch1}.
\end{minipage}


\subsubsection{Mehrere Quellen zitieren}

\begin{minipage}[c]{0.48\textwidth}
    \begin{lstlisting}[language={[LaTeX]TeX}]
Laut Mustermann und Doe \cite{buch1, artikel1} ist ...
    \end{lstlisting}
\end{minipage}
\hfill
\begin{minipage}[c]{0.48\textwidth}
    Laut Mustermann und Doe \cite{buch1, artikel1} ist ...
\end{minipage}


\subsubsection{Zitate mit Seitenangabe}

\begin{minipage}[c]{0.48\textwidth}
    \begin{lstlisting}[language={[LaTeX]TeX}]
Laut Mustermann \cite[S. 5]{buch1} ist ...
    \end{lstlisting}
\end{minipage}
\hfill
\begin{minipage}[c]{0.48\textwidth}
    Laut Mustermann \cite[S. 5]{buch1} ist ...
\end{minipage}


\subsubsection{Quellen nur im Literaturverzeichnis anzeigen}
Mit dem Befehl \textbf{\texttt{\textbackslash nocite\{Schlüssel\}}} können Einträge aus der Literaturdatenbank (literatur.bib) im Literaturverzeichnis angezeigt werden ohne im Text zitiert zu werden.
Die Verwendung von \textbf{\texttt{\textbackslash nocite\{*\}}} zeigt alle Einträge der Literaturdatenbank im Literaturverzeichnis an.


\subsection{Einheiten mit \texttt{siunitx}}

Das Paket \textbf{siunitx} ermöglicht die einfache und konsistente Darstellung von Einheiten in Texten und Formeln.
Werte mit Einheiten können entweder mit dem Befehl \textbf{\texttt{\textbackslash SI\{Wert\}\{\textbackslash Einheit\}}} oder direkt mit \textbf{\texttt{Wert\textbackslash si\{\textbackslash Einheit\}}} gesetzt werden.

\begin{minipage}[c]{0.6\textwidth}
    \begin{lstlisting}[language={[LaTeX]TeX}, lineskip=2pt]
8,9 \si{\micro\meter}               
45 \si{\degreeCelsius}              
5 \si{\milli\meter\per\second}      
5 \si{\milli\meter\div\second}      
5 \si{\frac{\milli\meter}{\second}} 
\SI{44444,2}{\giga\meter\per\minute}    
    \end{lstlisting}
\end{minipage}
\hfill
\begin{minipage}[c]{0.3\textwidth}
    \begin{itemize}[itemsep=2pt, label=$\rightarrow$]
        \item 8,9 \si{\micro\meter}
        \item 45 \si{\degreeCelsius}
        \item 5 \si{\milli\meter\per\second}
        \item 5 \si{\milli\meter\div\second}
        \item 5 \si{\frac{\milli\meter}{\second}}
        \item \SI{44444,2}{\giga\meter\per\minute}
    \end{itemize}
\end{minipage}

In der Präambel des Dokuments kann mit \texttt{\textbf{\textbackslash sisetup\{output-decimal-marker = \{,\}\}}} das Dezimaltrennzeichen global auf ein Komma und ggf. mit \textbf{\texttt{\textbackslash sisetup\{group-separator = \{.\}\}}} das Gruppentrennzeichen auf einen Punkt gesetzt werden.


\subsection{Eigene Befehle (Makros)}
In latex können eigene Befehle definiert werden, um häufig verwendete Befehle oder Texte zu vereinfachen. Dies kann mit dem Befehl \textbf{\texttt{\textbackslash newcommand\{\textbackslash befehl\}[Anzahl-Argumente]\{Definition\}}} erreicht werden.

Darüber hinaus kann mit \textbf{\texttt{\textbackslash renewcommand\{\textbackslash befehl\}[Anzahl-Argumente]\{Definition\}}} ein bereits existierender Befehl überschrieben werden und mit \newline \textbf{\texttt{\textbackslash providecommand\{\textbackslash befehl\}[Anzahl-Argumente]\{Definition\}}} wird der Befehl nur definiert, wenn er noch nicht existiert.

\subsubsection{Makros ohne Argumente}

\begin{minipage}[c]{0.65\textwidth}
    \begin{lstlisting}[language={[LaTeX]TeX}]
\newcommand{\meinText}{Das ist ein Beispiel.}
\meinText
        \end{lstlisting}
\end{minipage}
\hfill
\begin{minipage}[c]{0.3\textwidth}
    \newcommand{\meinText}{Das ist ein Beispiel.}
    \meinText
\end{minipage}

\subsubsection{Makros mit Argumenten}

\begin{minipage}[c]{0.65\textwidth}
    \begin{lstlisting}[language={[LaTeX]TeX}]
\newcommand{\quersumme}[3]{Die Summe aus #1, #2 und #3 ist:\newline\[ #1 + #2 + #3 \]}
\quersumme{5}{6}{9}
        \end{lstlisting}
\end{minipage}
\hfill
\begin{minipage}[c]{0.3\textwidth}
    \newcommand{\quersumme}[3]{Die Summe aus #1, #2 und #3 ist:\newline\[ #1 + #2 + #3 \]}
    \quersumme{5}{6}{9}
\end{minipage}