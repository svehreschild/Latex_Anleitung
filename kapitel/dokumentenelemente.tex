\section{Dokumentenelemente}
Neben dem Text gibt es noch weitere Elemente, die in einem Dokument vorkommen können. Dazu gehören unter anderem Bilder, Tabellen, Listen und Quellcode. In diesem Kapitel werden die wichtigsten Elemente vorgestellt.

\subsection{Listen}
In \LaTeX{} gibt es verschiedene Arten von Listen. Es folgen die wichtigsten Beispiele.

\subsubsection{Aufzählungen / ungeordnete Listen}

\begin{minipage}{0.58\textwidth}
    \begin{lstlisting}
\begin{itemize}
    \item Erster Punkt
    \item Zweiter Punkt
    \item Dritter Punkt
\end{itemize}
\end{lstlisting}
\end{minipage}
\hfill
\begin{minipage}{0.35\textwidth}
    \begin{itemize}
        \item Erster Punkt
        \item Zweiter Punkt
        \item Dritter Punkt
    \end{itemize}
\end{minipage}

\subsubsection{Nummerierungen / geordnete Listen}

\begin{minipage}{0.58\textwidth}
    \begin{lstlisting}
\begin{enumerate}
    \item Erster Punkt
    \item Zweiter Punkt
    \item Dritter Punkt
\end{enumerate}
\end{lstlisting}
\end{minipage}
\hfill
\begin{minipage}{0.35\textwidth}
    \begin{enumerate}
        \item Erster Punkt
        \item Zweiter Punkt
        \item Dritter Punkt
    \end{enumerate}
\end{minipage}

Falls die Nummerierung mit römischen Zahlen erfolgen soll, kann dies durch den Befehl \textbf{\texttt{\textbackslash begin\{enumerate\}[label=\textbackslash Roman*]}} erreicht werden. Analog dazu können für kleine römische Zahlen \textbf{\texttt{\textbackslash roman*}} und für Buchstaben \textbf{\texttt{\textbackslash alph*}} verwendet werden. Für diese Anpassungen wird das Paket \texttt{enumitem} benötigt.

\subsubsection{Definitionslisten}

\begin{minipage}{0.58\textwidth}
    \begin{lstlisting}
\begin{description}
    \item[Erster Punkt] Beschreibung 1
    \item[Zweiter Punkt] Beschreibung 2
    \item[Dritter Punkt] Beschreibung 3
\end{description}
\end{lstlisting}
\end{minipage}
\hfill
\begin{minipage}{0.35\textwidth}
    \begin{description}
        \item[Erster Punkt] Beschreibung 1
        \item[Zweiter Punkt] Beschreibung 2
        \item[Dritter Punkt] Beschreibung 3
    \end{description}
\end{minipage}

\subsubsection{Angepasste Listen}

\begin{minipage}{0.58\textwidth}
    \begin{lstlisting}
\begin{itemize}[label=$\rightarrow$]
    \item Erster Punkt
    \item Zweiter Punkt
    \item Dritter Punkt
\end{itemize}
\end{lstlisting}
\end{minipage}
\hfill
\begin{minipage}{0.35\textwidth}
    \begin{itemize}[label=$\rightarrow$]
        \item Erster Punkt
        \item Zweiter Punkt
        \item Dritter Punkt
    \end{itemize}
\end{minipage}

Neben $\rightarrow$ können auch andere Symbole oder Zeichenketten als Aufzählungszeichen verwendet werden. Es ist allerdings darauf zu achten ggf. (\$) für Objekte aus der Mathematikumgebung zu verwenden. Vorstellbar wären beispielsweise ein Haken $\checkmark$ (\$\textbackslash checkmark\$) oder ein Kreuz $\times$ (\$\textbackslash times\$). Es wird das Paket \texttt{enumitem} benötigt.

\subsubsection{Mehrspaltige Listen}

\begin{minipage}{0.50\textwidth}
    \begin{lstlisting}
\begin{multicols}{2}
\begin{itemize}
    \item Erster Punkt
    \item Zweiter Punkt
    \item Dritter Punkt
    \item Vierter Punkt
\end{itemize}
\end{multicols}
\end{lstlisting}
\end{minipage}
\hfill
\begin{minipage}{0.48\textwidth}
    \begin{multicols}{2}
        \begin{itemize}
            \item Erster Punkt
            \item Zweiter Punkt
            \item Dritter Punkt
            \item Vierter Punkt
        \end{itemize}
    \end{multicols}
\end{minipage}

Mit der Umgebung \texttt{multicols} können mehrspaltige Listen erstellt werden. Die Anzahl der Spalten wird als Argument in geschweiften Klammern übergeben. Es wird das Paket \texttt{multicol} benötigt.

\subsection{Gleichungen / Mathematikumgebungen}
In \LaTeX{} gibt es verschiedene Umgebungen, um mathematische Formeln zu setzen. Die wichtigsten Aspekte zu den mathematischen Umgebungen werden im Folgenden vorgestellt.

\subsubsection{Inline-Gleichungen}
In einem Text können mathematische Formeln direkt eingebettet werden. Dazu wird der Text in zwei Dollarzeichen (\$) eingeschlossen.
Beispiel: $a^2 + b^2 = c^2$ (Syntax: \texttt{\$a\^{}2 + b\^{}2 = c\^{}2\$})

\subsubsection{Abgesetzte Gleichungen}
Für abgesetzte Gleichungen gibt es die Umgebung \textbf{\texttt{equation}}. Diese setzt die Gleichung zentriert und nummeriert sie.
Wird in der \textbf{\texttt{equation}}-Umgebung ein \textbf{\texttt{\textbackslash label\{...\}}} Befehl verwendet, kann die Gleichung im Text referenziert werden (\autoref{eq:planck}) (siehe auch \ref{sec:querverweise}).
Alternativ kann auch die Kurzform \textbf{\texttt{\textbackslash [ ... \textbackslash ]}} verwendet werden, bei welcher jedoch keine Nummerierung erfolgt.

\begin{minipage}{0.5\textwidth}
    \begin{lstlisting}
\begin{equation}
    \label{eq:planck}
    E = h \cdot f
\end{equation}

\[
    E = h \cdot f
\]

\end{lstlisting}
\end{minipage}
\hfill
\begin{minipage}{0.5\textwidth}
    \begin{equation}
        \label{eq:planck}
        E = h \cdot f
    \end{equation}

    \[
        E = h \cdot f
    \]
\end{minipage}
Alternativ kann die Nummerierung auch unterdrückt werden, indem die \textbf{\texttt{equation*}}-Umgebung verwendet wird oder in dem der Befehl \textbf{\texttt{\textbackslash nonumber}} eingesetzt wird.

\subsubsection{Mehrzeilige Gleichungen}
Für mehrzeilige Gleichungen gibt es die \textbf{\texttt{align}}-Umgebung. Mit dem \textbf{\texttt{\&}}-Zeichen wird die Ausrichtung der Gleichungen festgelegt. Mit dem \textbf{\texttt{\textbackslash\textbackslash}}-Zeichen wird eine neue Zeile begonnen. Die Gleichungen werden an den \textbf{\texttt{\&}}-Zeichen ausgerichtet.

\begin{lstlisting}
\begin{align}
    \label{eq:trig}
    1                 & = \sin^2{(x)} + \cos^2{(x)}     \\
    \label{eq:euler}
    \exp{(j \cdot x)} & = \cos{(x)} + j \cdot \sin{(x)}
\end{align}
\end{lstlisting}

\begin{align}
    \label{eq:trig}
    1                 & = \sin^2{(x)} + \cos^2{(x)}     \\
    \label{eq:euler}
    \exp{(j \cdot x)} & = \cos{(x)} + j \cdot \sin{(x)}
\end{align}

Auch in den \textbf{\texttt{align}}-Umgebungen können die Gleichungen mit \textbf{\texttt{\textbackslash label\{...\}}} versehen werden, um sie im Text referenzieren zu können (\autoref{eq:trig} und \autoref{eq:euler}). Soll die Nummerierung einer einzelnen Gleichung unterdrückt werden, kann dies mit \textbf{\texttt{\textbackslash nonumber}} erreicht werden oder durch das Verwenden von \textbf{\texttt{\textbackslash begin\{align*\}}}.

In der \textbf{\texttt{equation}}-Umgebung kann die Unterumgebung \textbf{\texttt{aligned}} verwendet werden, um mehrzeilige Gleichungen mit dem \textbf{\texttt{\&}}-Zeichen auszurichten.

\subsubsection{Formatierung in Gleichungen}
In mathematischen Umgebungen können verschiedene Formatierungen vorgenommen werden. Dazu gehören unter anderem:

\begin{table}[h]
    \centering
    \renewcommand{\arraystretch}{1.3}
    \begin{tabular}{lll}
        \toprule
        \textbf{Formatierung}                & \textbf{LaTeX-Code}                                 & \textbf{Beispiel}        \\
        \midrule
        Fette Symbole (kann mehr als mathbf) & \texttt{\textbackslash bm\{\textbackslash varphi\}} & \( \bm{\varphi} \)       \\
        Fette Buchstaben                     & \texttt{\textbackslash mathbf\{x\}}                 & \( \mathbf{x} \)         \\
        Serifenlose Schrift                  & \texttt{\textbackslash mathsf\{x\}}                 & \( \mathsf{x} \)         \\
        Monospace (Typewriter)               & \texttt{\textbackslash mathtt\{x\}}                 & \( \mathtt{x} \)         \\
        Kursive Schrift                      & \texttt{\textbackslash mathit\{x\}}                 & \( \mathit{x} \)         \\
        Normale Schrift (Roman)              & \texttt{\textbackslash mathrm\{sin\}}               & \( \mathrm{sin} x \)     \\
        \midrule
        Durchgestrichen                      & \texttt{\textbackslash cancel\{x\}}                 & \( \cancel{x} \)         \\
        Unterstrichen                        & \texttt{\textbackslash underline\{x\}}              & \( \underline{x} \)      \\
        Überstrichen                         & \texttt{\textbackslash overline\{x\}}               & \( \overline{x} \)       \\
        \midrule
        Text in Mathemodus                   & \texttt{\textbackslash text\{Text\}}                & \( \text{Text} \)        \\
        \midrule
        Farbige Schrift (rot)                & \texttt{\textbackslash textcolor\{red\}\{x\}}       & \( \textcolor{red}{x} \) \\
        Farbige Gleichung (blau)             & \texttt{\textbackslash color\{blue\} x = 5}         & \( \color{blue} x = 5 \) \\
        \midrule
        Leerzeichen                          & \texttt{a b}                                        & \( a  b \)               \\
        Kleiner Abstand                      & \texttt{a\textbackslash ,b}                         & \( a\,b \)               \\
        Mittlerer Abstand                    & \texttt{a\textbackslash :b}                         & \( a\: b \)              \\
        Großer Abstand                       & \texttt{a\textbackslash ;b}                         & \( a\;b \)               \\
        Sehr großer Abstand                  & \texttt{a\textbackslash quad b}                     & \( a \quad b \)          \\
        Extra großer Abstand                 & \texttt{a\textbackslash qquad b}                    & \( a \qquad b \)         \\
        Negativer Abstand                    & \texttt{a\textbackslash!b}                          & \( a\!b \)               \\
        \bottomrule
    \end{tabular}
    \caption{Formatierungsoptionen in Mathematikumgebungen}
    \label{tab:math_formatierung}
\end{table}



\subsubsection{Mathematische Symbole}
In \LaTeX{} gibt es viele mathematische Rechenoperatoren und Symbole, die mit folgenden Befehlen in mathematische Umgebungen eingefügt werden können:

\begin{table}[H]
    \centering
    \renewcommand{\arraystretch}{1.3}
    \begin{tabular}{lll}
        \toprule
        \textbf{Operator}       & \textbf{LaTeX-Code}                                                 & \textbf{Beispiel}          \\
        \midrule
        Gleichheit              & \texttt{=}                                                          & $ a = b $                  \\
        Ungleichheit            & \texttt{\textbackslash neq}                                         & $ a \neq b $               \\
        Größer als              & \texttt{>}                                                          & $ a > b $                  \\
        Kleiner als             & \texttt{<}                                                          & $ a < b $                  \\
        Größer-gleich           & \texttt{\textbackslash geq}                                         & $ a \geq b $               \\
        Kleiner-gleich          & \texttt{\textbackslash leq}                                         & $ a \leq b $               \\
        Plus                    & \texttt{+}                                                          & $ a + b $                  \\
        Minus                   & \texttt{-}                                                          & $ a - b $                  \\
        Mal (×)                 & \texttt{\textbackslash times}                                       & $ a \times b $             \\
        Mal (Punkt)             & \texttt{\textbackslash cdot}                                        & $ a \cdot b $              \\
        Geteilt (Bruch)         & \texttt{\textbackslash frac\{a\}\{b\}}                              & $ \frac{a}{b} $            \\
        Geteilt (÷)             & \texttt{\textbackslash div}                                         & $ a \div b $               \\
        Prozent                 & \texttt{\%}                                                         & $ 50\% $                   \\
        Summenzeichen           & \texttt{\textbackslash sum\_\{i=1\}\^{}\{n\} ...}                   & $ \sum_{i=1}^{n} ... $     \\
        Produktzeichen          & \texttt{\textbackslash prod\_\{i=1\}\^{}\{n\} ...}                  & $ \prod_{i=1}^{n} ... $    \\
        Integral                & \texttt{\textbackslash int\_\{a\}\^{}\{b\} f(x) \textbackslash, dx} & $ \int_{a}^{b} f(x) \,dx $ \\
        Wurzel                  & \texttt{\textbackslash sqrt\{x\}}                                   & $ \sqrt{x} $               \\
        n-te Wurzel             & \texttt{\textbackslash sqrt[n]\{x\}}                                & $ \sqrt[n]{x} $            \\
        Logarithmus             & \texttt{\textbackslash log{(x)}}                                    & $ \log{(x)} $              \\
        Natürlicher Logarithmus & \texttt{\textbackslash ln{(x)}}                                     & $ \ln{(x)} $               \\
        Exponentialfunktion     & \texttt{e\^{}x}, \texttt{\textbackslash exp{(x)}}                   & $ e^x, \exp{(x)} $         \\
        Sinus                   & \texttt{\textbackslash sin{(x)}}                                    & $ \sin{(x)} $              \\
        Kosinus                 & \texttt{\textbackslash cos{(x)}}                                    & $ \cos{(x)} $              \\
        Tangens                 & \texttt{\textbackslash tan{(x)}}                                    & $ \tan{(x)} $              \\
        Cotangens               & \texttt{\textbackslash cot{(x)}}                                    & $ \cot{(x)} $              \\
        Arkussinus              & \texttt{\textbackslash arcsin{(x)}}                                 & $ \arcsin{(x)} $           \\
        Arkuskosinus            & \texttt{\textbackslash arccos{(x)}}                                 & $ \arccos{(x)} $           \\
        Arkustangens            & \texttt{\textbackslash arctan{(x)}}                                 & $ \arctan{(x)} $           \\
        Modulo                  & \texttt{a \textbackslash bmod b}                                    & $ a \bmod b $              \\
        Konjunktion (UND)       & \texttt{\textbackslash wedge}                                       & $ A \wedge B $             \\
        Disjunktion (ODER)      & \texttt{\textbackslash vee}                                         & $ A \vee B $               \\
        Negation                & \texttt{\textbackslash neg}                                         & $ \neg A $                 \\
        Implikation             & \texttt{\textbackslash Rightarrow}                                  & $ A \Rightarrow B $        \\
        Äquivalenz              & \texttt{\textbackslash Leftrightarrow}                              & $ A \Leftrightarrow B $    \\
        Natürliche Zahlen       & \texttt{\textbackslash mathbb\{N\}}                                 & \( \mathbb{N} \)           \\
        Ganze Zahlen            & \texttt{\textbackslash mathbb\{Z\}}                                 & \( \mathbb{Z} \)           \\
        Rationale Zahlen        & \texttt{\textbackslash mathbb\{Q\}}                                 & \( \mathbb{Q} \)           \\
        Reelle Zahlen           & \texttt{\textbackslash mathbb\{R\}}                                 & \( \mathbb{R} \)           \\
        Komplexe Zahlen         & \texttt{\textbackslash mathbb\{C\}}                                 & \( \mathbb{C} \)           \\
        \bottomrule
    \end{tabular}
    \caption{Wichtige Rechenoperatoren und mathematische Symbole}
    \label{tab:operatoren}
\end{table}

Darüber hinaus gibt es noch viele weitere mathematische Symbole, die in der VSC-Umgebung einfach über die \LaTeX{}-Erweiterung eingefügt werden können.

\subsubsection{Matritzen}
Matritzen können in \LaTeX{} mit der \textbf{\texttt{bmatrix}}-Umgebung erstellt werden. Die Spalten werden durch \textbf{\texttt{\&}} und die Zeilen durch \textbf{\texttt{\textbackslash\textbackslash}} getrennt.

\begin{minipage}{0.5\textwidth}
    \begin{lstlisting}
\begin{equation}
    \begin{bmatrix}
        1 & 2 & 3 \\
        4 & 5 & 6 \\
        7 & 8 & 9
    \end{bmatrix}
\end{equation}
\end{lstlisting}
\end{minipage}
\hfill
\begin{minipage}{0.5\textwidth}
    \begin{equation}
        \begin{bmatrix}
            1 & 2 & 3 \\
            4 & 5 & 6 \\
            7 & 8 & 9
        \end{bmatrix}
    \end{equation}
\end{minipage}

Anstelle der \textbf{\texttt{bmatrix}}-Umgebung können auch die Umgebungen \textbf{\texttt{pmatrix}} (runde Klammern), \textbf{\texttt{Bmatrix}} (geschweifte Klammern), \textbf{\texttt{vmatrix}} (einfache Striche) und \textbf{\texttt{Vmatrix}} (doppelte Striche) verwendet werden.

\subsubsection{Fallunterscheidungen}
Fallunterscheidungen können in \LaTeX{} mit der \textbf{\texttt{cases}}-Umgebung erstellt werden. Die einzelnen Fälle werden durch \textbf{\texttt{\textbackslash\textbackslash}} getrennt und mit \textbf{\texttt{\&}} ausgerichtet.

\begin{minipage}{0.5\textwidth}
    \begin{lstlisting}
\begin{equation}
    f(x) =
    \begin{cases}
        0 & \text{:\quad} x < 0 \\
        1 & \text{:\quad} x \geq 0
    \end{cases}
\end{equation}
\end{lstlisting}
\end{minipage}
\hfill
\begin{minipage}{0.5\textwidth}
    \begin{equation}
        f(x) =
        \begin{cases}
            0 & \text{:\quad} x < 0    \\
            1 & \text{:\quad} x \geq 0
        \end{cases}
    \end{equation}
\end{minipage}
