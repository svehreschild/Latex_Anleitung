\section{Dokumentenelemente}
Neben dem Text gibt es noch weitere Elemente, die in einem Dokument vorkommen können. Dazu gehören unter anderem Bilder, Tabellen, Listen und Quellcode. In diesem Kapitel werden die wichtigsten Elemente vorgestellt.

\subsection{Listen}
In \LaTeX{} gibt es verschiedene Arten von Listen. Es folgen die wichtigsten Beispiele.

\subsubsection{Aufzählungen / ungeordnete Listen}

\begin{minipage}{0.58\textwidth}
    \begin{lstlisting}
\begin{itemize}
    \item Erster Punkt
    \item Zweiter Punkt
    \item Dritter Punkt
\end{itemize}
\end{lstlisting}
\end{minipage}
\hfill
\begin{minipage}{0.35\textwidth}
    \begin{itemize}
        \item Erster Punkt
        \item Zweiter Punkt
        \item Dritter Punkt
    \end{itemize}
\end{minipage}

\subsubsection{Nummerierungen / geordnete Listen}

\begin{minipage}{0.58\textwidth}
    \begin{lstlisting}
\begin{enumerate}
    \item Erster Punkt
    \item Zweiter Punkt
    \item Dritter Punkt
\end{enumerate}
\end{lstlisting}
\end{minipage}
\hfill
\begin{minipage}{0.35\textwidth}
    \begin{enumerate}
        \item Erster Punkt
        \item Zweiter Punkt
        \item Dritter Punkt
    \end{enumerate}
\end{minipage}

Falls die Nummerierung mit römischen Zahlen erfolgen soll, kann dies durch den Befehl \textbf{\texttt{\textbackslash begin\{enumerate\}[label=\textbackslash Roman*]}} erreicht werden. Analog dazu können für kleine römische Zahlen \textbf{\texttt{\textbackslash roman*}} und für Buchstaben \textbf{\texttt{\textbackslash alph*}} verwendet werden. Für diese Anpassungen wird das Paket \texttt{enumitem} benötigt.

\subsubsection{Definitionslisten}

\begin{minipage}{0.58\textwidth}
    \begin{lstlisting}
\begin{description}
    \item[Erster Punkt] Beschreibung 1
    \item[Zweiter Punkt] Beschreibung 2
    \item[Dritter Punkt] Beschreibung 3
\end{description}
\end{lstlisting}
\end{minipage}
\hfill
\begin{minipage}{0.35\textwidth}
    \begin{description}
        \item[Erster Punkt] Beschreibung 1
        \item[Zweiter Punkt] Beschreibung 2
        \item[Dritter Punkt] Beschreibung 3
    \end{description}
\end{minipage}

\subsubsection{Angepasste Listen}

\begin{minipage}{0.58\textwidth}
    \begin{lstlisting}
\begin{itemize}[label=$\rightarrow$]
    \item Erster Punkt
    \item Zweiter Punkt
    \item Dritter Punkt
\end{itemize}
\end{lstlisting}
\end{minipage}
\hfill
\begin{minipage}{0.35\textwidth}
    \begin{itemize}[label=$\rightarrow$]
        \item Erster Punkt
        \item Zweiter Punkt
        \item Dritter Punkt
    \end{itemize}
\end{minipage}

Neben $\rightarrow$ können auch andere Symbole oder Zeichenketten als Aufzählungszeichen verwendet werden. Es ist allerdings darauf zu achten ggf. (\$) für Objekte aus der Mathematikumgebung zu verwenden. Vorstellbar beispielsweise wäre ein Haken $\checkmark$ (\$\textbackslash checkmark\$) oder ein Kreuz $\times$ (\$\textbackslash times\$). Es wird das Paket \texttt{enumitem} benötigt.

\subsubsection{Mehrspaltige Listen}

\begin{minipage}{0.50\textwidth}
    \begin{lstlisting}
\begin{multicols}{2}
\begin{itemize}
    \item Erster Punkt
    \item Zweiter Punkt
    \item Dritter Punkt
    \item Vierter Punkt
\end{itemize}
\end{multicols}
\end{lstlisting}
\end{minipage}
\hfill
\begin{minipage}{0.48\textwidth}
    \begin{multicols}{2}
        \begin{itemize}
            \item Erster Punkt
            \item Zweiter Punkt
            \item Dritter Punkt
            \item Vierter Punkt
        \end{itemize}
    \end{multicols}
\end{minipage}

Mit der Umgebung \texttt{multicols} können mehrspaltige Listen erstellt werden. Die Anzahl der Spalten wird als Argument in geschweiften Klammern übergeben. Es wird das Paket \texttt{multicol} benötigt.

\subsection{Gleichungen / Mathematikumgebungen}