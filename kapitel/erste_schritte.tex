\section{Erste Schritte}
\subsection{Einstellungen und Struktur}

Zunächst werden die Grundeinstellungen für das Layout des Dokuments festgelegt. Dazu wird eine LaTeX-Datei \textbf{main.tex} erstellt, in welcher die Dokumentenklasse, die verwendeten Pakete und die Layout-Einstellungen festgelegt werden. Außerdem sind dort das Deckblatt, die Kopf- und Fußzeilen und das Inhaltsverzeichnis zu definieren, sowie die einzelnen Kapitel einzubinden.

\subsubsection{Festlegen der Dokumentenklasse}
Mit dem Befehl \textbf{\texttt{\textbackslash documentclass[Option1, Option2, ...]\{Dokumentenklasse\}}} wird die Dokumentenklasse festgelegt außerdem können weitere Optionen angegeben werden.

Mögliche Dokumentenklassen sind:
\begin{table}[h]
    \centering
    \begin{tabular}{ll}
        \toprule
        \textbf{Klasse}  & \textbf{Beschreibung}                                                               \\
        \midrule
        \texttt{article} & Kurze Texte, wissenschaftliche Artikel, Berichte,                                   \\
                         & Unterstützt \textbackslash section\{...\}  aber keine \textbackslash chapter\{...\} \\
        \texttt{report}  & Längere Dokumente mit Kapiteln (z. B. Abschlussarbeiten)                            \\
                         & Unterstützt \textbackslash chapter\{...\}                                           \\
        \texttt{book}    & Bücher mit Kapiteln, Abschnitten und Teilen                                         \\
                         & Unterstützt \textbackslash part\{...\}                                              \\
        \texttt{letter}  & Briefe                                                                              \\

        \texttt{beamer}  & Präsentationen (wie PowerPoint)                                                     \\
        \bottomrule
    \end{tabular}
    \caption{Standard-Dokumentenklassen für \texttt{\textbackslash documentclass}}
    \label{tab:dokumentklassen}
\end{table}

Mögliche Optionen sind:
\begin{table}[h]
    \centering
    \begin{tabular}{ll}
        \toprule
        \textbf{Option}                        & \textbf{Beschreibung}                                              \\
        \midrule
        \texttt{...pt}                         & Schriftgröße (Standard: 10pt)                                      \\
        \texttt{a4paper, b5paper, letterpaper} & Papierformat (A4, B5, Letter, usw.)                                \\
        \texttt{onecolumn, twocolumn}          & Einspaltig, zweispaltig (Standard: onecolumn)                      \\
        \texttt{titlepage, notitlepage}        & Eigene Titelseite oder nicht    (Standard: notitlepage)            \\
        \texttt{twoside, oneside}              & Doppelseitiges oder einseitiges Layout (Bücher vs. Artikel)        \\
        \texttt{landscape}                     & Querformat                                                         \\
        \texttt{draft, final}                  & Entwurfsmodus (keine Bilder) oder finale Version (Standard: final) \\
        \texttt{fleqn}                         & Mathe-Formeln linksbündig (Standard: zentriert)                    \\
        \texttt{leqno, reqno}                  & Gleichungsnummern links oder rechts (`amsmath`) (Standard: reqno)  \\
        \bottomrule
    \end{tabular}
    \caption{Wichtige Optionen für \texttt{\textbackslash documentclass}}
    \label{tab:documentclass-options}
\end{table}


\subsubsection{Inkludieren von Paketen}
Der Befehl \textbf{\texttt{\textbackslash usepackage[Optionen]\{Paketname\}}} ermöglicht das Einbinden von Paketen, die zusätzliche Funktionen und Einstellungen bereitstellen.
Eine Übersicht an Paketen und deren Funktionen ist in der bereitgestellten main.tex Datei zu finden.

\subsubsection{Öffnen und Schließen eines Dokuments}
Das eigentliche Dokument wird mit dem Befehl \textbf{\texttt{\textbackslash begin\{document\}}} eröffnet und mit dem Befehl \textbf{\texttt{\textbackslash end\{document\}}} beendet.
Alle Anweisungen davor dienen lediglich der Vorbereitung des Dokuments. Alle Anweisungen danach werden ignoriert. Alle Inhalte der folgenden Unterpunkte sind innerhalb dieser beiden Befehle zu platzieren.

\subsubsection{Einbinden / Erstellen des Deckblatts}
In der Regel wird ein Deckblatt benötigt, welches Informationen wie den Titel, den Autor, das Datum und ein Logo enthält. Dieses kann entweder als PDF-Datei eingebunden, falls ein bereits erstelltes Deckblatt verwendet werden soll oder als neue LaTeX-Datei erstellt werden.
Das einbinden einer PDF-Datei erfolgt mit dem Befehl \textbf{\texttt{\textbackslash includepdf\{Pfad/zur/Datei.pdf\}}}. Für diesen Befehl wird das Paket \textbf{pdfpages} benötigt.
Alternativ kann das Deckblatt auch als LaTeX-Datei eingebunden werden, indem der Inhalt des Deckblatts in eine neue Datei geschrieben und mit dem Befehl \textbf{\texttt{\textbackslash input\{Pfad/zur/Datei.tex\}}} eingebunden wird. Hilfreich ist die Funktion \textbf{\texttt{\textbackslash begin\{titlepage\}}} und \textbf{\texttt{\textbackslash end\{titlepage\}}} um zwischen diesen Befehlen das Deckblatt zu erstellen.

\subsubsection{Anpassen der Kopf- und Fußzeilen}
Die Kopf- und Fußzeilen können mit dem Paket \textbf{fancyhdr} besonders elegant angepasst werden. Zunächst wird mit dem Befehl \textbf{\texttt{\textbackslash pagestyle\{fancy\}}} die Verwendung von individuellen Kopf- und Fußzeilen aktiviert. Anschließend werden mit den Befehlen \textbf{\texttt{\textbackslash fancyfoot\{\}}} und \textbf{\texttt{\textbackslash fancyhead\{\}}} die Standardeinstellungen für Fuß- und Kopfzeile gelöscht. Die Befehle \textbf{\texttt{\textbackslash fancyfoot[Position]\{Text\}}} und \textbf{\texttt{\textbackslash fancyhead[Position]\{Text\}}} ermöglichen das Hinzufügen von Text oder Bildern in die Fuß- oder Kopfzeile.

Als Position sind die folgenden Angaben möglich:
\begin{itemize}
    \item \textbf{C} - zentriert
    \item \textbf{L} - linksbündig
    \item \textbf{R} - rechtsbündig
    \item \textbf{C} / \textbf{L} / \textbf{R} in Kombination mit \textbf{E} oder \textbf{O} für gerade oder ungerade Seitenzahlen möglich
\end{itemize}

Ist eine Seitennummerierung in der Form Seite X von Y gewünscht, ist dies mit dem Befehl \textbf{\texttt{\textbackslash fancyfoot[C]\{Seite \textbackslash thepage$\thicksim$von$\thicksim$\textbackslash pageref\{LastPage\}\}}} möglich. Hierbei wird das Paket \\ \textbf{lastpage} benötigt.

Soll die Kopfzeile den aktuellen \textbackslash section\{\} Namen anzeigen, kann dies mit dem Befehl \\ \textbf{\texttt{\textbackslash fancyhead[R]\{\{\textbackslash leftmark\}\}}} erreicht werden.

Eine Trennlinie zwischen Fußzeile und Dokument, sowie zwischen Kopfzeile und Dokument kann mit den Befehlen \textbf{\texttt{\textbackslash renewcommand\{\textbackslash footrulewidth\}\{...pt\}}} und \textbf{\texttt{\textbackslash renewcommand\{\textbackslash headrulewidth\}\{...pt\}}} eingefügt oder mit 0pt gelöscht werden.

% •	\thispagestyle{empty} – Keine Kopf- oder Fußzeile auf der aktuellen Seite
% •	\thispagestyle{plain} – Normale Fußzeile ohne Kopfzeile


\newpage

\section{Fehlt noch:}


%kommentare
fett, kursiv
reffernzen
%\ und {} in anweisungen
% \text, texttt, textbf, textit, underline , emph , farbig mit \textcolor{color}{text}
%\input
pdf
stichpunkjte
quellcode
%Tabellen, Booktabs, Multicolumn, Multirow, Landscape
%\newcommand{cmd}{def}
einheiten mit siunitx
neue seite
%\hspace{} \vspace{} \vfill
%\\ für neue zeile