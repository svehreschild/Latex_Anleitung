\section{Erste Schritte}
\subsection{Einstellungen und Struktur}

Zunächst werden die Grundeinstellungen für das Layout des Dokuments festgelegt. Dazu wird eine \LaTeX{}-Datei \textbf{main.tex} erstellt, in welcher die Dokumentenklasse, die verwendeten Pakete und die Layout-Einstellungen festgelegt werden. Außerdem sind dort das Deckblatt, die Kopf- und Fußzeilen und das Inhaltsverzeichnis zu definieren, sowie die einzelnen Kapitel einzubinden.

\subsubsection{Festlegen der Dokumentenklasse}
Mit dem Befehl \textbf{\texttt{\textbackslash documentclass[Option1, Option2, ...]\{Dokumentenklasse\}}} wird die Dokumentenklasse festgelegt außerdem können weitere Optionen angegeben werden.

Mögliche Dokumentenklassen sind:
\begin{table}[h]
    \centering
    \begin{tabular}{ll}
        \toprule
        \textbf{Klasse}  & \textbf{Beschreibung}                                                               \\
        \midrule
        \texttt{article} & Kurze Texte, wissenschaftliche Artikel, Berichte,                                   \\
                         & Unterstützt \textbackslash section\{...\}  aber keine \textbackslash chapter\{...\} \\
        \texttt{report}  & Längere Dokumente mit Kapiteln (z. B. Abschlussarbeiten)                            \\
                         & Unterstützt \textbackslash chapter\{...\}                                           \\
        \texttt{book}    & Bücher mit Kapiteln, Abschnitten und Teilen                                         \\
                         & Unterstützt \textbackslash part\{...\}                                              \\
        \texttt{letter}  & Briefe                                                                              \\

        \texttt{beamer}  & Präsentationen (wie PowerPoint)                                                     \\
        \bottomrule
    \end{tabular}
    \caption{Standard-Dokumentenklassen für \texttt{\textbackslash documentclass}}
    \label{tab:dokumentklassen}
\end{table}

Mögliche Optionen sind:
\begin{table}[h]
    \centering
    \begin{tabular}{ll}
        \toprule
        \textbf{Option}                        & \textbf{Beschreibung}                                              \\
        \midrule
        \texttt{...pt}                         & Schriftgröße (Standard: 10pt)                                      \\
        \texttt{a4paper, b5paper, letterpaper} & Papierformat (A4, B5, Letter, usw.)                                \\
        \texttt{onecolumn, twocolumn}          & Einspaltig, zweispaltig (Standard: onecolumn)                      \\
        \texttt{titlepage, notitlepage}        & Eigene Titelseite oder nicht    (Standard: notitlepage)            \\
        \texttt{twoside, oneside}              & Doppelseitiges oder einseitiges Layout (Bücher vs. Artikel)        \\
        \texttt{landscape}                     & Querformat                                                         \\
        \texttt{draft, final}                  & Entwurfsmodus (keine Bilder) oder finale Version (Standard: final) \\
        \texttt{fleqn}                         & Mathe-Formeln linksbündig (Standard: zentriert)                    \\
        \texttt{leqno, reqno}                  & Gleichungsnummern links oder rechts (`amsmath`) (Standard: reqno)  \\
        \bottomrule
    \end{tabular}
    \caption{Wichtige Optionen für \texttt{\textbackslash documentclass}}
    \label{tab:documentclass-options}
\end{table}


\subsubsection{Inkludieren von Paketen}
Der Befehl \textbf{\texttt{\textbackslash usepackage[Optionen]\{Paketname\}}} ermöglicht das Einbinden von Paketen, die zusätzliche Funktionen und Einstellungen bereitstellen.
Eine Übersicht an Paketen und deren Funktionen ist in der bereitgestellten main.tex Datei zu finden.

\subsection{Öffnen und Schließen eines Dokuments}
Das eigentliche Dokument wird mit dem Befehl \textbf{\texttt{\textbackslash begin\{document\}}} eröffnet und mit dem Befehl \textbf{\texttt{\textbackslash end\{document\}}} beendet.
Alle Anweisungen davor dienen lediglich der Vorbereitung des Dokuments. Alle Anweisungen danach werden ignoriert. Alle Inhalte der folgenden Unterpunkte sind innerhalb dieser beiden Befehle zu platzieren.

\subsection{Einbinden / Erstellen des Deckblatts}
In der Regel wird ein Deckblatt benötigt, welches Informationen wie den Titel, den Autor, das Datum und ein Logo enthält. Dieses kann entweder als PDF-Datei eingebunden, falls ein bereits erstelltes Deckblatt verwendet werden soll oder als neue \LaTeX{}-Datei erstellt werden.
Das einbinden einer PDF-Datei erfolgt mit dem Befehl \textbf{\texttt{\textbackslash includepdf\{Pfad/zur/Datei.pdf\}}}. Für diesen Befehl wird das Paket \textbf{pdfpages} benötigt (siehe auch \ref{sec:einbinden_von_pdf_dateien}).
Alternativ kann das Deckblatt auch als \LaTeX{}-Datei eingebunden werden, indem der Inhalt des Deckblatts in eine neue Datei geschrieben und mit dem Befehl \textbf{\texttt{\textbackslash input\{Pfad/zur/Datei.tex\}}} eingebunden wird. Hilfreich ist die Funktion \textbf{\texttt{\textbackslash begin\{titlepage\}}} und \textbf{\texttt{\textbackslash end\{titlepage\}}} um zwischen diesen Befehlen das Deckblatt zu erstellen. In dem Ordner \textbf{kapitel} sind zwei Beispieldateien zu finden.

\subsection{Anpassen der Kopf- und Fußzeilen}
Die Kopf- und Fußzeilen können mit dem Paket \textbf{fancyhdr} besonders elegant angepasst werden. Zunächst wird mit dem Befehl \textbf{\texttt{\textbackslash pagestyle\{fancy\}}} die Verwendung von individuellen Kopf- und Fußzeilen aktiviert. Anschließend werden mit den Befehlen \textbf{\texttt{\textbackslash fancyfoot\{\}}} und \textbf{\texttt{\textbackslash fancyhead\{\}}} die Standardeinstellungen für Fuß- und Kopfzeile gelöscht. Die Befehle \textbf{\texttt{\textbackslash fancyfoot[Position]\{Text\}}} und \textbf{\texttt{\textbackslash fancyhead[Position]\{Text\}}} ermöglichen das Hinzufügen von Text oder Bildern in die Fuß- oder Kopfzeile.

Als Position sind die folgenden Angaben möglich:
\begin{itemize}
  \item \textbf{C} - zentriert
  \item \textbf{L} - linksbündig
  \item \textbf{R} - rechtsbündig
  \item \textbf{C} / \textbf{L} / \textbf{R} in Kombination mit \textbf{E} oder \textbf{O} für gerade oder ungerade Seitenzahlen möglich
\end{itemize}

Ist eine Seitennummerierung in der Form Seite X von Y gewünscht, kann dies mit dem Befehl \textbf{\texttt{\textbackslash fancyfoot[C]\{Seite \textbackslash thepage$\thicksim$von$\thicksim$\textbackslash pageref\{LastPage\}\}}} ermöglicht werden. Hierbei wird das Paket \textbf{lastpage} benötigt.

Soll die Kopfzeile den aktuellen \textbackslash section\{\} Namen anzeigen, kann dies mit dem Befehl \\ \textbf{\texttt{\textbackslash fancyhead[R]\{\{\textbackslash leftmark\}\}}} erreicht werden.

Eine Trennlinie zwischen Fußzeile und Dokument, sowie zwischen Kopfzeile und Dokument kann mit den Befehlen \textbf{\texttt{\textbackslash renewcommand\{\textbackslash footrulewidth\}\{...pt\}}} und \textbf{\texttt{\textbackslash renewcommand\{\textbackslash headrulewidth\}\{...pt\}}} eingefügt oder mit 0pt gelöscht werden.

Manchmal ist es gewünscht auf einer Seite keine Kopf- und/oder Fußzeile zu haben. Dies ist möglich mit:
\begin{itemize}
  \item \textbf{\texttt{\textbackslash thispagestyle\{empty\}}} - Keine Kopf- oder Fußzeile
  \item \textbf{\texttt{\textbackslash thispagestyle\{plain\}}} - Normale Fußzeile ohne Kopfzeile
\end{itemize}

\subsection{Erstellen des Inhaltsverzeichnisses (und weiterer Verzeichnisse)}
\label{sec:inhaltsverzeichnis}
Das Inhaltsverzeichnis wird mit dem Befehl \textbf{\texttt{\textbackslash tableofcontents}} erstellt. Die Tiefe des Inhaltsverzeichnis kann mit dem Befehl \textbf{\texttt{\textbackslash setcounter\{tocdepth\}\{X\}}} festgelegt werden. Dabei steht X für die \underbar{maximale Tiefe} des Inhaltsverzeichnisses und kann folgende Werte annehmen:

\begin{itemize}
  \item \textbf{0} - Nur \texttt{\textbackslash part\{\}} (nur in \texttt{book} oder \texttt{report})
  \item \textbf{1} - \texttt{\textbackslash part\{\}} und \texttt{\textbackslash chapter\{\}} (nur in \texttt{book} oder \texttt{report})
  \item \textbf{2} - \texttt{\textbackslash chapter\{\}} und \texttt{\textbackslash section\{\}}
  \item \textbf{3} - \texttt{\textbackslash section\{\}} und \texttt{\textbackslash subsection\{\}}
  \item \textbf{4} - \texttt{\textbackslash subsection\{\}} und \texttt{\textbackslash subsubsection\{\}}
  \item \textbf{5} - \texttt{\textbackslash subsubsection\{\}} und \texttt{\textbackslash paragraph\{\}}
  \item \textbf{6} - \texttt{\textbackslash paragraph\{\}} und \texttt{\textbackslash subparagraph\{\}}
\end{itemize}

Für die Anwendung dieses Befehls ist darauf zu achten, dass der Befehl \textbf{\texttt{\textbackslash setcounter\{tocdepth\}\{X\}}} vor dem Befehl \textbf{\texttt{\textbackslash tableofcontents}} platziert werden muss.

Sollen weitere Inhaltsverzeichnisse für Abbildungen, Tabellen, etc. erstellt werden, so kann dies mit den Befehlen \textbf{\texttt{\textbackslash listoffigures}}, \textbf{\texttt{\textbackslash listoftables}}, etc. erreicht werden.


Darüber hinaus sind weitere Einstellungen mit den Paketen \textbf{tocloft}, \textbf{titletoc} und \textbf{minitoc} möglich.

\subsection{Einbinden von Kapiteln}
Besonders bei langen und umfangreichen Dokumenten ist es sinnvoll, die einzelnen Kapitel in separaten Dateien zu erstellen und diese dann in das Hauptdokument einzubinden. Dies geschieht mit dem Befehl \textbf{\texttt{\textbackslash input\{Pfad/zur/Datei.tex\}}}. Dabei ist darauf zu achten, dass der Pfad relativ zum Hauptdokument angegeben wird.
Ich persönlich strukturiere meine \LaTeX{}-Dateien in einem eigenen Ordner (kapitel) und binde sie in einer dafür vorgesehenen Datei (uebersicht.tex) ein, welche in der main.tex Datei eingefügt wird. Somit muss die main.tex Datei nach einmaliger Einrichtung nicht mehr verändert werden.

Mehr zu dem Inkludieren von \LaTeX{}-Dateien ist in Kapitel \ref{sec:einbinden_von_latex_dateien} zu finden.


\subsection{Literaturverzeichnis}
\label{sec:literaturverzeichnis_erklärung}
In \LaTeX{} kann ein \hyperref[sec:literaturverzeichnis]{Literaturverzeichnis} mit Hilfe von BibTeX erstellt werden.
Zunächst werden die Literaturangaben in einer \textbf{\texttt{.bib}}-Datei (literatur.bib) gespeichert . Vorzugsweise im gleichen Ordner wie die Hauptdatei.

Es folgt ein Beispiel für eine \textbf{\texttt{literatur.bib}}-Datei:

\begin{lstlisting}[language={[LaTeX]TeX}, caption={Beispiel für eine \texttt{.bib}-Datei}]
@book{buch1,
  author    = {Anna Beispiel and Bernd Mustermann and Carla Demo},
  title     = {Fortgeschrittene LaTeX-Techniken},
  publisher = {Beispiel-Verlag},
  year      = {2018}
}

@article{artikel1,
  author  = {John Doe},
  title   = {Eine neue Theorie zur Typografie},
  journal = {Journal fur Wissenschaftliche Typografie},
  volume  = {12},
  number  = {3},
  pages   = {45-67},
  year    = {2015}
}

@inproceedings{konferenz1,
  author    = {Jane Smith and Paul Example},
  title     = {Typografische Verbesserungen mit LaTeX},
  booktitle = {Proceedings der Konferenz fur Dokumentenverarbeitung},
  pages     = {123-130},
  year      = {2022}
}

@misc{web1,
  author    = {Overleaf},
  title     = {LaTeX-Wiki},
  year      = {2025},
  note      = {URL: \url{https://www.overleaf.com/learn/latex} (abgerufen am 10. Marz 2025)}
}

@phdthesis{thesis1,
  author  = {Alex Student},
  title   = {Untersuchungen zur mathematischen Satztechnik},
  school  = {Universitat Beispielstadt},
  year    = {2023}
}
\end{lstlisting}

Zunächst wird der Stil des Literaturverzeichnisses mit dem Befehl \textbf{\texttt{\textbackslash bibliographystyle\{Stil\}}} festgelegt.
Es stehen folgende Stile zur Auswahl:
\begin{itemize}
  \item \textbf{plain} – Alphabetische Sortierung nach Autoren (Standardstil).
  \item \textbf{unsrt} – Reihenfolge der Zitierung im Text (nicht alphabetisch).
  \item \textbf{alpha} – Alphabetische Sortierung nach Autoren, aber mit Kurzreferenzen wie `[Ein05]`.
  \item \textbf{abbrv} – Wie `plain`, aber mit abgekürzten Vornamen und Titeln.
  \item \textbf{apalike} – Ähnlich wie APA-Zitierstil (Autor-Jahr statt Nummerierung).
\end{itemize}

Anschließend wird die \textbf{\texttt{.bib}}-Datei mit dem Befehl \textbf{\texttt{\textbackslash bibliography\{Pfad/zur/Datei\}}} eingebunden.

Nun können im Text Zitate eingefügt werden (siehe auch: \nameref{sec:zitieren}).

Damit alle Einträge aus der literatur.bib-Datei im Literaturverzeichnis erscheinen, kann der Befehl \textbf{\texttt{\textbackslash nocite\{*\}}} verwendet werden.